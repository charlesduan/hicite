%%
%% \iffalse filename: nondocrefs.dtx \fi
%%
%<*doc>
\input driver
\thisis{nondocrefs}{Non-Document References}

These references and reference types are not associated with documents or
authorities, but serve other purposes in this package's system.


\input docparams


%</doc>
\hi@newcite\defabbrev{An abbreviation for a name}{%
    \hi@ifset\hi@kv@name{%
        \hi@replacethe\hi@kv@name{\def\hi@kv@name}{}%
        \@expandarg\hi@abbrev@name{\hi@kv@name}{\def\hi@kv@aname}%
    }{%
        \PackageError\hi@pkgname{%
            For abbreviation #1, no name given%
        }{You must provide the name}%
    }%
    \hi@ifset\hi@kv@inline{%
        \hi@replacethe\hi@kv@inline{\def\hi@kv@inline}{}%
        \@expand{\hi@short@register{#1}}\hi@kv@inline i{y}%
        %
        % Unless the user has explicitly rejected it, tack on a parenthetical of
        % the short name to the full citation form.
        %
    }{%
        \PackageError\hi@pkgname{%
            For abbreviation #1, no inline given%
        }{You must provide the inline form}%
    }%
    \hi@newcite@form{fc}{#1}{%
        \hi@citeguts{%
            \noexpand\if@hi@inline
                \noexpand\@capnext\hi@kv@name
            \noexpand\else
                \noexpand\@capnext\hi@kv@aname
            \noexpand\fi
            \hi@ifset\hi@kv@noinlineparen{}{%
                \hi@short@maybeshow{#1}{\space
                    {\noexpand\@hi@adjectivetrue(``\hi@kv@inline'')}%
                    \noexpand\@%
                }%
            }%
        }%
        \hi@invis@inline
    }%
    \hi@newcite@form{sc}{#1}{%
        \hi@citeguts{\noexpand\@capnext\hi@kv@inline}%
        \noexpand\@%
        \hi@short@use{#1}%
        \hi@invis@inline
    }%
}
%<*doc>
The full citation form produces the unabbreviated name and a definitional
parenthetical, and the short citation form produces the abbreviated name. This
reference type should only be used in inline citations. \unskip\footnote{In
future versions, it is hoped that named abbreviations can be used as
institutional author titles.} The abbreviation reference type makes use of the
short name management features of this package described in \sec{short}. If an
abbreviation is used only once, the parenthetical is automatically suppressed.
It also manages use of ``The'' at the beginning of names correctly, in
accordance with |\adjective| as described in \sec{iface}. Thus, if the following
is defined:
\begin{demo}
|\defabbrev{fda}{|\\
|    name=the Food and Drug Administration,|\\
|    inline=the FDA|\\
|}|
\end{demo}
then the following will be produced:
\begin{demo}
\begin{tabular}{lp{0.4\textwidth}}
\textbf{Input} & \textbf{Output} \\
\hline
|\inline{fda}| (used only once) & the Food and Drug Administration \\
|\inline{fda}| & the Food and Drug Administration (``FDA'') \\
|\inline{fda}| & the FDA \\
|\adjective{fda} regulations| & FDA regulations \\
\end{tabular}
\end{demo}

This reference type produces invisible inlines (see \sec{iface-invisible}) so
that it does not affect \emph{id.}\ citations surrounding it.

\keyparameters{
    {name}{The unabbreviated name. Include ``The'' at the beginning if that is
    how the full name is used in ordinary speech.}
    {inline}{The short or abbreviated name. Include ``The'' at the beginning if
    that is how the abbreviation is used in ordinary speech.}
}
%</doc>
%
\hi@newcite\defalias{An alias for a citation reference}{%
    \hi@ifset\hi@kv@citation{}{%
        \hi@ifset\hi@kv@rep{%
            \let\hi@kv@citation\hi@kv@rep
        }{%
            \PackageError\hi@pkgname{%
                In defining alias #1, you must provide citation or rep%
            }{Add the requested parameter}%
        }%
    }%
    %
    % The full citation form should never be used
    %
    \hi@newcite@form{fc}{#1}{%
        \noexpand\PackageError\noexpand\hi@pkgname{%
            Reference #1 is an alias and should never be cited%
        }{Something is wrong with the package}%
    }%
    %
    % The modification macro should set the volume and reference name
    %
    \hi@newcite@form{mc}{#1}{%
        \def\noexpand\@this@case{\hi@kv@citation}%
        \hi@ifset\hi@kv@vol{%
            \noexpand\ifx\noexpand\@this@vol\relax
                \def\noexpand\@this@vol{\hi@kv@vol}%
            \noexpand\fi
        }{}%
        \hi@ifset\hi@kv@page{%
            \noexpand\ifx\noexpand\@this@page\relax
                \def\noexpand\@this@orig@page{\expandonce\hi@kv@page}%
                \def\@this@page{\@format@page@macro\hi@kv@page}%
            \noexpand\fi
        }{}%
    }%
    %
    % The page number macro should combine page numbers and set \@this@orig@page
    % and \@this@page appropriately
    %
    \hi@newcite@form{pc}{#1}##1{%
        \hi@ifset\hi@kv@page{%
            \noexpand\hi@pages@join{\expandonce\hi@kv@page}{##1}%
        }{%
            \noexpand\protected@edef\noexpand\@this@page{%
                \noexpand\@format@pageno{##1}%
            }%
        }%
    }%
}
%<*doc>

Aliases a reference name to another. The main advantage is that the alias can
have a default volume and pin cite. For example, consider the alias:
\begin{demo}
|\defalias{sec-5}{|\\
|    name=ftc-act,|\\
|    page=S 5|\\
|}|
\end{demo}
with |ftc-act| defined as the relevant statute. Now, the following two are
equivalent:
\begin{demo}
|\sentence{sec-5}.|\\
|\sentence{ftc-act at S 5}.|
\end{demo}
Moreover, pin cites given for the alias in citations use the pin cite joining
algorithm given in \sec{pages-join}. Thus:
\begin{demo}
|\sentence{sec-5 at /a}.|
\end{demo}
will cite \textsection~5(a) of the statute.

\keyparameters{%
    {citation, rep}{Reference name to be aliased.}
    {page}{Pin cite to be used as the default pin cite for the alias.}
}

%</doc>
%
\newcount\hi@anonselfcitecount
\hi@newcite\defselfcite{A self-citation}{%
    \global\advance\hi@anonselfcitecount\@ne
    \hi@newcite@form{fc}{#1}{%
        \hi@citeguts{%
            [Self-citation \#\the\hi@anonselfcitecount]%
        }%
    }%
}%
%<*doc>
This is useful for quickly replacing self-citations when submitting to journals
that require anonymity. No parameters are required; a unique name is
automatically generated for the citation.

\subsection{\texttt{this}}

\label{rtype:this}

Reference |this| is not a reference type but the name of a single predefined
reference, used for cross-references. See \sec{xref}.

\subsection{\texttt{verbatim}}

\label{rtype:verbatim}

Reference |verbatim| is not a reference type but the name of a single predefined
reference, the purpose of which is to enable arbitrary text in a citation item.
When using |verbatim| in a citation item, the pin cite text is not parsed but is
used verbatim as the formatted citation text:
\begin{demo}
|\sentence{see verbatim at {the \textbf{Moon}} (shining)}.|
\Produces \emph{See} the \textbf{Moon} (shining).
\end{demo}
Note how signals and parentheticals are still parsed and formatted around the
user-defined citation text.

%</doc>
%<*package>
%    \begin{macrocode}
\def\fc@verbatim{%
    \hi@citeguts{%
        \@this@page
    }%
}
\def\pc@verbatim#1{%
    \def\@this@page{#1}%
    \if@endswithdot{#1}{\addto@macro\@this@page{\@hi@dottrue}}%
}
\let\idc@verbatim\fc@verbatim
%    \end{macrocode}
%</package>
