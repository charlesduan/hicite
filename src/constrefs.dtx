%%
%% \iffalse filename: constrefs.dtx \fi
%%
%<*doc>
\input driver
\thisis{constrefs}{U.S. Constitutions and Founding Documents}

These references cover the U.S. Constitution, state constitutions, and other
founding documents such as the \emph{Federalist} papers and the Declaration of
Independence.

\input docparams

%</doc>
\hi@newcite{const}{Citation to a constitution}{const}{%
    \hi@ifset\hi@kv@name\relax{%
        \PackageWarning\hi@pkgname{%
            Constitution citation requires "name" argument
        }%
        \let\hi@kv@name\@empty
    }%
    \hi@replacethe\hi@kv@name{\hi@const@part{#1}}{%
        \hi@const@part{#1}{\@capnext\hi@kv@name}%
    }%
}
%<*doc>
Because of the intensive way in which parts of constitutions are analyzed, this
package provides that each \emph{part} of a constitution to be cited should be
defined as a separate reference.

\keyparameters{
    {name}{The textual name of the constitution part being cited. This can be a
    textual representation of the pin cite (e.g., ``Article II'') or a common
    name (e.g., ``the Equal Protection Clause'').}
    {page}{The pin cite specification for the constitution part for this
    reference. Typically this will be a listing of named divisions:
    \texttt{article 1, S 8, clause 8}, for example.}
}
\optparameters{
    {place, state}{The state from which this constitution originates, if not
    ``U.S\@.'' The place will be abbreviated using the \abbscheme{name} scheme.}
    {year}{The year of adoption, for superseded constitutions. To enter years
    for adoption of specific amendments or provisions, use \param{paren}.}
}

\pagejoining

\begin{demo}
\Example
\begin{verbatim}
\defconst{a2s3}{
    name=the Take Care Clause,
    page={Article II, S 3}
}
\end{verbatim}
\end{demo}

%</doc>
%
\hi@newcite{constamend}{A Constitutional amendment}{const}{%
    \edef\hi@kv@page{amendment \expandafter\@Roman\hi@kv@number}%
    \hi@const@part{#1}{%
        \hi@inline@the
        \@expandarg\hi@numtotxt{\hi@kv@number}
        Amendment%
    }%
}
%<*doc>
This is just a shortcut for \rtype{const} when defining whole amendments. The
parameters and usage are identical to \rtype{const} except the \param{name} and
\param{page} parameters are automatically set to ``the \meta{ordinal-number}
Amendment\\ and ``amendment \meta{number}'' respectively.

\keyparameters{
    {number}{The number of the amendment. This should be a regular Arabic
    number; it will be converted to Roman numerals for the citation form and
    ordinal number text for the inline form.}
}

\begin{demo}
\Example
\begin{verbatim}
\defconstamend{1a}{
    number=1,
}
\end{verbatim}
\end{demo}

%</doc>
%
%<*package>
%
% Generic macro for setting up constitution part citations. \#1 is the name, \#2
% the inline text. |\hi@kv@page| must be set as well.
%    \begin{macrocode}
\def\hi@const@part#1#2{%
    \hi@ifset\hi@kv@place{%
        \@expandarg\hi@abbrev@name{\def\hi@kv@place}%
    }{\def\hi@kv@place{U.S.}}%
    \hi@statcode@setup@idpc{#1}% Set up id. and page numbering
    \hi@newcite@form{fc}{#1}{%
        \hi@citeguts{%
            \hi@inline@never{%
                \noexpand\hi@fn@const{%
                    \hi@kv@place\space % State, and space
                    Const.%
                }\noexpand\@hi@dottrue
                \hi@ifset\hi@kv@year{%
                    \noexpand\@hi@dotfalse\space
                    of \hi@kv@year
                }{}%
                \hi@pageordefault{%
                    \noexpand\@hi@dotfalse
                    \hi@ifset\hi@kv@year{,}{}\space
                }{%
                    \hi@ifset\hi@kv@page{%
                        \hi@ifset\hi@kv@year{,}{}\space
                        \noexpand\@hi@dotfalse
                        \@format@page@macro\hi@kv@page
                    }{}%
                }%
            }%
            \hi@inline@only{#2}%
        }%
        \hi@invis@inline
        \the\hi@param@parens
    }%
    \hi@newcite@form{lc}{#1}{%
        \noexpand\hi@toa@duptitle{}{%
            \hi@fn@const{\hi@kv@place\space Const.}%
            \hi@ifset\hi@kv@year{\space of \hi@kv@year}{}%
        }{%
            \hi@pageordefault{%
                \hi@ifset\hi@kv@year{, }{ }%
            }{%
                \hi@ifset\hi@kv@page{%
                    \hi@ifset\hi@kv@year{, }{ }%
                    \@format@page@macro\hi@kv@page
                }{}%
            }%
        }{%
            \hi@pageordefault{%
                \hi@ifset\hi@kv@year{, }{ }%
            }{%
                \hi@ifset\hi@kv@page{%
                    \hi@ifset\hi@kv@year{, }{ }%
                    \@format@page@macro\hi@kv@page
                }{}%
            }%
        }%
        \the\hi@param@parens
    }%
}
%    \end{macrocode}
%</package>
\hi@newcite{founding}{A national founding document}{found}{%
    \hi@statcode@setup@idpc{#1}% Set up id. and page numbering
    \hi@include@page@in@toa{#1}
    \hi@newcite@form{fc}{#1}{%
        \hi@citeguts{%
            \noexpand\hi@fn@const{\hi@kv@name}%
            \hi@pageordefault{ }{%
                \hi@ifset\hi@kv@page{ \@format@page@macro\hi@kv@page}{}%
            }%
        }%
        \the\hi@param@parens
        \hi@ifset\hi@kv@year{%
            \hi@parens@add{\hi@paren@date}{\hi@kv@year}%
        }{}%
    }%
    \hi@newcite@form{lc}{#1}{%
        \noexpand\hi@toa@duptitle{}{\hi@fn@const{\hi@kv@name}}{%
            \hi@pageordefault{ }{%
                \hi@ifset\hi@kv@page{ \@format@page@macro\hi@kv@page}{}%
            }%
            \the\hi@param@parens
            \hi@ifset\hi@kv@year{%
                \hi@parens@add{\hi@paren@date}{\hi@kv@year}%
            }{}%
        }{%
            \hi@pageordefault{ }{%
                \hi@ifset\hi@kv@page{ \@format@page@macro\hi@kv@page}{}%
            }%
        }%
        \the\hi@param@parens
        \hi@ifset\hi@kv@year{%
            \hi@parens@add{\hi@paren@date}{\hi@kv@year}%
        }{}%
    }%
}
%<*doc>
This generalizes the form for citing the Declaration of Independence, for
which the reference type is otherwise unclear. \unskip\footnote{The
\emph{Bluebook} does not define a type for the Declaration of
Independence but gives it as an example of how to cite subdivisions.
Nevertheless, the chosen format is puzzling. It puts ``The'' in front of the
name, even though ``The'' is typically omitted from titles of statutes. (``The''
is not omitted from book titles that actually start with ``The,'' but the
Declaration does not have an official title on its face.)

It then puts ``U.S.'' in the date parenthetical. Putting aside the question of
whether the United States existed at the time the Declaration was signed, it is
unclear why the country is included in the parenthetical rather than as part of
the title. Typically, that placement of a jurisdiction would identify a court or
similar tribunal issuing a decision, but it is a stretch to say that the
Declaration is an adjudicated result. If it is to identify the country issuing
the Declaration to distinguish from other nations' declarations of independence,
then ``U.S.'' ought to be added in a separate parenthetical based on the usual
rule for foreign materials. The best interpretation perhaps is that the
Declaration is being cited as a book, explaining the superfluous ``The,'' and
that the United States is being identified as the publisher of a pre-1900
work---a strange approach for many reasons.}
Mainly, this form provides a flexible citation container that sets the given
title in the same font as that for constitutions.

\keyparameters{
    {name}{The name of the document.}
}
\optparameters{
    {year}{The relevant year of the document.}
}

\pagejoining

%</doc>
%
\hi@newcite{federalist}{A Federalist paper}{other}{%
    % TODO: add editor info
    \hi@newcite@form{fc}{#1}{%
        \hi@citeguts{%
            \hi@fn@booktitle{\hi@book@the Federalist No.~\hi@kv@number}%
            \space(\hi@kv@author)%
        }%
    }
    \protected@edef\reserved@a{%
        {\noexpand\hi@fn@booktitle{The Federalist No.~\hi@kv@number}}%
    }
    \@expand{\hi@supra@form{#1}}\reserved@a i%
}
%<*doc>

\keyparameters{
    {number}{The number of the paper.}
}
\optparameters{
    {author}{The author of the paper.}
}
%</doc>
%
