%%
%% \iffalse filename: draw.dtx \fi
%%
%<*doc>
\input driver
\thisis{draw}{General Formatting}

After parsing a citation string, the package must produce formatted text for
each of the citation items in the string. This section describes the overall
process of joining together citations, choosing among several possible citation
forms for a given reference, and setting up any other requisite information for
proper citation display.

An important element of citation formatting is keeping track of ``state,''
namely information about references cited so far that is needed for proper
formatting of later citations in the document. As a simple example, it is
necessary to keep track of the previously cited reference so that if the next
reference is the same, text such as \emph{id.}\ can be written. This feature of
legal citation is discussed in \sec{state}.

%</doc>
%<*package>
%
% To implement information capture from a citation data structure, recall that
% the data structure itself uses macros |\hi@pse@cite|, |\hi@pse@page|, and so
% on that delimit the components of citation items. We simply need to define
% macros for each of those delimiters, thereby setting up all the necessary
% information.
%
% \DescribeMacro\hi@draw@citation
% This macro is the general entry point for user citation commands: It
% sets up the overall citation environment, invokes the citation string parser
% to obtain a data structure of parsed citation information, and then iterates
% over that data structure to produce formatted text. \#1 and \#2 are hooks that
% user citation commands can use to set up the citation environment, and \#3 is
% the data structure received from |\hi@pse@parse|.
%
%    \begin{macrocode}
\def\hi@draw@citation#1#2#3{%
    %
    % Set up the environment
    \ifnum\hi@citelevel>\z@ \begingroup\fi
    \advance\hi@citelevel\@ne
    \hi@state@startstring
    \@hi@dotfalse
    \hi@signals@init
    %
    % The TOA sometimes needs to be in vertical mode.
    \if@hi@in@toa\else \leavevmode \fi
    %
    % Allow pre-citation hooks from the caller.
    #1%
    %
    % Execute the citation.
    #3%
    %
    % Execute post-citation hooks.
    #2%
    %
    % Clean up
    \hi@state@endstring
    \advance\hi@citelevel\m@ne
    \ifnum\hi@citelevel>\z@
        \if@hi@dot \aftergroup\@hi@dottrue \fi
        \endgroup
    \fi
}
%    \end{macrocode}
%
%
% Counter for citation nesting level. This allows for citations inside citations
% (e.g., in parentheticals).
%    \begin{macrocode}
\newcount\hi@citelevel
%    \end{macrocode}
%
%
%</package>
%<*doc>


\subsection{Drawing a Citation Item}

The following tasks must be performed when drawing each citation item within a
citation string. While this list of tasks is largely of importance only to
software implementers, it may be of interest to some cite-checkers to see the
full list of considerations that go into formatting a single citation item.

\begin{enumerate}
\item Citation state must be updated; see \sec{state}.
\item Each citation item maintains a list of parentheticals, which come from
either the reference definition or the user's citation string content.
\item The citation item's components must be collected. Some references are able
to modify the components afterwards.
\item The punctuation that precedes the citation item must be determined,
according to the previous and current signals.
\item The short, long, or \emph{id.}\ form of the citation must be chosen. That
choice must be accounted for in computations for future citations.
\item If the citation is in a footnote, then how the citation affects this and
other footnotes must be considered. See the discussion in \sec{state}.
\item The citation must be recorded for indexing in a Table of Authorities.
\item If the citation text ends with a period, then any period following the
item must be suppressed. This is obvious to human editors but surprisingly
difficult for computers.
\item For a sentence-style citation, the first letter of each sentence must be
capitalized. That could be the signal or a part of the citation text itself.
\end{enumerate}

%</doc>
%<*package>
%
%
% This macro is run for every citation item inside a citation string. It
% performs the following tasks.
%
%    \begin{macrocode}
\def\hi@pse@cite#1{%
%    \end{macrocode}
%
%   First, it initializes the citation state and parentheticals list for this
%   citation.
%
%
%    \begin{macrocode}
    \hi@state@startelt % Resets all the \@this@[x] macros to \relax
    \hi@parens@reset
%    \end{macrocode}
%
% Next, it runs all the |\hi@pse@|\meta{component} macros to collect necessary
% citation information. Furthermore, if this citation has |\mc@|\meta{cite}
% defined, then execute it at this point to allow the desired modifications to
% the state variables.
%
%    \begin{macrocode}
    #1%
    \@ifundefined{mc@\@this@case}{}{\@nameuse{mc@\@this@case}}%
%    \end{macrocode}
%
% Next it chooses the citation form to use according to |\hi@draw@choosecite|,
% saving the desired form to a temporary macro.
%
%    \begin{macrocode}
    \hi@draw@choosecite{\def\hi@draw@this@type}%
%    \end{macrocode}
%
% The citation's effect on footnotes is next considered. See the discussion on
% citation counts in footnotes. Also the citation must be added to the Table of
% Authorities.
%
%    \begin{macrocode}
    \@testcase
        \ifx\hi@draw@this@type\hi@draw@idname\fi{\hi@note@advance@id}%
        \if@hi@name\fi{\hi@note@advance}%
        \default{\hi@note@advance@id}%
    \hi@draw@addtotoa
%    \end{macrocode}
%
% Finally, the preceding punctuation and the citation itself can be produced.
%
%    \begin{macrocode}
    \hi@draw@this@punct
    \@test\ifx\hi@draw@this@punct\@empty\fi{}{\hi@deferred@note@enditem\space}%
    \@hi@dotfalse
    \hi@draw@this@fmtsig
    \hi@draw@cite
%    \end{macrocode}
%
% After the citation text is produced, we record the use of the reference.
% The recordation seems to need to come after producing the text because some
% citation macros (inline statutes) call |\hi@record@choose| to figure out which
% form to use. We also update the state variables as appropriate.
%
%    \begin{macrocode}
    \hi@record@cite\@this@case
    \hi@state@endelt
%    \end{macrocode}
%
% Next, the parentheticals are displayed. They must be displayed \emph{after}
% state variables are updated, because nested citations within the
% parentheticals ought to take into account the main citation. For example, a
% citation of \texttt{case at 100 (explaining: case at 150)} should be typeset
% with an \emph{id.}\ form inside the parenthetical.
%
%    \begin{macrocode}
    \hi@parens@show
    \hi@nocap
    \@hi@nametrue
}
%    \end{macrocode}
%
% Adds the citation to the Table of Authorities.
%    \begin{macrocode}
\def\hi@draw@addtotoa{%
    \ifx\@this@page\relax
        \hi@addtotoa{\@this@case}{}%
    \else
        \hi@addtotoa{\@this@case}{\@this@orig@page}%
    \fi
}
%    \end{macrocode}
%
% Executes the appropriate drawing citation macro. The type has been set already
% in the macro |\hi@draw@this@type|.
%
%    \begin{macrocode}
\def\hi@draw@cite{%
    \@test\ifx\hi@draw@this@type\hi@draw@idname\fi{%
        \@ifundefined{\hi@draw@this@type @\@this@case}{%
            \hi@draw@id
        }{%
            \@nameuse{\hi@draw@this@type @\@this@case}%
        }%
    }{%
        \@ifundefined{\hi@draw@this@type @\@this@case}{%
            \hi@draw@cite@undefined
        }{%
            \@nameuse{\hi@draw@this@type @\@this@case}%
        }%
    }%
}
%    \end{macrocode}
%
%    If there is no macro for the current reference and citation type, then this
%    macro is run by default.
%
%    \begin{macrocode}
\def\hi@draw@cite@undefined{%
    \expandafter\hi@missing@cite\expandafter{\@this@case}%
    \textbf{\hi@usevol{}\@this@case\hi@maybepage{ at }}\hi@nocap
}
%    \end{macrocode}
%
% Draws \emph{id.}\ citations, which consist of a volume number, keyword
% (defined as |\hi@id|), and page number.
%
%    \begin{macrocode}
\def\hi@id{\hi@fn@latin{id.}\protect\@hi@dottrue}%
\def\hi@draw@idname{idc}
\def\hi@draw@id{%
    \hi@citeguts{%
        \@testcase
            \ifx\@this@vol\relax\fi{}%
            \ifx\@this@vol\@last@vol\fi{}%
            \default{\@this@vol\hi@nocap\space}%
        \hi@id
        \@test\ifx\@this@page\@last@page\fi{}{%
            \hi@pageordefault{%
                ~\hi@page@atorsect
            }{}%
        }%
    }%
}
%    \end{macrocode}
%
%
%
% \paragraph{Reading Citation Item Components}
% These are macros corresponding to each of the components from the citation
% state machine.
%
% Reads the optional argument, setting the appropriate state variable.
%
%    \begin{macrocode}
\def\hi@pse@opt#1{\def\@this@opt{#1}}
%    \end{macrocode}
%
% Reads the signal, volume number, and reference name, setting state variables
% as appropriate.
%
%    \begin{macrocode}
\def\hi@pse@svr#1{\hi@pse@sigvolref{#1}\hi@draw@sigvolref}
\def\hi@draw@sigvolref#1#2#3{%
    %
    % Set the next signal, and receive the punctuation and formatted signal to
    % draw in the two given macros.
    %
    \hi@signals@set{#1}{\@twodef\hi@draw@this@punct\hi@draw@this@fmtsig}%
    %
    % Set the remaining state variables.
    \ifstrempty{#2}{}{\def\@this@vol{#2}}%
    \def\@this@case{#3}%
}
%    \end{macrocode}
%
%
% Reads and formats the page number per |\@this@case| (which should already be
% set).
%
%    \begin{macrocode}
\def\hi@pse@page#1{%
    \def\@this@orig@page{#1}%
    \@ifundefined{pc@\@this@case}{%
        \protected@edef\@this@page{\@format@pageno{#1}}%
    }{%
        \@nameuse{pc@\@this@case}{#1}%
    }%
}
%    \end{macrocode}
%
%
% Processes a parenthetical in a citation item. See the \sec{parens}.
%
%    \begin{macrocode}
\def\hi@pse@paren#1{%
    \hi@paren@parse{#1}%
}
%    \end{macrocode}
%
%
%</package>
%<*doc>

\subsection{Citation Forms}

\label{s:draw-form}

When formatting a citation item, each reference can generate a variety of
citation texts. The types are given and explained here, along with their
internal macro prefixes:
\begin{description}
\item[Full] (|fc|) The full citation form.
\item[Short] (|sc|) The non-\emph{id.}\ short form. Defaults to the full form.
\item[List] (|lc|) The form used in the Table of Authorities. Defaults to the
full form.
\item[Pin cite] (|pc|) Some references have special ways of treating pin cite
numbers. This citation form is used in the pin cite formatting routine (see
\sec{pages}).
\item[\emph{Id.}] (|idc|) The form to be used for immediately successive
citations, if not the standard \emph{id.}~form.
\item[Inline full] (|ifc|) The inline citation form to be used upon the first
inline usage.
\item[Inline short] (|isc|) The inline citation form to be used on subsequent
inline usages.
\end{description}

For each reference \meta{ref-name}, a macro of the form
|\|\meta{prefix}|@|\meta{ref-name} may be defined containing the relevant text
for the citation form. For example, |\fc@brown| would be the definition of the
full citation for a reference named |brown|. It may be helpful for debugging to
see these macros. To do so, include the following in your document:
\begin{demo}
|\expandafter\show\csname |\meta{prefix}|@|\meta{ref-name}|\endcsname|
\end{demo}

\subsection{Choosing the Citation Form}

Upon each citation to the reference, it is necessary to
choose among these forms. The following rules inform that choice.

\begin{itemize}
\item An \emph{id.}\ citation is appropriate only when the reference being cited
matches the last reference, recorded as described in \sec{state}.
In any event, it is never appropriate in an inline citation.

\item The table of authorities form is chosen for non-nested citations inside a
table of authorities. If no table of authorities form is present for a given
reference, the full citation form is used instead. For nested citations inside a
table of authorities, the full citation is always used.

\item In other cases, a complex algorithm must be applied to choose between the
short and full citation forms. \unskip\footnote{As described in \sec{short},
the algorithm is complex because it depends on where previous citations were
used: A full citation in a footnote, for example, does not justify a short form
in main body text.} This algorithm is described in \sec{short}.
In any event, if no short citation form is defined for a reference, the full
citation form is used.

\item In an inline context, the same algorithm described in \sec{short} is used,
except the inline full/short citation forms are preferred if they are defined.
\end{itemize}

In addition to choosing a citation form, the package sometimes allows a
``nameless'' variant of a citation form. For example, a citation to a case may
omit the name of the case if the name was previously used in the text. In
general, a nameless citation to a reference is permitted if the previous inline
citation was to the same reference. There are some nuances to what constitutes
the ``previous'' citation, as described in \sec{state}.


%</doc>
%<*package>
% 
% Chooses among the citation forms. \#1 is a callback that is given the prefix
% of the chosen citation form. Also sets |\hi@noname| as appropriate.
%
%    \begin{macrocode}
\def\hi@draw@choosecite#1{%
    \@testcase
        \if@hi@inline\fi{\hi@draw@choosecite@inline{#1}}%
        \ifx\@last@case\@this@case\fi{#1{idc}}%
        \default{%
            \ifx\@last@inline\@this@case \hi@noname \fi
            \hi@draw@choosecite@{#1}%
        }%
}
%    \end{macrocode}
%
% Chooses the citation form when it is certain not to be \emph{id.} This macro
% distinguishes the table of authorities, and otherwise defers to
% |\hi@record@choose|.
%
%    \begin{macrocode}
\def\hi@draw@choosecite@#1{%
    \@test\if@hi@in@toa\fi{%
        \@test \ifnum\hi@citelevel=\@ne\fi{%
            \hi@draw@which{lc,fc}{#1}%
        }{#1{fc}}%
    }{%
        \hi@record@choose{\@this@case}{%
            #1{fc}%
        }{%
            \hi@draw@which{sc,fc}{#1}%
        }%
    }%
}%
%    \end{macrocode}
%
%
% Chooses the citation form for inline citations.
%
%    \begin{macrocode}
\def\hi@draw@choosecite@inline#1{%
    \hi@record@choose{\@this@case}{%
        \hi@draw@which{ifc,fc}{#1}%
    }{%
        \hi@draw@which{isc,ifc,sc,fc}{#1}%
    }%
}%
%    \end{macrocode}
%
% Selects the first prefix for which the current reference has a defined macro.
% \#1 is a comma-separated list of prefixes, and \#2 is a callback that is given
% an argument of the selected reference.
%
%    \begin{macrocode}
\def\hi@draw@which#1#2{%
    \find@in{,}{#1}{\hi@draw@which@{#2}}{#2{#1}}%
}
% |#1| is the callback, |#2| the first prefix, |#3| the rest of the prefixes.
\def\hi@draw@which@#1#2#3{%
    \@ifundefined{#2@\@this@case}{\hi@draw@which{#3}{#1}}{#1{#2}}%
}
%    \end{macrocode}
%
%
% In previous versions, the nameless variant of citations was chosen by the
% pseudo-reference name |*|. The package now determines whether to use nameless
% variants automatically, but still accepts |*| as a synonym for the previous
% inline reference.
%
%    \begin{macrocode}
\@namedef{mc@*}{%
    \ifx\@last@inline\relax
        \PackageError\hi@pkgname{Star citation must follow an inline
        citation}{Probably change the star citation out}%
        \def\@this@case{???}%
    \else
        % \hi@noname -- See \hi@draw@choosecite
        \let\@this@case\@last@inline
    \fi
}
%    \end{macrocode}
%
%</package>
%<*doc>

\subsection{Missing References}

Sometimes it is desirable to cite to a reference that has not yet been defined.
This allows authors familiar with a reference to cite it without having to stop
immediately to enter the reference definition and parameters. The package
formats the undefined references in bold and also compiles a list of undefined
references, which is displayed at the end of document compilation.

As a convenience, the package also keeps track of defined references that are
missing page numbers. The user enters the unknown page number as a question
mark, and the package displays a list of all unentered page numbers at the end
of compilation. This tracking is turned on by default; if the
\DescribeMacro{nomissing} |nomissing| option is given to the package then an
error will be raised for missing reference definitions.

%</doc>
%<*package>
%
%
%
%    \begin{macrocode}
\def\hi@missing@cites{}
\def\hi@missing@pages{}
%
\def\hi@missing@cite#1{%
    \expandafter\hi@missing@cite@\expandafter{\the\inputlineno}{#1}%
}
\def\hi@missing@cite@#1#2{%
    \@ifundefined{hi@miss@cite@#2}{%
        \g@addto@macro\hi@missing@cites{\hi@missing@display{#2}{l. #1}}%
        \expandafter\let\csname hi@miss@cite@#2\endcsname\@empty
    }{}%
}
\def\hi@missing@page#1{%
    \expandafter\hi@missing@page@\expandafter{\the\inputlineno}{#1}%
}
\def\hi@missing@page@#1#2{%
    \g@addto@macro\hi@missing@pages{\hi@missing@display{#2}{l. #1}}%
}
\def\hi@missing@display#1#2{%
    \space\space\space\space\pad@string{#1}{iiiiiiiiiiiiiiiiiiiiiiii} #2%
    \MessageBreak
}
\AtEndDocument{%
    \ifx\hi@missing@cites\@empty\else
        \PackageWarning\hi@pkgname{%
            The following citations were undefined:\MessageBreak
            \hi@missing@cites
        }%
    \fi
    \ifx\hi@missing@pages\@empty\else
        \PackageWarning\hi@pkgname{%
            The following citations were missing page numbers:\MessageBreak
            \hi@missing@pages
        }%
    \fi
}
%    \end{macrocode}
%
%
%</package>
%
%<*test>

\section{draw.dtx}

\subsection{Which}

\begingroup

    \def\@this@case{whichtest}
    \def\fc@whichtest{a}

    \AssertCallback{\hi@draw@which{sc,fc}}{fc}

    \def\sc@whichtest{a}

    \AssertCallback{\hi@draw@which{sc,fc}}{sc}
    \AssertCallback{\hi@draw@which{isc,sc,fc}}{sc}

\endgroup

%</test>
