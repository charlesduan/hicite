%%
%% \iffalse filename: anonymous.dtx \fi
%%
%<*doc>
\input driver
\thisis{anonymous}{References Within References}

Often a shorter work will be included in a larger compilation, and both need to
be cited in order to facilitate access to the shorter work. A private letter may
be reprinted in a volume of collected letters, for example, or a brief filed in
a case may appear within the docket for a case. \hia* provides a number of ways
to support this.

First, some reference parameters accept another reference nickname as input. All
references accept the parameter \param{reprinted}, and some such as
\rtype{casedoc} accept the parameter \param{citation}. More generally, the
reference type \rtype{citecontainer} provides a comprehensive mechanism for
citing a reference contained within another reference.

As an example, consider the following reference definitions:
\begin{quote}
\begin{verbatim}
\defletter{madison-letter}{
    author=James Madison,
    to=W.T. Barry,
    date=aug 4 1822,
    reprinted=madison-writings,
}
\defbook{madison-writings}{
    name=Letters and Other Writings of James Madison,
    year=1884,
}
\end{verbatim}
\end{quote}
The parameter \param{reprinted} in the letter's reference definition identifies
the book in which the letter has been reprinted by its reference nickname, and
the book is defined separately as a reference. This approach works perfectly
well, but it can be cumbersome to define the book as its own reference with its
own nickname. If the book is never used
anywhere else, it might be convenient to define the book reference inside the
letter reference definition.

``Anonymous references'' allow for this.
Instead of providing a reference name as the value to
\param{reprinted}, the book can be included in the letter reference definition
as follows:
\begin{quote}
\begin{verbatim}
\defletter{madison-letter}{
    author=James Madison,
    to=W.T. Barry,
    date=aug 4 1822,
    reprinted=book: {
        name=Letters and Other Writings of James Madison,
        year=1884,
    },
}
\end{verbatim}
\end{quote}

More generally, anonymous reference definitions consist of the reference type
(without the |\def| prefix), a colon and space, and a set of parameter-value
pairs in braces. Note that whitespace following the parameter-value pairs can
confuse the parser, so it is best to place a comma after the closing brace as
shown above.

%</doc>
%<*package>
%
% A counter used to make unique names for anonymous references.
%
%    \begin{macrocode}
\newcount\hi@anonrefno
\make@find@in{: }
%    \end{macrocode}
%
% Used inside a reference parameter definer, determines if the given parameter
% value is an anonymous reference. If so, creates a new reference object, gives
% it a name, and runs a callback on the created name. Otherwise, the value is a
% reference name, so the callback is run directly on it. \#1 is the reference
% name or an anonymous reference definition; \#2 is the callback.
%
%    \begin{macrocode}
\def\hi@anonref#1#2{%
    \find@in{: }{#1}{\hi@anonref@{#2}}{#2{#1}}
}
%    \end{macrocode}
%
% \#1 is the callback, \#2 is the type, \#3 the parameter definition.
%    \begin{macrocode}
\def\hi@anonref@#1#2#3{%
    \global\advance\hi@anonrefno\@ne
    \edef\reserved@a{anon@cite@\the\hi@anonrefno}%
    \begingroup
        \hi@param@clear
        \chop@space@then@run{#3}{%
            \@expand{\csname def#2\endcsname}{\reserved@a}i%
        }%
    \endgroup
    \@expand{#1}\reserved@a i%
}
%    \end{macrocode}
%
% Defining finders that cause syntax highlighting problems.
%    \begin{macrocode}
\make@find@in{ (}%
\make@find@end{)}%
%    \end{macrocode}
%
%</package>
