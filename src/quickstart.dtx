%%
%% \iffalse filename: intro.dtx \fi
%%
%<*doc>

You've done the research, your stack of cases and articles is overflowing, the
ideas are swimming in your head, and are ready to start writing. But as you open
your document, a troubling question creeps in.

\emph{Will you format citations as you
write?} That will interrupt your train of thought every few minutes as you pull
out the style guide. And as you edit and rearrange, your citations
will need to be redone many times over. All those \emph{supra} and \emph{id.\@}
citations will quickly become out of place and perhaps unintelligible as you
polish the text.

\emph{Or will you put off citation formatting
for later?} In that case your coauthors and draft reviewers will see a mess of
an unfinished paper. And you'll have to reserve hours or days before submitting
for publication, to
slog through an unrewarding round of citation formatting.

\hia* offers a better way. With it, you can type citations in your writing
concisely and efficiently, but with all the features of
a rich legal citation system. You can edit and reorganize with almost no
need to manually update existing citations in your document. And you can
produce correctly formatted citations at any stage of the writing process, so
that
when you and others are reviewing drafts, you aren't distracted by incomplete
formatting.

I have been developing \hia* since I started law school in 2004. In the years
since, I have used it to format tens of thousands of citations in hundreds of
legal briefs, administrative comments, and journal articles. It produces tables
of authorities without hesitation, handles lengthy law review pieces without
difficulty, and even is compatible with widely used citation management
software. Most
importantly, when the computer handles the appearance of my writing, I get to
focus on the substance, which has made my writing process far more rewarding and
enjoyable. To me, \hia* is an indispensable tool for my legal writing, and
I hope it will also become your choice hereinafter.




\section{Quick Start Guide}


This introductory guide will show the basic concepts you'll need to use \hia* to
add citations to your document. As with other automated citation systems, you
will need to do two things:
\begin{enumerate}
\item Define your references (cases, books, articles,
and so on) by entering bibliographic information for them.
\item Cite references by entering citation commands into your document.
\end{enumerate}

Importantly, these two steps do not need to be done in order. In my personal
workflow, I will often first write, entering citation commands for references
that I've read but have not defined. I find that this order preserves my train
of thought best. \hia* displays a list of undefined references and other missing
items at the end of document compilation, conveniently letting you know what
needs to be fixed.


\subsection{Prerequisites}

At this stage, I assume that you have general knowledge both of legal citation
and of the \LaTeX\ typesetting system. \hia* operates purely within \LaTeX, and
is controlled by commands described below in this manual.

To begin, you should include the package in the preamble of your document:
\begin{demo}
|\usepackage|\oarg{options}|{hicite}|
\end{demo}
The \meta{options} are described in \sec{options}. The most important of these
are the options \opt{review} and \opt{memo}, which determine whether the package
will follow the rules for law review articles or for legal memoranda,
respectively.




\subsection{Defining References}

A \emph{reference} is a single work of authority that you want to cite. In order
to cite a reference in \hia*, you must ``define'' it by entering key information
about the reference.

To define a reference, you will need to (1) choose a nickname
for the reference, (2) determine the type of reference, and (3)
identify the relevant bibliographic information for the type.
Part~\ref{p:alltypes} lists the available reference types; common ones are
\rtype{book}, \rtype{case}, \rtype{website}, and \rtype{jrnart} (journal
articles). The bibliographic information is entered as parameter-value pairs.
For example:
\begin{demo}
|\defcase{wheaton}{| \\
|    p=Wheaton,| \\
|    d=Peters,| \\
|    vol=33,| \\
|    rep=U.S.,| \\
|    page=591,| \\
|    year=1834,| \\
|}|
\end{demo}
defines a reference named |wheaton| for the famous lawsuit between two Supreme
Court reporters.

\hia* permits for various shortcuts to simplify the entry of references. The
following are equivalent to the above definition:
\begin{demo}
|\defcase{wheaton}{| \\
|    parties=Wheaton v. Peters,| \\
|    cite=33 U.S. 591,| \\
|    year=1834,| \\
|}| \\
|\defcase{wheaton}{| \\
|    Wheaton v. Peters, 33 U.S. 591 (1834)| \\
|}|
\end{demo}

There is also rich support for author names and dates. The following definition
shows how to enter a report with an individual author and an institutional
author:
\begin{demo}
|\defbook{ftc-pharma}{| \\
|    author=Markus H. Meier et al.,| \\
|    instauth=Federal Trade Commission,| \\
|    title=Overview of FTC Actions [...],| \\
|    date=june 2019,| \\
|}|
\end{demo}

For a more detailed overview of how to define references, see \sec{refs-syntax}
and the sections that follow. Shortcuts for reference entry are described in
\sec{cparse}, and authors and dates in \sec{names} and \sec{dates} respectively.





\subsection{Entering Citations}

To cite references in a document, the key commands are |\sentence| and
|\clause|, which produce sentence-style and clause-style citations. For law
review articles, these commands should be used in footnotes, and the |\note|
command creates a footnote with a sentence-style citation.

All of these commands receive a single argument of a citation string following a
syntax described in \sec{parse}. By design, the syntax looks like a simplified
legal citation. The following would generate a two-element string-cite to
\emph{Wheaton v.~Peters}, describing a (hypothetical) tension between two parts
of the decision:
\begin{demo}
|\sentence{|see, e.g., wheaton at 598 note 3 (Story, J.) (describing holding);
but see wheaton at 600-602 (noting alternative views)|}|.
\Produces
\emph{See, e.g.}, \emph{Wheaton v.~Peters}, 33 U.S. 591, 598 n.3
(1834) (Story, J.) (describing holding); \emph{but see} \emph{id.}~at 600--02
(noting alternative views).
\end{demo}

This example shows a number of notable features of \hia*. First, references are
always cited by identifier; any short or \emph{id.}~form is selected
automatically without user intervention. This means that citations are portable:
They largely can be moved around the document unchanged, simplifying
reorganizations of one's writing.

Second, there is rich support for signals, pin cites, and parentheticals in
citations. In particular, pin cites can involve page numbers, section numbers,
footnotes, and more, as described in \sec{pages}. They are abbreviated
automatically, as shown above. Parentheticals and signals are
also ordered, arranged, and formatted appropriately.

Third and perhaps most importantly, nearly all the formatting work is hidden
from the writer. Abbreviations, fonts, and keywords are automatically chosen to
conform to a given citation system. Importantly, this makes it easy to conform
an already-written document to another citation system. If one journal prefers
small caps but another does not, it is simple to produce different documents for
each.

\paragraph{Inline Citations}
In addition to sentence and clause citations, \hia* supports citing references
as parts of the text: naming a case as the subject of a sentence, for example.
The relevant commands are |\inline|, |\Inline|, |\adjective|, and |\Adjective|
depending on whether the reference is being cited at the beginning of a sentence
and the part of speech the reference is taking on:
\begin{demo}
|In \inline{wheaton}, the Court....| \\
|\sentence{wheaton at 603}.| \\
\Produces In \emph{Wheaton v.~Peters}, the Court\ldots. \emph{See} 33 U.S. 591,
603 (1834).
\end{demo}
Note how the |\sentence| citation automatically omits the case name because it
was displayed in with the previous |\inline| command.

For more on the parts and syntax of sentence, clause, and inline citations, see
Part~\ref{p:citations}.



\subsection{Additional Features}

Most of \hia* involves transforming reference definitions and citation commands
into formatted text. However, the package provides a few additional features of
interest: It can produce tables of authorities (see \sec{toa}), switch from
footnotes to endnotes (see \sec{endnotes}), and manage abbreviations in text
(see reference type \rtype{abbrev}).

Perhaps the most important additional feature is cross-references, described in
\sec{xref}. To mark a part of a document for cross-referencing:
\begin{demo}
|\hilabel{keypart}| \\
|This is the key part!| \\
|\hiendlabel{keypart}|
\end{demo}
The |\hilabel| command automatically chooses the best way to refer to the marked
section depending on where it is used (in a footnote, in a figure, or in plain
text). The |\hisectlabel| command can label sections by number.

To insert a cross-reference, simply use a citation to the predefined reference
\rtype{this} with the cross-reference label as the pincite:
\begin{demo}
|\sentence{see this at keypart}.|
\end{demo}
Again, the value of this cross-referencing system is that the cited numbers
automatically update as the document is edited, minimizing the amount of manual
updating.



\section{Structure of This Manual}

Besides simply describing how \hia* works, this manual is intended to explain my
conceptual framework for legal citation that is embedded in the system.
With that framework in mind, this manual on a
more practical level then teaches writers how to write citation input commands
that will produce formatted citations meeting their needs, and explains how
those inputs are formatted. As a result, it has three main objectives:
\begin{itemize}
\item Describing the data model behind \hia*'s legal citations
\item Specifying the input formats that writers must use
\item To a lesser extent, explaining the algorithms for producing formatted
citations
\end{itemize}

The manual is organized topically for different elements of legal citations,
with each section covering these three objectives of the data model, input
formats, and formatting algorithms (to the extent that each objective is
relevant). There are five parts:
\begin{enumerate}
\item How references are defined, including common parameters of references such
as authors and dates.
\item How citations are entered into documents.
\item General formatting algorithms used, such as keeping track of previous
citations for short forms.
\item All reference definition parameters.
\item All reference types.
\end{enumerate}

As currently written, this manual is primarily for writers of legal
documents who already have a baseline knowledge of legal citation, and
secondarily for software developers who wish to learn how the system works so
that they can modify it to produce different citation formats. It is not as
useful for learning how to format legal citations, as most citation manuals are.
First, it explains the formatting of citations in a manner that is useful to
programmers but probably confusing to human cite-checkers. \unskip\footnote{In
particular, a computer reads and formats citations sequentially, starting
from the beginning and never looking backwards, so at any given point it must
keep track of a great deal of information in memory about prior citations.
Computers do this easily, but a person would have to write copious and
error-prone notes to check the citations purely in forward order. A more
reasonable approach for a person would be to read the citations backwards to
find necessary information about previous citations.} Second, this manual does
not discuss the meanings of citation elements: It does not explain what citation
signals mean, when parentheticals are appropriate, or which edition of a book to
cite, for example.

It would not be difficult to revise this manual to address
these deficiencies and turn it into a guide on manual citation formatting, by
augmenting the data model sections to describe semantics and by rewriting the
formatting algorithm sections. Were that work to be done, the hope is that
presenting students with a data model and principles in tandem with formats
would offer them a clearer, enriched view of the seemingly arcane world of legal
citation.

\paragraph{Grammatical structures}
In many places, this manual describes expected inputs in terms of grammatical
structures. These are shown in blue boxes and use a version of Backus-Naur form.
\unskip\footnote{Spaces between items are not specified rigorously in these
grammars.}
For example:
\begin{grammar}
\meta{address} := \meta{number} \meta{street} [ \meta{apartment} ] \|
\littext{PO Box} \meta{number}
\end{grammar}
For those unfamiliar with this notation:
\begin{itemize}
\item The items between angle brackets represent text values, akin to entry
spaces in a form.
\item := means that the item on the left is defined to consist of the specified
items on the right.
\item \| means to choose either of the items to the left or right.
\item Square brackets [ ]  mean that the material inside is optional.
\item \underline{Underlined} text needs to be included literally.
\item ( \meta{item} )* means that whatever is inside the parentheses can be
repeated (or used once, or omitted entirely).
\end{itemize}

So to explain the above statement, an \meta{address} can take on two forms. It
can be a \meta{number} followed by a \meta{street} and optionally an
\meta{apartment} number, such as ``123 Elm Lane'' or ``15 Broadway \#5. Or it
can be a post office box like ``PO Box 123''. In general, grammatical
definitions will be followed by examples to help demonstrate what input is
expected.

In other cases, the manual will show expected inputs using a \LaTeX\ syntax:
\begin{grammar}
|\command|\oarg{option}\marg{required}
\end{grammar}
This means that the macro |\command| accepts an optional argument \meta{option},
surrounded by square brackets if present, and needs a mandatory argument
\meta{required}, surrounded by curly braces.

\paragraph{Documentation files}
This manual is one of two parts of documentation for \hia*. The package source
code is contained in multiple module files compiled into a unified main package,
and most chapters of this manual correspond to module files. The module files
are written as literate programming \LaTeX\ |.dtx| files, and contain further
documentation of the internal macros and implementation. For each chapter
having a corresponding module file, the filename is listed in the margin so that
the expanded documentation can be identified.

The manual alone should be sufficient for writers of legal documents to
understand how to use \hia*. The module file documentation is intended to help
package authors who wish to correct, modify, or extend the citation program
itself, perhaps to adapt it to other citation formatting regimes.




\section{Installation and Usage}


\hia* comprises a suite of files:
\begin{itemize}
\item The package file itself, \texttt{hicite.sty}.
\item Several auxiliary package files, described in \pt{support}.
\item Tables of abbreviations and other information.
\end{itemize}
All of these files need to be installed in a \TeX-accessible location.

To use the package, simply include it in a document:
\begin{demo}
|\usepackage|\oarg{options}|{hicite}|
\end{demo}
Then compile it using \LaTeX\ or an equivalent. No auxiliary programs are
needed, even for table indexing. However, multiple compilation runs may be
necessary in order to get cross-references and short citations to be correct.

If you have an existing Biblatex file, see \sec{hibib} for possible
compatibility options.



\section{Intellectual Property Concerns}

Over the years, there have been multiple disputes over intellectual property in
legal citation systems and manuals. It is both necessary to address both what
rights I claim as the author of \hia*, and what rights of others it implicates.

The source code of \hia* is released under the GNU General Public License (GPL)
version 3.0. That license applies to any executable code. The text of this
manual itself and all other documentation is released under a Creative Commons
license as described on the first page of the manual.
For purposes of clarity, the Creative Commons license applies to any text found
in this user manual and any text found in comments in the package source code;
the GPL license applies otherwise.

With respect to other legal citation manuals and systems, this software and
manual are not endorsed by, affiliated with, or authorized by any other legal
citation system, such as the Bluebook, the University of Chicago Manual of Legal
Citation (the ``Maroonbook''), the ALWD Guide to Legal Citation, or the Indigo
Book. It does not claim to be compatible with or identical to any of those
systems; indeed a design principle of \hia* is to be more flexible than any of
these existing systems. To the extent that material from existing citation
systems is used (primarily for tables of abbreviations), it has been taken from
the Indigo Book, which has been placed in the public domain.

%</doc>
