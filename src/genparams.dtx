%%
%% \iffalse filename: genparams.dtx \fi
%%
%<*doc>
\input driver
\thisis{genparams}{Generally Applicable Parameters}

These parameters may be used in defining any reference, regardless of type.

\input docparams

%</doc>
\hi@defparam{accessdate}{URL access date}{%
    \hi@date@parse[last visited]{#1}{\def\reserved@a}%
    \expandafter\hi@param@addparen\expandafter\hi@paren@accessdate
        \expandafter{\reserved@a}%
}
%<*doc>
This is added as a parenthetical immediately after the \param{url} parameter.
(It can be used without \param{url}, perhaps to indicate the access date of a
non-linked online resource such as a case in a database service.) The date
qualifier ``last visited'' is prepended to the date unless another qualifier is
given.
%</doc>
%
\hi@defparam@noval{bookreview}{Indicator that an article is a book review}{%
    \ifstrempty{#1}{%
        \hi@param@addparen\hi@paren@bookreview{book review}%
    }{%
        \hi@param@addparen\hi@paren@bookreview{reviewing \clause{#1}}%
    }%
}
%<*doc>
This adds a parenthetical. If no value is given, then the parenthetical reads
just ``book review.'' Otherwise, the value should be a reference name for the
book being reviewed, which will be cited in the parenthetical.
%</doc>
%
\hi@defparam{toacat}{Table of Authorities category for this citation}{%
    \@expandarg\SetTOACategory\@this@case{#1}%
}
%<*doc>
The valid category types are listed in \sec{toa-categories}.
%</doc>
%
\hi@param@alias{comment}{paren}
%<*doc>
This parameter was originally used for commentary, but now the \param{paren}
parameter handles those as well.
%</doc>
%
\hi@defparam{country}{Country for citation}{%
    \@ifundefined{hi@ctry@#1}{\hi@param@addparen\hi@paren@country{#1}}{%
        \@expand{\hi@param@addparen\hi@paren@country}{%
            \csname hi@ctry@#1\endcsname
        }{ii}%
    }%
}
%<*doc>
This should generally be added to any reference with legal authority outside the
United States. The country name is automatically abbreviated.
%</doc>
%
\hi@newparam{defaultopt}{The default optional parameter}
%<*doc>
Some reference types that use the optional argument will use this value if the
optional argument is unset.
%</doc>
%
\hi@defparam{hereinafter}{Text to use for hereinafter citation form}{%
    \expandafter\hi@short@register\expandafter{\@this@case}{#1}{y}%
    \hi@supra@form\@this@case{#1}%
    \hi@param@addparen\hi@paren@hia{#1}%
}
%<*doc>
This parameter will define the short-form citation for the reference.
%</doc>
%
\hi@defparam{hyphenate}{Set hyphenation for a word}{%
    \hyphenation{#1}%
}
%<*doc>
This just calls |\hyphenation| on the word(s) given in the value. It is
convenient if unusual hypheniation is required for a word in the reference
(e.g., a named party to a case), because it keeps the hyphenation as
bibliographic data.
%</doc>
%
\hi@param@alias{hyphenation}{hyphenate}
\hi@newparam{kind}{Type of reference}
%<*doc>
This is automatically set when the |\def|\meta{kind} command is called to define
a reference, and it should not be changed.
%</doc>
%
\hi@newparam@noval{noetal}{No et al. with 3 or more authors}
%<*doc>
This parameter has no display effect, but advises the package not to issue a
warning if it detects an name list of three or more entities.
%</doc>
%
\hi@defparam@noval{notoa}{Do not include in the table of authorities}{%
    \SetTOACategory{\@this@case}{}%
}
%
\hi@newparam{opturl}{Optional URL}
%<*doc>
Generally this parameter is ignored. However, if the package option
\opt{useopturl} is set, then this parameter is an alias for \param{url}. This
is useful for references where a URL could be helpful for some publications but
not others, as it enables the writer to quickly switch between presentation
forms.
%</doc>
%
\hi@defparam{origyear}{Year of original publication of the reference}{%
    \hi@date@parse{#1}{\hi@param@addparen\hi@paren@origdate}%
}
%<*doc>
This is typically used when a modern edition of a pre-1900 work is cited, such
that the \param{year} parameter is set to the modern publication year and this
parameter set to give the original publication date of the work.
%</doc>
%
\hi@param@alias{origdate}{origyear}
\hi@defparam{parallel}{Parallel citation}{%
    \find@in{: }{#1}{\hi@defkey@parallel}{%
        \PackageError\hi@pkgname{%
            Parameter `parallel' must include a page segment\MessageBreak
            key name, followed by a colon%
        }{It should look like `key: 123 Rep. 456'}%
    }%
}
%<*doc>
This must be entered in the form \meta{key}|: |\meta{cite}. The \meta{key} is
used as a pin cite segment key, described in \sec{pages}. When the relevant
reference is cited, the parallel cite is added before the date parenthetical,
along with a page number found in the appropriate section of the given pin cite
if present. The \meta{cite} text is used as given and not parsed or abbreviated.

\textbf{Note}: Currently the parallel citation is only displayed in the full
citation. This is because the package has no mechanism for displaying
document-level parentheticals in short citations. Hopefully this will be fixed
in the future.
%</doc>
%<*package>
%
% Sets up the parallel citation. This adds a parenthetical of level
% |\hi@paren@parallel| that inserts the parallel citation text. If a page number
% is defined, then the parallel citation page number is extracted and inserted
% as well. \#1 is the key, \#2 is the parallel citation text.
%    \begin{macrocode}
\def\hi@defkey@parallel#1#2{%
    \hi@fpg@defsegment{#1}%
    \hi@param@addparen\hi@paren@parallel{%
        #2%
        \@test\ifx\@this@page\relax\fi{}{%
            \@expandarg\hi@fpg@segment{\@this@orig@page}{#1}{, }{%
                \PackageWarning\hi@pkgname{%
                    In citation to \@this@case, parallel cite\MessageBreak
                    with key `#1' not given in page number%
                }, \textbf{??}%
            }%
        }%
    }%
}
%    \end{macrocode}
%</package>
%
\hi@defparam{paren}{A parenthetical to be included with citation}{%
    \find@try\find@in{%
        { in: }{\hi@defkey@cmt{ in}}%
        { by: }{\hi@defkey@cmt{ by}}%
        { at: }{\hi@defkey@cmt{ at}}%
    }{#1}{%
        \hi@param@addparen\hi@paren@doc{#1}%
    }%
}
%<*doc>
The parenthetical will be included in any full citation. If the parenthetical
contains the words |in:|, |by:|, or |at:|, it is treated as commentary such that
what follows the colon is a citation string to be formatted. Thus,
|quoted in: ref| will produce a parenthetical ``(quoted in |\clause{ref}|).''
%</doc>
%<*package>
%    \begin{macrocode}
\make@find@in{ in: }
\make@find@in{ by: }
\make@find@in{ at: }
\def\hi@defkey@cmt#1#2#3{%
    \hi@param@addparen\hi@paren@comment{\hi@fn@comment{#2#1} \clause{#3}}%
}
%    \end{macrocode}
%</package>
%
\hi@defparam{parse}{Apply automatic citation parser}{\hi@cparse{#1}}
%<*doc>
The value is a formatted reference input as described in \sec{cparse}. It is
useful if that input contains an equal sign, which would otherwise cause this
package to mistakenly interpret the formatted input as parameter-value input.
%</doc>
%
\hi@defparam{reprinted}{Reprinted in parenthetical}{%
    \hi@anonref{#1}{\hi@reprinted}%
}
%<*doc>
The value should be a reference identifier or an anonymous reference defintion
as described in \sec{anonymous}.
%</doc>
%<*package>
%
% |\hi@kv@reprinted| is not used here, but is needed by \sec{hibib} in order
% to know what to place in the Bibtex file.
%
%    \begin{macrocode}
\def\hi@reprinted#1{%
    \def\hi@kv@reprinted{#1}%
    \hi@param@addparen\hi@paren@reprint{%
        \hi@fn@comment{reprinted in} \clause{#1}%
    }%
}
%    \end{macrocode}
%</package>
%
\hi@defparam{supplement}{Year/date of a supplement to the reference}{%
    \hi@date@parse[Supp.]{#1}{\def\hi@kv@supplement}%
    \hi@param@addhook{%
        \hi@ifset\hi@kv@year{%
            \addto@macro\hi@kv@year{ \& }%
            \expandafter\addto@macro\expandafter\hi@kv@year\expandafter{%
                \hi@kv@supplement
            }%
        }{%
            \let\hi@kv@year\hi@kv@supplement
        }%
    }%
}
%<*doc>
The value should be a date (see \sec{dates}), for which the date qualifier
``Supp.''\ will be added if no other qualifier is given. If both this parameter
and \param{year} are set, then the two will be joined with an ampersand (e.g.,
``1995 \& Supp.~1996'').
%</doc>
%
\hi@defparam@noval{toapage}{Include page numbers in Table of Authorities}{%
    \hi@include@page@in@toa{\@this@case}%
    \def\hi@kv@toapage{}%
}
%<*doc>
Generally, Table of Authorities entries are for individual references; this
parameter will cause an entry to be generated for each page of the reference
cited. Some reference types, such as \rtype{statcode}, do this by default.
Additionally, some reference types, such as \rtype{book}, will format Table of
Authorities entries differently when this parameter is given.
%</doc>
%
\hi@defparam{url}{URL for reference}{%
    \find@try\find@start{%
        {available at }{\hi@urlprefixwarn\KV@hi@url}%
        {at }{\hi@urlprefixwarn\KV@hi@url}%
    }{#1}{%
        \hi@parseurl{#1}\hi@kv@url
        \@expandarg\hi@defkey@url{\hi@kv@url}%
    }%
}
%<*doc>
See \sec{urls} for information on how to present the URL and how it is
formatted.
%</doc>
%<*package>
%    \begin{macrocode}
\make@find@start{available at }
\make@find@start{at }
\def\hi@defkey@url#1{%
    \hi@param@addparen\hi@paren@url{\hi@showurl{#1}}%
}
\def\hi@urlprefixwarn{%
    \PackageWarning\hi@pkgname{URL started with `at' or `available at'}%
}
%    \end{macrocode}
%</package>
%<*test>

\section{genparams.dtx}

\subsection{Parallel Cites}

\defcase{semayne}{
name=Semayne's Case,
court=K.B.,
cite=77 Eng. Rep. 194,
year=1604,
parallel=coke: 5 Coke 91a,
}

\AssertBox{\note{semayne}}{%
    \InFootnote{Semayne's Case, 77 Eng.~Rep.~194, 5 Coke 91a (K.B.
    1604).}%
}

\fullcite{semayne}

\AssertBox{\note{semayne at 195 ::coke: 91b}}{%
    \InFootnote{Semayne's Case, 77 Eng.~Rep.~194, 195, 5 Coke 91a, 91b (K.B.
    1604).}%
}

%</test>
