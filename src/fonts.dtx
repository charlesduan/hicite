%%
%% \iffalse filename: fonts.dtx \fi
%%
%<*doc>
\input driver
\thisis{fonts}{Fonts}

Legal citations use a variety of fonts in formatting citations. Complicating
matters, there is not consistency in which fonts are used in any given context.
Case names are italicized in legal memoranda but not in law review articles;
journal titles are set in small caps in some law review articles but in roman in
legal memoranda and some journals. Yet despite this variance across dialects of
legal citation, there tends to be much consistency within a single dialect:
Article titles tend to be formatted identically across a variety of article-like
references, as are book authors, signals, and so on.

To deal with this variance, the package introduces a layer of abstraction for
fonts. It first defines a set of font contexts: case names, article titles, book
authors, and so on. It then introduces a concept of ``font packs,'' namely
mappings between the font contexts and the actual fonts to be used. Reference
type definition macros do not invoke fonts directly, but rather specify font
contexts, which are then translated into actual fonts based on the font packs
the user chooses. This way, a variety of font conventions can be supported
without the need to revise reference definitions.

%</doc>
%<*package>
%
% \DescribeMacro\hi@font@context@list
% This is a list of all the permitted font contexts.
%    \begin{macrocode}
\let\hi@font@context@list\@empty
\def\hi@font@context@add#1#2{%
    \listadd\hi@font@context@list{#1}%
}
%    \end{macrocode}
%</package>
%<*doc>

\subsection{Font Contexts}

The following is a table of all font contexts supported.

\begin{demo}
\def\hi@font@context@add#1#2{#1&#2\\}
\begin{tabular}{ll}
\textbf{Name} & \textbf{Description} \\
\hline
%</doc>
%<*doc|package>
%
% \iffalse [This code is performed in both doc and package] \fi
%
\hi@font@context@add{bookauthor}{Author of a book}
\hi@font@context@add{booktitle}{Title of a book}
\hi@font@context@add{congdoc}{Congressional document number}
\hi@font@context@add{hearing}{Hearing title}
\hi@font@context@add{arttitle}{Journal article title}
\hi@font@context@add{webtitle}{Website page title}
\hi@font@context@add{jrntitle}{Journal title}
\hi@font@context@add{caseproc}{Procedural phrase in a case}
\hi@font@context@add{casefc}{Case full name}
\hi@font@context@add{casesc}{Case short name}
\hi@font@context@add{const}{Constitution name}
\hi@font@context@add{statute}{Codified statute}
\hi@font@context@add{sig}{Signal}
\hi@font@context@add{comment}{Signal before commentary}
\hi@font@context@add{ell}{The letter (l) in subsections}
\hi@font@context@add{dateprefix}{The qualifier words before a date}
\hi@font@context@add{latin}{Latin words that are typically italicized}
%
%</doc|package>
%<*doc>
\end{tabular}
\end{demo}

%
% Note that the above table was produced by executing source code, meaning that
% several lines of source code are not displayed. Consult the |dtx| or |sty|
% files to see the code itself.
%


\subsection{Fonts}

Next, the available font commands that will be linked with font
contexts are defined. Standard \LaTeX\ font commands are not used. This is to
help with keeping track of which font is being used, to manage
capitalization, and to provide a few special features.

The fonts provided are emph, empht, rm, and sc. All have meanings identical
to typical \LaTeX, except for empht which is like emph except that it does not
turn internal emph into roman.

%</doc>
%<*package>
%
% \DescribeMacro\@capemph
% Emphasizes the argument text, capitalizes the first letter, and sets a flag
% for tracking italics.
%    \begin{macrocode}
\DeclareRobustCommand\@capemph[1]{%
    \emph{\@hi@emphtrue\@capnext#1}\hi@nocap
}
\newif\if@hi@emph
%    \end{macrocode}
%
% \DescribeMacro\@capempht
% Like |\@capemph| but prevents internal |\emph| commands from de-italicizing
% text.
%    \begin{macrocode}
\DeclareRobustCommand\@capempht[1]{%
    \@test\if@hi@emph\fi{%
        \@caprm{#1}%
    }{%
        \emph{\@hi@emphtrue\let\eminnershape\relax\@capnext#1}\hi@nocap
    }%
}
%    \end{macrocode}
%
% Font of emph when non-emph, does nothing when emph.
%    \begin{macrocode}
\DeclareRobustCommand\hi@maybeemph{%
    \if@hi@emph\expandafter\@iden\else\expandafter\@capemph\fi
}
%    \end{macrocode}
%
% \DescribeMacro\@capsc \DescribeMacro\@caprm
% Like |\@capemph| but for different fonts.
%    \begin{macrocode}
\DeclareRobustCommand\@capsc{\@capfont\textsc}
\DeclareRobustCommand\@caprm{\@capfont\@iden}
%    \end{macrocode}
%
%</package>
%<*doc>

\subsection{Font Packs}

\DescribeMacro\MakeFontPack
A font pack associates font contexts with font commands. Each context receives
two font commands: one for use in non-inline citations, and one for use with
inline texts. The syntax for defining a font pack is as follows:
\begin{grammar}
|\MakeFontPack|\marg{name}\marg{font-specs} \\
\meta{font-specs} := \meta{font-spec} ( \littext{, } \meta{font-spec} )* \\
\meta{font-spec} := \meta{context} \littext{: } \meta{font} \littext{/}
\meta{font} \\
\meta{font} := \littext{emph} \| \littext{empht} \| \littext{rm} \| \littext{sc}
\\
\Example |\MakeFontPack{new}{booktitle: sc/empht, bookauthor: sc/rm}|
\end{grammar}
A font pack need not define every font context, since a user can select multiple
font packs. In the case of two font packs containing the same font context, the
last selected pack overrides earlier ones.

\DescribeMacro\UseFontPack
To invoke a font pack, call |\UseFontPack|\marg{name}.

%</doc>
%<*package>
%
%    \begin{macrocode}
\def\MakeFontPack#1#2{%
    \csgdef{hi@fontpack@#1}{\forcsvlist\hi@font@dopack{#2}}%
}
\def\UseFontPack#1{\csname hi@fontpack@#1\endcsname}
%    \end{macrocode}
%
% These macros create the font macros themselves. The macros are structured as
% |\hi@fn@|\meta{context}, which is defined robustly to select the right font
% between citation and inline contexts. |\hi@font@dopack| is run for each item
% in the pack; it parses the content, constructs the context and font macros,
% and then defines the font context macro accordingly.
%
%    \begin{macrocode}
\def\hi@font@dopack#1{\ifblank{#1}{}{\hi@font@dopack@#1\par}}
% Use |\par| as the delimiter to catch errors early
\def\hi@font@dopack@#1: #2/#3\par{%
    \edef\reserved@a{%
        \noexpand\hi@font@dopack@elt
            \expandonce{\csname hi@fn@#1\endcsname}%
            \expandonce{\csname @cap#2\endcsname}%
            \expandonce{\csname @cap#3\endcsname}%
    }\reserved@a
}
% \#1 is |\hi@fn@|\meta{context}, \#2 is the cite font macro, \#3 is the inline
% font macro
\def\hi@font@dopack@elt#1#2#3{%
    \DeclareRobustCommand#1{\@test \if@hi@inline \fi{#3}{#2}}%
}
%    \end{macrocode}
%
%</package>
%<*doc>

The following font packs are predefined:
\begin{itemize}
\item \textbf{review}: Law review fonts.
\item \textbf{memo}: Legal memorandum fonts.
\item \textbf{italcase}: Full case names are always italicized. Select this
after review or memo.
\end{itemize}

%</doc>
%<*package>
%
% Make the font packs.
%
%    \begin{macrocode}
\MakeFontPack{review}{%
    bookauthor: sc/rm,
    booktitle: sc/rm,
    congdoc: sc/empht,
    hearing: empht/empht,
    arttitle: empht/empht,
    webtitle: empht/empht,
    jrntitle: sc/empht,
    caseproc: empht/empht,
    casefc: rm/empht,
    casesc: empht/empht,
    const: sc/rm,
    statute: sc/rm,
    sig: emph/rm,
    comment: emph/rm,
    ell: emph/emph,
    dateprefix: rm/rm,
    latin: emph/emph%
}
\MakeFontPack{memo}{%
    bookauthor: rm/rm,
    booktitle: empht/empht,
    congdoc: rm/empht,
    hearing: empht/empht,
    arttitle: empht/empht,
    webtitle: empht/empht,
    jrntitle: rm/empht,
    caseproc: empht/empht,
    casefc: empht/empht,
    casesc: empht/empht,
    const: rm/rm,
    statute: rm/rm,
    sig: emph/rm,
    comment: emph/rm,
    ell: emph/emph,
    dateprefix: rm/rm,
    latin: emph/emph%
}
\MakeFontPack{italcase}{%
    caseproc: empht/empht,
    casefc: empht/empht,
    casesc: empht/empht,
}
\MakeFontPack{italdateprefix}{%
    dateprefix: empht/empht,
}
%    \end{macrocode}
%
% Install hooks for the font packs, and call memo fonts by default.
%
%    \begin{macrocode}
\UseFontPack{memo}
\appto\hi@hooks@review{\UseFontPack{review}}
\appto\hi@hooks@memo{\UseFontPack{memo}}
%    \end{macrocode}
%
% \DescribeMacro\hi@italdateprefix
% Given an argument, temporarily chooses the italdateprefix font pack (so that
% date qualifier words are italicized) and executes the argument.
%
%    \begin{macrocode}
\DeclareRobustCommand\hi@italdateprefix[1]{%
    \begingroup
        \UseFontPack{italdateprefix}#1%
    \endgroup
}
%    \end{macrocode}
%
%</package>
%<*doc>

\subsection{Shortcut Commands for Fonts}

\DescribeMacro\UseFontFor
Occasionally, the user needs to select a font manually, for example when
defining a hereinafter form that includes a part of a journal article's title.
The command |\UseFontFor|\marg{context}\marg{text} selects the font for
\meta{context} and sets \meta{text} in that font.

%</doc>
%<*package>
%    \begin{macrocode}
\DeclareRobustCommand\UseFontFor[2]{%
    \ifinlist{#1}{\hi@font@context@list}{%
        \csname hi@fn@#1\endcsname{#2}%
    }{%
        \PackageWarning\hi@pkgname{Font context `#1' is unknown}%
        #2%
    }%
}
%    \end{macrocode}
%
%</package>
%<*test>

\section{fonts.dtx}

\begingroup
    \UseFontPack{memo}
    \AssertBox{\UseFontFor{booktitle}{Book Title}}{\emph{Book Title}}
\endgroup

\begingroup
    \UseFontPack{review}
    \AssertBox{\UseFontFor{booktitle}{Book Title}}{\textsc{Book Title}}
    \AssertBox{\UseFontFor{casefc}{Case Title}}{Case Title}

    \UseFontPack{italcase}
    \AssertBox{\UseFontFor{casefc}{Case Title}}{\emph{Case Title}}
\endgroup

%</test>
