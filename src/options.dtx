%%
%% \iffalse filename: options.dtx \fi
%%
%<*doc>
\input driver
\thisis{options}{Package Options}

The following options are accepted by this package.

\ExplainOption{review} Law review formatting: selects law review fonts, expects
footnotes, formats \emph{supra} cites with note numbers, and so on.

%</doc>
%<*package>
%    \begin{macrocode}
\DeclareOption{review}{\hi@hooks@review}
%    \end{macrocode}
%</package>
%<*doc>

\ExplainOption{journal} Alias for \opt{review}.

%</doc>
%<*package>
%    \begin{macrocode}
\DeclareOption{journal}{\hi@hooks@review}
%    \end{macrocode}
%</package>
%<*doc>

\ExplainOption{journalfonts} Selects law review fonts. Same as
|\UseFontPack{review}|. Other formatting aspects are not changed.

%</doc>
%<*package>
%    \begin{macrocode}
\DeclareOption{journalfonts}{\UseFontPack{review}}
%    \end{macrocode}
%</package>
%<*doc>

\ExplainOption{memofonts} Selects legal memorandum fonts. Same as
|\UseFontPack{memo}|. Other formatting aspects are not changed.

%</doc>
%<*package>
%    \begin{macrocode}
\DeclareOption{memofonts}{\UseFontPack{memo}}
%    \end{macrocode}
%</package>
%<*doc>

\ExplainOption{memo} Legal memorandum formatting: selects memorandum fonts,
expects citations in text, and so on.

%</doc>
%<*package>
%    \begin{macrocode}
\DeclareOption{memo}{\hi@hooks@memo}
%    \end{macrocode}
%</package>
%<*doc>

\ExplainOption{brief} Alias for \opt{memo}.


%</doc>
%<*package>
%    \begin{macrocode}
\DeclareOption{brief}{\hi@hooks@memo}
%    \end{macrocode}
%</package>
%<*doc>

\ExplainOption{endnote} Uses endnotes rather than footnotes for formatting.

%</doc>
%<*package>
%    \begin{macrocode}
\DeclareOption{endnote}{\hi@useendnotes}
%    \end{macrocode}
%</package>
%<*doc>

\ExplainOption{endnotes} Same as \opt{endnote}.

%</doc>
%<*package>
%    \begin{macrocode}
\DeclareOption{endnotes}{\hi@useendnotes}
%    \end{macrocode}
%</package>
%<*doc>

\ExplainOption{fullurl} URLs are displayed with their full text.

%</doc>
%<*package>
%    \begin{macrocode}
\DeclareOption{fullurl}{\hi@urlstyle@full}
%    \end{macrocode}
%</package>
%<*doc>

\ExplainOption{toaurl} URLs are displayed with their full text in the table of
authorities and a shortcut otherwise, as described in \sec{urls-appearance}.

%</doc>
%<*package>
%    \begin{macrocode}
\DeclareOption{toaurl}{\hi@urlstyle@toa}
%    \end{macrocode}
%</package>
%<*doc>

\ExplainOption{linkurl} URLs are displayed as a linked word, as described in
\sec{urls-appearance}.

%</doc>
%<*package>
%    \begin{macrocode}
\DeclareOption{linkurl}{\hi@urlstyle@link}
%    \end{macrocode}
%</package>
%<*doc>

\ExplainOption{italcase} Case names are italicized regardless of font packs
previously chosen.

%</doc>
%<*package>
%    \begin{macrocode}
\DeclareOption{italcase}{\UseFontPack{italcase}}
%    \end{macrocode}
%</package>
%<*doc>

\ExplainOption{toastar} Stars are added to common references in the table of
authorities, as described in \sec{toa-star}.

%</doc>
%<*package>
%    \begin{macrocode}
\DeclareOption{toastar}{\hi@toa@starcount=4\relax}
%    \end{macrocode}
%</package>
%<*doc>

\ExplainOption{somenotename} In general, if a case is named inline in text and
a citation to the case is made immediately thereafter, then the case name is
suppressed in the citation so as to avoid redundancy. This option makes an
exception where the citation to the case is the first citation in a footnote. In
such cases, the inline use was in text potentially distant from the footnote, so
naming the case again in the footnote is less redundant and potentially helpful
to readers.

%</doc>
%<*package>
%    \begin{macrocode}
\DeclareOption{somenotename}{\hi@noname@sometimes}
%    \end{macrocode}
%</package>
%<*doc>

\ExplainOption{nonotename} Suppresses case names even in citations at the
start of footnotes. This is typical of most legal journals.

%</doc>
%<*package>
%    \begin{macrocode}
\DeclareOption{nonotename}{\hi@noname@always}
%    \end{macrocode}
%</package>
%<*doc>

\ExplainOption{useopturl} Redefines the \param{opturl} parameter to be the same
as \param{url}, as described in the documentation for the former parameter.

%</doc>
%<*package>
%    \begin{macrocode}
\DeclareOption{useopturl}{\hi@param@alias{opturl}{url}}
%    \end{macrocode}
%
%</package>
%<*doc>

\ExplainOption{noterefs} Cross-references to text will refer to the closest
footnote numbers. This is typical of law reviews and is the standard when the
\opt{review} option is selected.

%</doc>
%<*package>
%    \begin{macrocode}
\DeclareOption{noterefs}{\hi@xref@opt@noterefs}
%    \end{macrocode}
%
%
%</package>
%<*doc>

\ExplainOption{pagerefs} Cross-references to text will refer to the page
numbers on which the relevant text is found. This is typical of legal memoranda
and is the standard when the \opt{memo} option is selected.

%</doc>
%<*package>
%    \begin{macrocode}
\DeclareOption{pagerefs}{\hi@xref@opt@pagerefs}
%    \end{macrocode}
%
%
%
%
% \paragraph{Hooks for Memo and Review Formats}
% The choice between review and memo style triggers a variety of differences in
% citation processing, so provided here are two hooks that other components of
% this package can add to for special processing depending on style. The hooks
% will be executed at the end of the package.
%
%    \begin{macrocode}
\def\hi@hooks@review{}
\def\hi@hooks@memo{}
%    \end{macrocode}
%</package>
