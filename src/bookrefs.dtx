%%
%% \iffalse filename: bookrefs.dtx \fi
%%
%<*doc>
\input driver
\thisis{bookrefs}{Books and Containers}

This section describes two reference types for larger book-like works, and also
a common parenthetical used across many citation types.


\subsection{Publication Parentheticals}

\label{s:bookrefs-pubparen}

Many citation types accept a common parenthetical that contains a date of
publication and other publication information, such as editions, editors, and
publishers. This standard parenthetical accepts the following parameters, all of
which are optional (although some reference types will require \param{year}):

\optparameters{
    {editor}{Names of one or more editors of the book.}
    {edtype}{The abbreviation to follow the editor names, by default ``ed.''\ or
    ``eds.''}
    {number}{A serial number for the book. This parameter is used only if
    \param{publisher} is also given. (For \rtype{book} references, a serial
    number with no publisher will be used differently.)}
    {edition}{The edition number of the book. An ordinal suffix will be added.}
    {publisher}{The name of the publisher.}
    {forthcoming}{If set, the word ``forthcoming'' will be added to the
    parenthetical.}
    {year}{The year of publication.}
}

%</doc>
%<*package>
%
% Reference type definers should call this macro before creating citation format
% macros, to set up |\hi@book@paren|. They must then include |\hi@book@paren| in
% the content for the date parenthetical.
%
%    \begin{macrocode}
\def\hi@book@pubparen{%
    \let\hi@book@paren\@empty  % Date parenthetical
    \let\hi@book@sep\@empty % What separator to add next
    \hi@ifset\hi@kv@editor{%
        \@expandarg\hi@book@pubparen@add{\hi@kv@editor}{, }%
        \hi@ifset\hi@kv@edtype{%
            \ifx\hi@kv@edtype\@empty\else
                \addto@macro\hi@book@paren{ }%
                \add@macro@to@macro\hi@book@paren\hi@kv@edtype
            \fi
        }{%
            \@expand{\find@in@cs{hi@namelist@and}}{\hi@kv@editor}{i}{%
                \addto@macro\hi@book@paren{ eds.}\@gobbletwo
            }{%
                \@expand{\find@end{ et al.}}{\hi@kv@editor}{i}{%
                    \addto@macro\hi@book@paren{ eds.}\@gobble
                }{%
                    \addto@macro\hi@book@paren{ ed.}%
                }%
            }%
        }%
    }{}%
    \hi@ifset\hi@kv@publisher{%
        \@expandarg\hi@book@pubparen@add{\hi@kv@publisher}{ }%
        \hi@ifset\hi@kv@number{%
            \@expandarg\hi@abbrev@pub{\hi@kv@number}{\hi@book@pubparen@add}{, }%
        }{}%
    }{}%
    \hi@ifset\hi@kv@edition{%
        \@expandarg\hi@book@pubparen@add{\hi@kv@edition}{ }%
    }{}%
    \hi@ifset\hi@kv@forthcoming{%
        \hi@book@pubparen@add{forthcoming}{ }%
    }{}%
    \hi@ifset\hi@kv@year{%
        \ifx\hi@book@paren\@empty\else
            \add@macro@to@macro\hi@book@paren\hi@book@sep
        \fi
        \addto@macro\hi@book@paren{\hi@pstruct@use{year}}%
    }{}%
    \ifx\hi@book@paren\@empty\else
        \edef\hi@book@paren{%
            \noexpand\hi@parens@add\noexpand\hi@paren@date{%
                \expandonce{\hi@book@paren}%
            }%
        }%
    \fi
}
%    \end{macrocode}
%
% Adds an item to the book publication parenthetical. \#1 is the text to add,
% and \#2 is the separator.
%
%    \begin{macrocode}
\def\hi@book@pubparen@add#1#2{%
    \ifx\hi@book@paren\@empty\else \addto@macro\hi@book@paren{, }\fi
    \addto@macro\hi@book@paren{#1}%
    \def\hi@book@sep{#2}%
}
%    \end{macrocode}
%</package>
%<*doc>

\input docparams

%</doc>
\hi@newcite\defbook{A book or non-periodic material}{%
    \hi@citetype{#1}\hi@type@other
    \hi@book@pubparen
    \expandafter\hi@book@titlethe\expandafter{\hi@kv@name}%
    \hi@newcite@form{fc}{#1}{%
        \hi@pstruct@initialize\@this@vol
        \hi@ifset\hi@kv@author{%
            \noexpand\@gobble{\hi@kv@authln}%
        }{}% For sorting purposes
        \hi@citeguts{%
            \noexpand\hi@citecontainer@suppress{%
                \noexpand\hi@usevol{%
                    \hi@ifset\hi@kv@vol{\hi@kv@vol\space}{}%
                }%
                \hi@ifset\hi@kv@author{%
                    \hi@pstruct@use@font{author}\hi@fn@bookauthor, % space
                }{}%
            }%
            \hi@ifset\hi@kv@number{%
                \hi@ifset\hi@kv@publisher{}{%
                    \noexpand\hi@fn@booktitle{%
                        \hi@abbrev@pub{\hi@kv@number}{\@iden}%
                    }, % space
                }%
            }{}%
            \hi@pstruct@use@font{name}\hi@fn@booktitle
            \hi@maybepage{\hi@page@space\hi@kv@name}%
        }%
        \hi@book@paren
        \the\hi@param@parens
    }%
    \hi@ifset\hi@kv@toapage{%
        \hi@include@page@in@toa{#1}%
        \hi@newcite@form{lc}{#1}{%
            \noexpand\hi@toa@duptitle{}{%
                \noexpand\hi@fn@bookauthor{\relax\hi@kv@author@sortable},\space
                \noexpand\hi@usevol{\hi@ifset\hi@kv@vol{\hi@kv@vol\space}{}}%
                \hi@ifset\hi@kv@number{%
                    \noexpand\hi@fn@booktitle{\hi@kv@number}, % space
                }{}%
                \hi@fn@booktitle{\hi@kv@name}%
            }{%
                \hi@maybepage{\hi@page@space\hi@kv@name}%
                % The \reserved@a is to get the parens outside of \hi@citeguts
                % which is added kn \hi@toa@duptitle
                \gdef\noexpand\reserved@a{%
                    \hi@book@paren
                    \the\hi@param@parens
                }%
                \noexpand\aftergroup\noexpand\reserved@a
            }{%
                \hi@maybepage{ }%
            }%
        }%
    }{%
        \hi@newcite@form{lc}{#1}{%
            \hi@citeguts{%
                \hi@ifset\hi@kv@author{%
                    \hi@toa@dupauthor{\hi@fn@bookauthor}{%
                        \hi@kv@author@sortable
                    }%
                }{\hi@toa@dupnone}%
                \noexpand\hi@usevol{\hi@ifset\hi@kv@vol{\hi@kv@vol\space}{}}%
                \hi@ifset\hi@kv@number{%
                    \noexpand\hi@fn@booktitle{\hi@kv@number}, % space
                }{}%
                \hi@fn@booktitle{\hi@kv@name}%
            }%
            \hi@parens@add\hi@paren@date{%
                \hi@book@paren
                \hi@pstruct@use{year}%
            }%
            \the\hi@param@parens
        }%
    }%
    \hi@ifset\hi@kv@authln{}{\let\hi@kv@usetitle\relax}%
    \protected@edef\reserved@a{%
        \hi@ifset\hi@kv@usetitle
            {\noexpand\hi@fn@booktitle{\hi@kv@name}}%
            {\noexpand\hi@fn@bookauthor{\hi@kv@authln}}%
    }%
    \@expand{\hi@supra@form{#1}}\reserved@a i%
}
%<*doc>

\keyparameters{
    {author, name}{The author(s) and title of the work.}
    {vol}{A specific volume number being cited. Preferably, though, the volume
    number would be included as part of the citation item information.}
    {number}{A serial number for the book. If the serial number is attached to
    the work's author (i.e., it's an institutional series), then enter the
    serial number alone, protecting any commas with braces. If the serial number
    is attached to a publisher's series, then use a comma as described with
    regard to the publication parenthetical.}
}
\optparameters{
    {struct}{If volumes of the work have different authors, titles, or other
    information, provide the differing values in a struct as described in
    \sec{struct}.}
}
\bookparenparameters

%</doc>
%<*package>
%    \begin{macrocode}
\make@find@start{The }
\def\hi@book@titlethe#1{%
    \find@start{The }{#1}{\hi@book@titlethe@}{}%
}
\def\hi@book@titlethe@#1{%
    \def\hi@kv@name{\hi@book@the#1}%
}
\DeclareRobustCommand\hi@book@the{The }
\make@find@in{, }
%    \end{macrocode}
%</package>
\hi@newcite\defcitecontainer{A citation in another citation}{%
    %
    % Set the joining phrase.
    %
    \hi@citecontainer@settype
    %
    % Perform initial matters to set up the citecontainer; see the comment
    % before \hi@citecontainer@init.
    \edef\reserved@a{%
        {#1}%
        \expandafter\noexpand\csname tc@#1\endcsname
        \hi@ifset\hi@kv@citation{%
            \expandafter\noexpand\csname tc@\hi@kv@citation\endcsname
            \expandafter\noexpand\csname fc@\hi@kv@citation\endcsname
        }{\relax\relax}%
    }%
    \expandafter\hi@citecontainer@init\reserved@a
    %
    % Prepare the full citation form.
    \edef\reserved@a{%
        {#1}%
        \expandafter\noexpand\csname fc@#1\endcsname
        \expandafter\noexpand\csname tcpg@\hi@kv@in\endcsname
    }%
    \expandafter\hi@citecontainer@fc\reserved@a
    %
    % Prepare the short citation form.
    \edef\reserved@a{%
        {#1}%
        \expandafter\noexpand\csname sc@#1\endcsname
        \hi@ifset\hi@kv@citation{%
            \expandafter\noexpand\csname sc@\hi@kv@citation\endcsname
        }{\relax}%
    }%
    \expandafter\hi@citecontainer@sc\reserved@a
    %
    % Cite container TOA form.
    %
    \edef\reserved@a{%
        {#1}%
        \expandafter\noexpand\csname lc@#1\endcsname
        \hi@ifset\hi@kv@citation{%
            \expandafter\noexpand\csname
                \@ifundefined{lc@\hi@kv@citation}{fc}{lc}@\hi@kv@citation
            \endcsname
        }{\relax}%
        %
        % We always use the fullcite form for the container, not the TOA form.
        % This is because the TOA form is sometimes special in ways that are not
        % appropriate to being the second citation in a string cite, and also
        % because the TOA forms sometimes don't account for things like page
        % numbers.
        \expandafter\noexpand\csname fc@\hi@kv@in\endcsname
        %
        % This macro enables storing the volume/page number for the container
        % citation.
        \expandafter\noexpand\csname hi@citecontainer@data@#1\endcsname
    }
    \expandafter\hi@citecontainer@lc\reserved@a
}
%<*doc>
A cite container is a highly flexible vehicle for citing works that are
contained inside other larger works: chapters in edited volumes, introductions
to books, letters reprinted in compilations, and documents in appendices to
judicial opinions. The defining feature of the cite container is that the
pagination of the contained work follows the pagination of the container, such
that any pin cite needs to be attached to the container's locator information.

There are two flavors of cite containers. First, the contained item may be a
standalone reference with a type, such as a letter or case. If so, then the
contained item is defined as a reference and passed to the citation container's
\param{citation} parameter. \unskip\footnote{This may be done through anonymous
references as described in \sec{anonymous}.}
For example, the following would cite a letter in a volume of collected works:
\begin{demo}
\ttfamily
|\defletter|\{ltr\}\{\\
|    |author=Benjamin Franklin,\\
|    |to=George Washington,\\
|    |\ldots\\
\} \\
|\defbook|\{works\}\{\\
|    |name=Collected Works of Franklin,\\
|    |\ldots\\
\} \\
|\defcitecontainer|\{franklin-ltr\}\{\\
|    |citation=ltr, \\
|    |in=works, \\
|    |page=107, \\
\}
\end{demo}
A command of |\sentence{franklin-ltr at 109}| would produce something along the
lines of ``Letter from Benjamin Franklin to George Washington, \emph{in}
\textsc{Collected Works of Franklin} 107, 109.''

Second, the contained item may not require a standalone definition, as would be
the case for a book chapter. In that case, the cite container takes parameters
\param{author}, \param{name}, and \param{year} to specify the contained work's
information, instead of the \param{citation} parameter.

\keyparameters{
    {citation}{The reference name or anonymous reference definition for the
    contained work.}
    {author, name, year}{Information for the contained work, used in lieu of
    \param{citation}.}
    {in}{The reference name or anonymous reference definition for the container
    work.}
    {vol, page}{The pin cite information for the contained work inside the
    container (e.g., for a chapter in an edited volume, the volume and page
    number where the chapter begins).}
}
\optparameters{
    {type}{The preposition that should join the contained work and the
    container; default is ``in.'' If the word is ``to'' or ``of,'' then no comma
    is prepended and the word is set in roman type, as is useful for
    introductions or forewords.}
    {singleauthor}{Indicates that all works in the contained volume are by the
    same author, such that the author name should be set in the font used for
    book authors.}
    {inline}{A short form name for the citation. The \param{hereinafter}
    parameter is probably better in most situations.}
}

%</doc>
%<*package>
%
% Sets the type parameter, |\hi@kv@type|, as the connector phrase between the
% contained and container citations. If it is ``to'' or ``of'', then it is set
% in roman and no comma is prepended. Otherwise, it is set in roman and a comma
% is prepended. If no connector is given, it defaults to ``in''.
%
%    \begin{macrocode}
\def\hi@citecontainer@settype{%
    \hi@ifset\hi@kv@type{%
        \@expand{\find@try\find@eq{%
            {to}{\def\hi@kv@type{ to}}%
            {of}{\def\hi@kv@type{ of}}%
        }}\hi@kv@type i{%
            \@expand{\def\hi@kv@type}{\expandafter{\hi@kv@type}}i%
            \prepend@macro\hi@kv@type{, \noexpand\hi@fn@sig}%
        }%
    }{%
        \def\hi@kv@type{, \noexpand\hi@fn@sig{in}}%
    }%
}
%    \end{macrocode}
%
%
% Initializing a citecontainer involves two steps.
% \begin{enumerate}
% \item Set the TOA type |\tc@|\meta{\#1}, |\hi@citecontainer@toks| to the
% contained citation matter, and |\hi@kv@inline| as necessary.
% \item Set |\hi@kv@in|, if it was put into |\hi@kv@rep| instead.
% \end{enumerate}
% \#1 is the reference name; \#2 is the macro
% |\tc@|\meta{\#1}, \#3 is the macro |\tc@|\meta{\string\hi@kv@citation}, \#4 is
% |\fc@|\meta{\string\hi@kv@citation}.
% (If there is no |\hi@kv@citation|, \#3 and \#4 are |\relax|.)
%
%    \begin{macrocode}
\def\hi@citecontainer@init#1#2#3#4{%
    %
    % 2. If \hi@kv@citation is set, then the contained element is a citation.
    % Otherwise, the contained element is a generic article within the work, so
    % a citation form is constructed for it.
    %
    % The relevant contents from the contained citation are placed in
    % \hi@citecontainer@toks, to be added subsequently to the \fc@[#1] macro.
    %
    \hi@ifset\hi@kv@citation{%
        % In the case that the contained element is a citation:
        %
        % The TOA type for the citecontainer should be the TOA type of the
        % contained element.
        \global\let#2#3%
        %
        % No TOA entry should be made for the contained element alone.
        \global\let#3\relax
        %
        % Set \hi@citecontainer@toks to the (already-created) full-cite material
        % of \fc@[citation].
        \expandafter\hi@citecontainer@toks\expandafter{#4}%
    }{%
        % In the case that no citation is provided:
        %
        % Set up citecontainer with the author, name, and date.
        \protected@edef\reserved@a{%
            \hi@citecontainer@toks={%
                \hi@ifset\hi@kv@author{%
                    \hi@ifset\hi@kv@singleauthor{%
                        \noexpand\@gobble{\hi@kv@authln}%
                        \hi@ifset\hi@kv@vol{\hi@kv@vol\space}{}%
                        \noexpand\hi@fn@bookauthor{\hi@kv@author}%
                    }{%
                        \hi@kv@author@sortable
                    }%
                    ,\space
                }{}%
                \noexpand\hi@fn@arttitle{\hi@kv@name}%
                \hi@ifset\hi@kv@year{%
                    \noexpand\hi@parens@add\noexpand\hi@paren@date{\hi@kv@year}%
                }{}%
                %
                % Any parentheticals attached to the citation itself get placed
                % from here.
                \the\hi@param@parens
            }%
        }\reserved@a
        %
        % Set up the short-form information for this citation.
        \hi@ifset\hi@kv@inline{}{%
            \hi@ifset\hi@kv@authln{%
                \hi@ifset\hi@kv@singleauthor{%
                    \protected@edef\hi@kv@inline{%
                        \noexpand\hi@fn@bookauthor{\hi@kv@authln}%
                    }%
                }{%
                    \let\hi@kv@inline\hi@kv@authln
                }%
            }{%
                \protected@edef\hi@kv@inline{%
                    \noexpand\hi@fn@arttitle{\hi@kv@name}%
                }%
            }%
        }%
        %
        % Set the TOA cite type to 'other'.
        \hi@citetype{#1}\hi@type@other
    }%
    %
    % 3. The citation reference for the container should be in the citation
    % parameter 'in' or 'rep'.
    \hi@ifset\hi@kv@in{\let\hi@kv@rep\hi@kv@in}{%
        \hi@ifset\hi@kv@rep{%
            \let\hi@kv@in\hi@kv@rep
        }{%
            \PackageError\hi@pkgname{%
                Missing `in' citation for citecontainer #1
            }{%
                A reference to a case is required for \string\defcitecontainer
            }%
        }%
    }%
}
\make@find@in{ at }
\make@find@eq{to}
\make@find@eq{of}
\newtoks\hi@citecontainer@toks
%    \end{macrocode}
%
% Make a citecontainer full cite form. \#1 is the reference name, \#2 is
% |\fc@|\meta{\#1}, \#3 is |\tcpg@|\meta{\string\hi@kv@in}.
%
%    \begin{macrocode}
\def\hi@citecontainer@fc#1#2#3{%
    %
    % Insert the contained citation. This is done in a group, with pincite
    % information omitted since that information pertains to the container.
    %
    \hi@newcite@form{fc}{#1}{%
        \begingroup
            \let\noexpand\@this@page\relax
            \let\noexpand\@this@vol\relax
            \let\noexpand\@this@opt\relax
            %
            % We reset the parentheticals list here, which clears out any
            % citation-level parentheticals. \hi@citecontainer@toks will insert
            % parentheticals associated with the contained reference, which will
            % then be shown with the subsequent \hi@parens@show command.
            \noexpand\hi@parens@reset
            \the\hi@citecontainer@toks
            \noexpand\hi@parens@show
        \endgroup
        %
        % This is the end of the contained citation. Ensure that capitalization
        % is turned off.
        \hi@nocap
        %
        % Insert the transitional phrase between the container and contained
        % material.
        \hi@kv@type\space
        \begingroup
            % If singleauthor is set (which only makes sense if citation is not
            % set), instruct the book citation to suppress display of the author
            % and volume.
            \hi@ifset\hi@kv@citation{}{\hi@ifset\hi@kv@singleauthor{%
                \let\noexpand\hi@citecontainer@suppress\noexpand\@gobble
            }{}}%
            %
            % Insert the contents of the container citation. This is delegated
            % to the macro \hi@citecontainer@addpg.
            \let\noexpand#3\relax % Turn off page numbers for TOA references
            \hi@citecontainer@addpg{%
                \hi@ifset\hi@kv@vol{\hi@kv@vol\space}{}%
                \hi@kv@in
            }{%
                \space at\space
                \hi@ifset\hi@kv@page{\hi@kv@page, }{}%
            }{%
                \hi@ifset\hi@kv@page{ at \hi@kv@page}{}%
            }%
        \endgroup
        %
        % Add short citation form records for the container, so that way other
        % uses of the container use the short form.
        \hi@ifset\hi@kv@citation{\hi@record@cite{\hi@kv@citation}}{}%
        %
        % This may be duplicative since |\clause| will already call
        % |\hi@record@cite|; need to check
        %
        \hi@record@cite{\hi@kv@in}%
    }%
}
%    \end{macrocode}
%
% This flag determines whether other citations should suppress display of the
% volume number and author (because they've already been included in the
% \cs\citecontainer).
%
%    \begin{macrocode}
\let\hi@citecontainer@suppress\@iden
%    \end{macrocode}
%
% Cite container short form. \#1 is the reference name, \#2 is |\sc@|\meta{\#1},
% \#3 is |\sc@|\meta{\cs\hi@kv@citation} (or \cs\relax).
%
%    \begin{macrocode}
\def\hi@citecontainer@sc#1#2#3{%
    %
    % Short citation form. If a form is given in the 'inline' parameter, use
    % that. (If the contained material was an author/title, then 'inline' was
    % set already above.) Otherwise, we want to copy the citation form for the
    % contained reference.
    \hi@ifset\hi@kv@inline{%
        \@expand{\hi@supra@form{#1}}\hi@kv@inline i%
    }{%
        \hi@newcite@form{sc}{#1}{}%
        \global\let#2#3%
    }%
}
%    \end{macrocode}
%
% Cite container TOA form. \#1 is the reference name, \#2 is |\lc@|\meta{\#1},
% \#3 is |\lc@|\meta{|\hi@kv@citation|} (or |\fc@|\meta{\cs\hi@kv@citation}, or
% \cs\relax, depending on what is defined), \#4 is |\lc@|\meta{\cs\hi@kv@in} (or
% |\fc@|\meta{\cs\hi@kv@in}).
%
%    \begin{macrocode}
\def\hi@citecontainer@lc#1#2#3#4#5{%
    %
    % Set up the token lists
    \hi@ifset\hi@kv@citation{%
        \expandafter\hi@citecontainer@toks\expandafter{#3}%
    }{}%
    \expandafter\@temptokena\expandafter{#4}%
    %
    % Make the citation form
    \hi@newcite@form{lc}{#1}{%
        %
        % Unlike with the fc form, \hi@parens@reset is not necessary here
        % because there are no citation-level parentheticals.
        \the\hi@citecontainer@toks
        \noexpand\hi@parens@show
        % Because the container citation is not separated from the contained by
        % a group, we reset the parens here so that any of them shown thus far
        % are not repeated.
        \noexpand\hi@parens@reset
        \noexpand\hi@nocap \hi@kv@type\space
        %
        % Make the container citation. To do so, we set the volume and page (in
        % the macro for this citecontainer) and then dereference \@temptokena,
        % which houses the container citation matter.
        \hi@ifset\hi@kv@citation{}{\hi@ifset\hi@kv@singleauthor{%
            \let\noexpand\hi@citecontainer@suppress\noexpand\@gobble
        }{}}%
        #5\the\@temptokena
        \let\noexpand\hi@citecontainer@suppress\noexpand\@iden
    }%
    \global\let#2#2%
    %
    % Make the volume/page data.
    \protected@edef#5{%
        \hi@ifset\hi@kv@vol{\def\noexpand\@this@vol{\hi@kv@vol}}{}%
        \hi@ifset\hi@kv@page{%
            \def\noexpand\@this@orig@page{\hi@kv@page}%
            \@ifundefined{pc@#1}{%
                \def\noexpand\@this@page{\@format@page@macro\hi@kv@page}%
            }{%
                \@expand{\@nameuse{pc@#1}}\hi@kv@page i%
            }%
        }{}%
    }%
    \global\let#5#5%
}
%    \end{macrocode}
%
% \#1 is the initial citation text; \#2 is the text to add if a page number is
% given; \#3 is the text to add if no page number is given.
%
%    \begin{macrocode}
\DeclareRobustCommand\hi@citecontainer@addpg[3]{%
    \@test\ifx\@this@page\relax\fi{%
        \clause{#1#3}\hi@clause@endflag
    }{%
        \def\reserved@a{#1#2}%
        \add@macro@to@macro\reserved@a\@this@orig@page
        \expandafter\clause\expandafter{\reserved@a}\hi@clause@endflag
    }%
}
%    \end{macrocode}
%</package>
%<*test>

\section{bookrefs.dtx}

\subsection{Publication Parenthetical}

\begingroup
    \hi@param@clear

    \hi@book@pubparen
    \AssertMacro\hi@book@paren{}

    \def\hi@kv@year{2010}
    \hi@book@pubparen
    \AssertMacro\hi@book@paren{%
        \hi@parens@add\hi@paren@date{\hi@pstruct@use{year}}%
    }%

    \def\hi@kv@edition{3d ed.}
    \hi@book@pubparen
    \AssertMacro\hi@book@paren{%
        \hi@parens@add\hi@paren@date{3d ed. \hi@pstruct@use{year}}%
    }%

    \def\hi@kv@publisher{Doubleday}
    \hi@book@pubparen
    \AssertMacro\hi@book@paren{%
        \hi@parens@add\hi@paren@date{%
            Doubleday, 3d ed. \hi@pstruct@use{year}%
        }%
    }%

    \KV@hi@editor{E E}
    \KV@hi@editor{F F}
    \hi@book@pubparen
    \AssertMacro\hi@book@paren{%
        \hi@parens@add\hi@paren@date{%
            E E\hi@namelist@and F F eds.,
            Doubleday, 3d ed. \hi@pstruct@use{year}%
        }%
    }%

    \KV@hi@edtype{translators}
    \hi@book@pubparen
    \AssertMacro\hi@book@paren{%
        \hi@parens@add\hi@paren@date{%
            E E\hi@namelist@and F F trans.,
            Doubleday, 3d ed. \hi@pstruct@use{year}%
        }%
    }%

    \KV@hi@forthcoming{}
    \hi@book@pubparen
    \AssertMacro\hi@book@paren{%
        \hi@parens@add\hi@paren@date{%
            E E\hi@namelist@and F F trans.,
            Doubleday, 3d ed., forthcoming \hi@pstruct@use{year}%
        }%
    }%

    \hi@param@clear

    \KV@hi@serial{Congressional Research Service, Report No. A12345}
    \hi@book@pubparen
    \AssertMacro\hi@book@paren{%
        \hi@parens@add\hi@paren@date{%
            Cong. Rsch. Serv., Report No. A12345%
        }%
    }%

    \def\hi@kv@year{1998}
    \hi@book@pubparen
    \AssertMacro\hi@book@paren{%
        \hi@parens@add\hi@paren@date{%
            Cong. Rsch. Serv., Report No. A12345, \hi@pstruct@use{year}%
        }%
    }%


\endgroup


%</test>
