%%
%% \iffalse filename: refs.dtx \fi
%%
%<*doc>
\input driver
\thisis{refs}{Defining References}

A reference represents a single citable work of authority. What constitutes a
single work may be uncertain, particularly for statutes and multi-authored
works, as described previously. The general test for whether two items are two
references or subdivisions of a single reference is whether, if they were cited
immediately in sequence, \emph{id.}\ should be used for the second item. In some
cases this is a matter of the writer's taste, so this package generally gives
writers the flexibility to define references however they see fit.


If a reference is used in a citation string but not defined, this package will
issue an error, or it will keep track of the missing citation. See \sec{draw}
for more information on missing references.


\subsection{Data Model}

Each reference consists of:
\begin{itemize}
\item A reference type, such as a book, case, or statute. The complete list of
reference types is given in Part~\ref{p:alltypes}. The reference type determines
how the reference is formatted and what parameters are necessary to define the
reference.

\item An identifier nickname that will be used to identify this reference in
citations (and, on occasion, inside other references). The nickname must be
unique and generally should not contain spaces.

\item Parameters providing information about the reference. For
example, a book uses parameters for the author, title, and publication year.
The reference type determines which parameters are required and optional; this
is documented in Part~\ref{p:alltypes}.
There are also parameters of general applicability to all reference types,
described in \sec{genparams}. The complete list of parameters
is given in Part~\ref{p:allparams}.
\end{itemize}

Each reference type is further associated with a category, used to organize a
Table of Authorities as described in \sec{toa}.
The category can be altered using the reference
definition parameter \param{citetype}.
The categories are associated with a textual
description that is used to assemble the headings for the table of authorities,
as described in \sec{toa}.



%</doc>
%<*package>
%
% Unfortunately, when I first started writing this program I used the word
% ``cite'' to refer to what I now call references. While I have tried to be
% consistent in the documentation, I have not revised the internal macro names
% in the program to follow the current paradigm.
%
% \DescribeMacro\hi@citetype
% Defines a macro |\tc@|\meta{reference} to be the TOA category desired for
% \meta{reference}. This macro should be called within the reference definer
% macros to set the category, or it is invoked by the |citetype| parameter. If
% it is not called such that no |\tc@|\meta{reference} macro is set, then the
% reference will not appear in the table. \#1 should be the reference name, \#2
% a macro defined to the category name.
%    \begin{macrocode}
\def\hi@citetype#1#2{%
    \@ifundefined{tc@#1}{%
        \global\expandafter\let\csname tc@#1\endcsname#2%
    }{}%
}
\let\hi@newcitetype\def
%    \end{macrocode}
%
% Definitions of the categories follow in the source code, but they are used to
% produce a table in the documentation, so please consult the documentation for
% the code.
%
%</package>
%<*doc>

The following are the reference categories. The group associated with each
category is a label for the group in the table of authorities that the reference
will be placed together with (so, for example, statutes and regulations are
placed in the same category). The text associated with each category will be
used for the heading in the table of authorities.

\begin{demo}
\def\hi@newcitetype#1#2{\expandafter\hi@newcitetype@\string#1\@mark#2\@stop}
\expandafter\def\expandafter\hi@newcitetype@
    \string\hi@type@#1\@mark#2:#3\@stop{%
    #1&#2&#3\\%
}
\begin{tabular}{lll}
\textbf{Category} & \textbf{Group} & \textbf{Text} \\
\hline
%</doc>
%<*doc|package>
\hi@newcitetype\hi@type@other{other:Other Source/s}
\hi@newcitetype\hi@type@const{const:Constitutional Provision/s}
\hi@newcitetype\hi@type@found{const:Foundational Document/s}
\hi@newcitetype\hi@type@statute{statute:Statute/s}
\hi@newcitetype\hi@type@regulation{statute:Regulation/s}
\hi@newcitetype\hi@type@rule{statute:Rule/s}
\hi@newcitetype\hi@type@case{case:Case/s}
\hi@newcitetype\hi@type@admin{case:Administrative Decision/s}
\hi@newcitetype\hi@type@treaty{statute:Treat/y/ies}
%</doc|package>
%<*doc>
\end{tabular}
\end{demo}


%</doc>
%<*package>
%
% \DescribeMacro\hi@type@doforall
% This macro will perform |\do| for each of the Table of Authorities groups, in
% the proper order.
%    \begin{macrocode}
\def\hi@type@doforall{%
    \do{case}%
    \do{const}%
    \do{statute}%
    \do{other}%
}
%    \end{macrocode}
%
%</package>
%<*doc>


\subsection{Input Syntax}
\label{s:refs-syntax}

The package provides its own syntax for writers to use when defining references.
As an alternative, \sec{hibib} describes a compatibility layer that
accepts Biblatex reference definitions, so long as the parameter names
correspond with those of this package.

Generally, references are defined as follows:
\begin{demo}
|\def|\meta{reference type} \marg{name} |{| \\
|    |\meta{param1} |=| \meta{value1}|,| \\
|    |\meta{param2} |=| \meta{value2}|,| \\
|    |\ldots \\
|}|
\end{demo}
For example, to define the historic case on copyrightability:
\begin{demo}
|\defcase{baker}{|\\
|    parties=Baker v. Selden,|\\
|    cite=101 U.S. 99,|\\
|    year=1880,|\\
|}|
\end{demo}
The input follows \TeX\ syntax and uses the |keyval| package, so if there are
equal signs or commas in the parameter values, then the parameter should be
surrounded with braces.

Entering the parameter names for every reference can be tedious, so for common
reference types there are shortcuts described in \sec{cparse}.

%</doc>
%<*package>
%
% Contains a list of every reference type.
%    \begin{macrocode}
\let\hi@citeforms\@empty
%    \end{macrocode}
%
%
%
% Creates a new reference definer macro. \#1 is the |\def|\meta{type} command
% for the reference, \#2 a textual description, and \#3 the macro code to be run
% to define the reference. Executing the reference definition command should
% cause the definition of the following formatting macros of the form
% |\|\meta{prefix}|@|\meta{ref-name}, with prefixes given in \sec{draw-form}.
%
% For convenience of parameter definition macros, |\@this@case| is set to the
% name of the reference being defined.
%
%    \begin{macrocode}
\def\hi@newcite#1#2#3{%
    \g@addto@macro\hi@citeforms{\do{#1}{#2}}%
    \gdef#1##1##2{%
        \@ifundefined{fc@##1}{}{%
            \PackageError\hi@pkgname{%
                Citation for ##1 already defined
            }{%
                The citation for ##1 has already been defined.\MessageBreak
                Please select another name.
            }%
        }%
        \begingroup
            \def\@this@case{##1}%
            \hi@param@set{#1}{##2}%
            #3%
            % \hi@bib@output
        \endgroup
    }%
}
%    \end{macrocode}
%
% \DescribeMacro\hi@alias@cite
% Aliases a cite definer macro to another.
%
%    \begin{macrocode}
\def\hi@alias@cite#1#2{\def#1{#2}}
%    \end{macrocode}
%
% Within a reference definer macro, defines a new formatting macro of the form
% |\|\meta{prefix}|@|\meta{name}, where \#1 is \meta{prefix} and \#2 is
% \meta{name}. Mainly, this checks to make sure that a full citation macro isn't
% being redefined (indicating that the writer has used a duplicate name).
%    \begin{macrocode}
\def\hi@newcite@form#1#2{%
    \@ifundefined{#1@#2}{%
        \expandafter\protected@xdef\csname#1@#2\endcsname
    }{%
        \find@eq{fc}{#1}{%
            \PackageWarning\hi@pkgname{Refusing to redefine #1@#2}%
        }{}%
        \hi@newcite@form@gobble
    }%
}
\make@find@eq{fc}
\def\hi@newcite@form@gobble#1#{\@gobble}
%    \end{macrocode}
%
%
% \subsection{Implementation of Parameters}
%
% This implementation documentation now turns to how parameters are stored and
% set internally. For every defined parameter, there is a macro
% |\KV@hi@|\meta{param} as created by the |keyval| package. There is also a
% macro |\hi@kv@|\meta{param} defined to hold the value that the parameter is
% set to, which is used by most but not all parameters. By default, the
% |\hi@kv@|\meta{param} variable is given a value that will generate an error if
% used in a citation, alerting the user to the parameter being necessary but
% missing.
%
% This is a list of all the parameter registers. Each parameter is stored in the
% form |\do{|\meta{param}|}{|\meta{description}|}|.
%
%    \begin{macrocode}
\def\hi@params{}
%    \end{macrocode}
%
% Parenthetical parameters. We make a special token list of parenthetical
% creator commands, to be executed presumably upon the full citation.
%    \begin{macrocode}
\newtoks\hi@param@parens
%    \end{macrocode}
%
% Adds a parenthetical to the list. \#1 is the parenthetical priority, \#2 the
% text.
%    \begin{macrocode}
\def\hi@param@addparen#1#2{%
    \push@toks\hi@param@parens{%
        \hi@parens@add#1{#2}%
    }%
}
%
% Special token to set undefined parameters to.
%
%    \begin{macrocode}
\DeclareRobustCommand\hi@param@undef[1]{%
    \PackageError\hi@pkgname{%
        Citation parameter \noexpand#1 missing in \@this@case
    }{Include the missing citation parameter.}%
    \textbf{???}%
}
%    \end{macrocode}
%    \begin{macrocode}
\def\hi@undefine#1{\def#1{\hi@param@undef{\string#1}}}
%    \end{macrocode}
%
% Set all parameters to empty. This will be called after all the
% parameters have been created. Since reference definitions are executed inside
% groups, these initial settings should carry over into all definitions.
%
%    \begin{macrocode}
\def\hi@param@clear{%
    \def\do##1##2{\expandafter\hi@undefine\csname hi@kv@##1\endcsname}%
    \hi@params
    \hi@param@parens{}%
}
\AtEndOfPackage{\hi@param@clear}
%    \end{macrocode}
%
% Parameter definers. \#1 is the parameter, \#2 is an explanation, and anything
% that follows should be arguments to |\define@key| excluding the keyset and
% parameter name.
%
%    \begin{macrocode}
\def\hi@defparam#1#2{%
    \g@addto@macro\hi@params{\do{#1}{#2}}%
    \define@key{hi}{#1}%
}
\def\hi@newparam#1#2{%
    \@expand{\hi@defparam{#1}{#2}}{%
        \expandafter\def\csname hi@kv@#1\endcsname{##1}%
    }i%
}
\def\hi@defparam@noval#1#2{%
    \hi@defparam{#1}{#2}[#1]%
}
\def\hi@newparam@noval#1#2{%
    \@expand{\hi@defparam@noval{#1}{#2}}{%
        \expandafter\def\csname hi@kv@#1\endcsname{##1}%
    }i%
}
\def\hi@param@alias#1#2{%
    \g@addto@macro\hi@params{\do{#1}{Alias for #2}}%
    \edef\reserved@a{%
        \noexpand\def\expandafter\noexpand\csname KV@hi@#1\endcsname{%
            \expandafter\noexpand\csname KV@hi@#2\endcsname
        }%
    }\reserved@a
}
%    \end{macrocode}
%
% \paragraph{Setting Parameters}
% \DescribeMacro\hi@param@set
% The following macro is used when executing a reference definer, to read and
% set all the parameters in preparation for constructing the citation macros. It
% looks for a parseable formatted citation (either by finding a semicolon in the
% parameters, or determining that the parameters contain no equal sign), and
% otherwise uses |keyval| to read the parameters. It also executes any
% post-parameter hooks.
%
%    \begin{macrocode}
\def\hi@param@set#1#2{%
    \let\hi@param@set@hooks\@empty
    \expandafter\hi@param@set@kind\string#1\@stop
    \hi@param@read{#2}%
    \hi@param@set@hooks
}
\def\hi@param@read#1{%
    \find@in{;}{#1}{%
        \hi@param@read@{#1}%
    }{\hi@param@read@{#1}{#1}{}}%
}
\def\hi@param@read@#1#2#3{%
    \find@in{=}{#2}{%
        \@firstofthree{\setkeys{hi}{#1}}%
    }{%
        \chop@space@then@run{#2}\hi@cparse
        \ifstrempty{#3}{}{\setkeys{hi}{#3}}%
    }%
}
\make@find@in{=}
\make@find@in{;}
%    \end{macrocode}
%
% Sets the reference kind by removing |\def| from the beginning of the reference
% definer macro name.
%
%    \begin{macrocode}
\def\hi@param@set@kind#1#2#3#4#5\@stop{%
    \def\hi@kv@kind{#5}%
}
%    \end{macrocode}
%
% Adds a post-parameter-setting hook. \#1 is code to be run.
%
%    \begin{macrocode}
\def\hi@param@addhook#1{%
    \addto@macro\hi@param@set@hooks{#1}%
}
%    \end{macrocode}
%
% \paragraph{Testing Parameters in Citation Macros}
% Commands for testing parameters in citation macros.
%
%    \begin{macrocode}
\def\hi@ifset#1{%
    \expandafter\hi@ifset@#1.\@stop
}
\def\hi@ifset@#1#2\@stop{%
    \ifx\hi@param@undef#1\expandafter\@secondoftwo
    \else                \expandafter\@firstoftwo   \fi
}
%    \end{macrocode}
%    \begin{macrocode}
\def\hi@ifeitherset#1#2{%
    \hi@ifset#1{\@firstoftwo}{\hi@ifset#2}%
}
%    \end{macrocode}
%
%</package>
%<*doc>


\subsection{Citation Groups}

\label{s:refs-citegroups}

\DescribeMacro\defcitegroup
Often there will be several references that follow a standard conventional form,
so it would be convenient to define all of them in one go. The |\defcitegroup|
macro enables this to an extent. It takes four arguments:
\begin{itemize}
\item (optional) A prefix for the reference nicknames to be defined, default
blank
\item A reference definer macro (|\def|\meta{type})
\item Initial text for the reference definition body
\item A comma-separated list of texts used both as the nickname and as a suffix
to the reference definition
\end{itemize}

For example, say you want to define references for several sections of the
Patent Act. That could be achieved with the following three definitions:
\begin{demo}
|\defstatcode{101}{35 U.S.C. S 101}| \\
|\defstatcode{102}{35 U.S.C. S 102}| \\
|\defstatcode{103}{35 U.S.C. S 103}| \\
\end{demo}
This works (using the reference parser for statutes) but is repetitive. Instead,
you can write:
\begin{demo}
|\defcitegroup\defstatcode{35 U.S.C. S }{101, 102, 103}|
\end{demo}
Each of the items in the comma-separated list (e.g., |101|) becomes both a
reference nickname and the end of the reference definition.

With the optional argument, the reference names can be given a prefix:
\begin{demo}
|\defcitegroup[17usc]\defstatcode{17 U.S.C. S }{101, 102, 103}|
\end{demo}
would define |17usc101| as the statute ``17 U.S.C. \textsection~101'' and so on.


%</doc>
%<*package>
%    \begin{macrocode}
\newcommand\defcitegroup[4][]{%
    \forcsvlist{\hi@citegroup@{#1}{#2}{#3}}{#4}%
}
\def\hi@citegroup@#1#2#3#4{#2{#1#4}{#3#4}}
%    \end{macrocode}
%</package>
%<*test>

\section{refs.dtx}

\subsection{defcitegroup}

\begingroup

    \let\testregister\@empty
    \def\deftest#1#2{\appto\testregister{{#1}{#2}}}
    \defcitegroup\deftest{Prefix }{a, b, c}
    \AssertMacro\testregister{{a}{Prefix a}{b}{Prefix b}{c}{Prefix c}}

    \let\testregister\@empty
    \defcitegroup[pre-]\deftest{Prefix }{a, b, c}
    \AssertMacro\testregister{%
        {pre-a}{Prefix a}{pre-b}{Prefix b}{pre-c}{Prefix c}%
    }
\endgroup

%</test>
