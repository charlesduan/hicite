%%
%% \iffalse filename: caseref.dtx \fi
%%
%<*doc>
\input driver
\thisis{caseref}{Cases}


%</doc>
%<*package>
%    \begin{macrocode}
\def\hi@case@def#1{%
    \hi@case@setparties{#1}%
    \hi@case@updaterep
    \hi@case@history
    \hi@ifset\hi@kv@inline{%
        \@expand{\hi@short@register{#1}}\hi@kv@inline i{y}%
    }{}%
    \hi@case@proc
    \@expand{\hi@short@register{#1}}\hi@kv@inline i{n}%
    \hi@ifset\hi@kv@slip{\def\hi@kv@slip{slip op. }}{\def\hi@kv@slip{}}%
    \hi@newcite@form{fc}{#1}{%
        \hi@inline@only{\hi@citeguts{\hi@kv@name}}%
        \hi@inline@never{%
            \hi@citeguts{%
                \hi@name@only{\hi@kv@aname\hi@font@comma}%
                \hi@ifset\hi@kv@rep{%
                    \hi@kv@vol\space \hi@kv@rep\space
                    \@format@page@macro\hi@kv@page
                    \hi@maybepage{, }%
                }{%
                    \hi@kv@docket % Docket number
                    \hi@ifset{\hi@kv@dbid}{, \hi@kv@dbid}{}%
                    \hi@maybepage{, \hi@kv@slip\hi@page@atorsect}%
                }%
            }%
            \the\hi@param@parens
            \hi@case@dateparen{%
                \hi@ifeitherset\hi@kv@year\hi@kv@court{%
                    \hi@parens@add{\hi@paren@date}{%
                        \hi@param@optspc\hi@kv@court\hi@kv@year
                    }%
                }{}%
            }y%
        }%
    }%
    \hi@newcite@form{lc}{#1}{%
        \hi@citeguts{%
            \hi@name@only{\@capnext\hi@kv@lname\unkern,\hfill\break}%
            \hi@ifset\hi@kv@rep{%
                \hi@kv@vol\space\hi@kv@rep\space
                \@format@page@macro\hi@kv@page
                \hi@maybepage{, }%
            }{%
                \hi@kv@docket % Docket number
                \hi@ifset{\hi@kv@dbid}{, \hi@kv@dbid}{}%
                \hi@maybepage{, \hi@kv@slip\hi@page@atorsect}%
            }%
        }%
        \hi@case@dateparen{%
            \hi@parens@add{\hi@paren@date}{%
                \hi@param@optspc\hi@kv@court\hi@kv@year
            }%
        }y%
        \the\hi@param@parens
    }%
    \hi@newcite@form{sc}{#1}{%
        \hi@short@use{#1}%
        \hi@citeguts{%
            \hi@inline@only{\hi@citeguts{\hi@kv@inline}}%
            \hi@inline@never{%
                \hi@name@only{%
                    \hi@kv@ainline
                    \hi@font@comma
                }%
                \hi@ifset\hi@kv@rep{%
                    \hi@kv@vol\space\hi@kv@rep\space %
                    \hi@ifpage{\hi@maybepage{\hi@page@atorsect}}{%
                        \@format@page@macro\hi@kv@page
                    }%
                }{%
                    \hi@ifset{\hi@kv@dbid}{\hi@kv@dbid}{\hi@kv@docket}%
                    \hi@maybepage{, \hi@kv@slip\hi@page@atorsect}%
                }%
            }%
        }%
        \hi@case@dateparen{}n%
    }%
}
%    \end{macrocode}
%</package>
%<*doc>

\subsection{Parties and Case Titles}

\label{s:caseref-parties}

There are three ways a case can be named, which dictate the parameters that must
be given for a case reference definition:
\begin{itemize}
\item With adversarial parties, such as \emph{Brown v.~Board of Education}. The
parameters \param{p} and \param{d} set each party name, and the \param{parties}
parameter can be used to set both at once.

\item With a single name, such as \emph{In re Alappat}. The \param{name}
parameter accepts this type of case name.

\item With both adversarial parties and a common name. This occurs with older,
well-known Supreme Court cases, such as \emph{Consolidated Electric Light Co.\
v.\ McKeesport Light Co.}\ (The Incandescent Lamp Patent Case), and with modern
class-action litigation. All of the aforementioned parameters should be used for
these cases.
\end{itemize}

These input parameters are used to produce five formatted names for the case:
\begin{itemize}
\item The full title, used in full inline citations. This is abbreviated
according to the \abbscheme{inlcase} scheme.
\item The abbreviated full title, used in full non-inline citations. This is
abbreviated according to the \abbscheme{name} scheme.
\item The abbreviated full title to be used in the Table of Authorities. This
differs in spacing (an optional line break is placed before the versus) and in
text order (phrases like \emph{In re} are placed at the end rather than the
beginning). It is abbreviated according to the \abbscheme{inlcase} scheme.
\item The short form for inline citations, abbreviated with the
\abbscheme{inlcase} scheme.
\item The abbreviated short form for non-inline citations, using the
\abbscheme{name} scheme.
\end{itemize}

%</doc>
%<*package>
%
% Initially, macros |\case@|\meta{ref}|@|\meta{party} are set to match the case
% names given in parameters \param{p}, \param{d}, and \param{name}. These must
% be saved for comparison purposes against other cases (e.g., for subsequent
% history).
%
%    \begin{macrocode}
\def\hi@case@setparties#1{%
    \hi@ifset\hi@kv@p{%
        \global\cslet{case@#1@p}\hi@kv@p
    }{%
        \global\cslet{case@#1@p}\relax
    }%
    \hi@ifset\hi@kv@d{%
        \global\cslet{case@#1@d}\hi@kv@d
    }{%
        \global\cslet{case@#1@d}\relax
    }%
    \hi@ifset\hi@kv@name{%
        \global\cslet{case@#1@name}\hi@kv@name
    }{%
        \global\cslet{case@#1@name}\relax
    }%
}
%    \end{macrocode}
%
% This macro actually assembles the names.
%
%    \begin{macrocode}
\def\hi@case@proc{%
    \begingroup
        % For finding the best party for inline citations
        \let\hi@tmp@allnames\@empty
        % Scores will always be positive other than for impossible party names,
        % so setting this to -1 guarantees a party name will be selected
        \let\hi@tmp@bestscore\m@ne
        \hi@ifset\hi@kv@name{\@expand\hi@case@procname\hi@kv@name i}{}%
        \hi@ifset\hi@kv@p{\hi@case@procparties}{}%
        % Make the full-cite names, both inline and abbreviated
        \hi@case@proc@fullnames
        %
        % Set up the short-cite names
        \hi@case@proc@shortnames
        %
        % Now we need to get all the defined names out of the group. We do so by
        % creating |\reserved@a| to end the group and redefine all the name
        % macros.
        %
        \def\reserved@a{\endgroup}%
        \hi@ifset\hi@kv@name{\hi@redef@macro\reserved@a\hi@kv@name}{}%
        \hi@ifset\hi@kv@lname{\hi@redef@macro\reserved@a\hi@kv@lname}{}%
        \hi@ifset\hi@kv@aname{\hi@redef@macro\reserved@a\hi@kv@aname}{}%
        \hi@ifset\hi@kv@ainline{\hi@redef@macro\reserved@a\hi@kv@ainline}{}%
        \hi@ifset\hi@kv@inline{\hi@redef@macro\reserved@a\hi@kv@inline}{}%
    \reserved@a
}
\hi@undefine\hi@kv@lname
\def\hi@redef@macro#1#2{%
    \add@macro@to@macro#1{\expandafter\def\expandafter#2\expandafter{#2}}%
}
%    \end{macrocode}
%</package>
%<*doc>

\paragraph{Party Name Syntax}

Party names for \param{p} and \param{d} have the form:
\begin{grammar}
\meta{party} := \meta{text} [ \littext{ex rel.} \meta{text} ] \\
\Examples |Doe|\\|State ex rel. Adams|
\end{grammar}
The ``base'' name and the name after ``ex rel.''\ are treated as two separate
names, each a candidate for being the case's short name.

%</doc>
%<*package>
%
% Receives a party name and a baseline score for the party type. This macro (1)
% separates into a base name and \emph{ex rel.}\ part, (2) scores both parts as
% short names, (3) assembles the formatted name and abbreviates it, and (4)
% passes the resulting names to a callback. \#1 is the party name as text, \#2
% is the baseline score, \#3 the callback that should take two arguments for the
% unabbreviated and abbreviated name.
%
%    \begin{macrocode}
\def\hi@case@procparty#1#2#3{%
    \hi@abbrev@inlcase{#1}{\hi@case@procparty@}{#2}{#3}%
}
\def\hi@case@procparty@#1#2#3{%
    \find@in{ ex rel. }{#1}{\hi@case@procparty@parts}{%
        \hi@case@procparty@parts{#1}{}%
    }{#2}{#3}%
}
\make@find@in{ ex rel. }
%    \end{macrocode}
% \#1 is the base name, \#2 any ex-rel part, \#3 the baseline score, \#4 the
% callback.
%    \begin{macrocode}
\def\hi@case@procparty@parts#1#2#3#4{%
    \hi@case@cmpname{#1}{#3+10}%
    \ifblank{#2}{%
        % No ex-rel. Just run the callback
        \hi@abbrev@name{#1}{#4{#1}}%
    }{%
        % With an ex-rel. 
        \hi@case@cmpname{#2}{#3}%
        \hi@abbrev@name{#1\hi@case@exrel#2}{#4{#1\hi@case@exrel#2}}%
    }%
}
%
% Draw the word "ex rel.", with fonts.
\DeclareRobustCommand\hi@case@exrel{ \hi@fn@caseproc{ex rel.} }
%    \end{macrocode}
%
% If the case has party names, we must form the full titles and then pick the
% party for the short form.
%
%    \begin{macrocode}
\def\hi@case@procparties{%
    \hi@ifset\hi@kv@inlinedefendant{%
        \@expandarg\hi@case@procparty\hi@kv@p{-20000}{%
            \@twodef\hi@tmp@p\hi@tmp@pa
        }%
    }{%
        \@expandarg\hi@case@procparty\hi@kv@p{5}{\@twodef\hi@tmp@p\hi@tmp@pa}%
    }%
    \@expandarg\hi@case@procparty\hi@kv@d{3}{\@twodef\hi@tmp@d\hi@tmp@da}%
}
%    \end{macrocode}
%</package>
%<*doc>

For common case names with the \param{name} parameter, there are more
possibilities because prefixes \emph{In re} and \emph{Ex parte} can be used:
\begin{grammar}
\meta{name} := [ \littext{In re} \| \littext{Ex parte} ] \meta{text} [
\littext{ex rel.} \meta{text} ]\\
\Examples |In re Comiskey|\\|Ex parte New York ex rel. Jones|
\end{grammar}
The latter example is procedurally implausible but nevertheless syntactially
acceptable.

%</doc>
%<*package>
%
% Processes a common name provided for a case. Besides setting up the standard
% variables, this macro also splits off initial texts like procedural phrases
% and the word ``The.''
%
%    \begin{macrocode}
\def\hi@case@procname#1{%
    \let\hi@tmp@npre\@empty
    \@expand{\find@try\find@start{%
        {In re }{\def\hi@tmp@npre{\hi@fn@caseproc{In re} }\def\hi@kv@name}%
        {Ex parte }{%
            \def\hi@tmp@npre{\hi@fn@caseproc{Ex parte} }\def\hi@kv@name
        }%
    }}\hi@kv@name i{}%
    %
    % We use |\hi@inline@the| only if there is no other prefix. In such cases,
    % the word ``The'' seems more integrally part of the party name (In re The
    % Party), as opposed to when there is no other prefix and ``The'' is part of
    % a description of the litigation (The Slaughter-House Cases).
    %
    \ifx\hi@tmp@npre\@empty
        \hi@replacethe\hi@kv@name{\def\hi@kv@name}{}%
    \fi
    \@expandarg\hi@case@procparty\hi@kv@name{7}{\@twodef\hi@tmp@n\hi@tmp@na}%
    % The abbreviated full name always takes ``The'' if present
    \@expandarg\hi@replaceinlinethe\hi@tmp@na{The }{\def\hi@tmp@na}{}%
}
\make@find@start{In re }
\make@find@start{Ex parte }
%    \end{macrocode}
%
%
%</package>
%<*doc>


\paragraph{Full Name Formatting}

There are three full names required, for inline, non-inline, and Table of
Authorities citations. How each appears depends on which parameters are given:
\begin{itemize}
\item If party names are given alone, then the full names are ``\meta{p}
v.~\meta{d}'' appropriately abbreviated.
\item If party names and a common name are given, then the full names are
``\meta{p} v.~\meta{d} (\meta{name})'' appropriately abbreviated.
\item If only the name is given, this is the most complex case. If the name
begins with ``The,'' then the inline form uses |\hi@inline@the|, and the word
``The'' is placed at the end of the Table of Authorities form. If the name
begins with a prefix like \emph{In re}, then the prefix is placed at the end of
the Table of Authorities form. Otherwise, the name is assembled with ``The'' and
the prefix attached, appropriately abbreviated.
\end{itemize}
Additionally, the fonts need to be set up in assembling the full name forms.
This is because the name can use multiple fonts internally, for different parts.

%</doc>
%<*package>
%    \begin{macrocode}
\def\hi@case@proc@fullnames{%
    \hi@ifset\hi@kv@p{\hi@case@proc@fullnames@parties}{%
        \hi@case@proc@fullnames@nameonly
    }%
}
\def\hi@case@proc@fullnames@parties{%
    %
    % Create the parenthetical if a name was given. We have to do this up here
    % because we overwrite |\hi@kv@name| later.
    %
    \hi@ifset\hi@kv@name{%
        % Unabbreviated name will always use ``The''
        \@expandarg\hi@replaceinlinethe\hi@tmp@n{The }{\def\hi@tmp@n}{}%
        \edef\hi@kv@name{ (\hi@case@namefmt\hi@tmp@n)}%
        \edef\hi@kv@aname{ (\hi@case@namefmt\hi@tmp@na)}%
    }{%
        % If no name was given, then there's a possibility we need to do dot
        % management if |\hi@tmp@d| ends with a dot.
        \edef\hi@kv@name{\hi@case@testdot\hi@tmp@d}%
        \let\hi@kv@aname\@empty
    }%
    \epreto\hi@kv@name{%
        \noexpand\hi@fn@casefc{%
            \expandonce\hi@tmp@p \noexpand\hi@versus \expandonce\hi@tmp@d
        }%
        \hi@ifset\hi@kv@name{}{\hi@case@testdot\hi@tmp@d}%
    }%
    \epreto\hi@kv@aname{%
        \noexpand\hi@fn@casefc{%
            \expandonce\hi@tmp@pa \noexpand\hi@versus \expandonce\hi@tmp@da
        }%
    }%
    \let\hi@kv@lname\hi@kv@name
}
\def\hi@case@proc@fullnames@nameonly{%
    \hi@ifset\hi@kv@name{}{%
        \PackageError\hi@pkgname{%
            Case citation requires either parties or a common name%
        }{Please use the p, d, or name parameters}%
    }%
    % The TOA name should put ``The'' and any prefix at the end, so we have to
    % search for a leading |\hi@inline@the|.
    \@expandarg\hi@replaceinlinethe\hi@tmp@n{}{%
        \hi@case@proc@toaname{The}%
    }{\hi@case@proc@toaname{}{\expandonce\hi@tmp@n}}%
    % For the full inline name, ``The'' will always be used if there is a prefix
    % like In re.
    \ifx\hi@tmp@npre\@empty\else
        \@expandarg\hi@replaceinlinethe\hi@tmp@n{The }{\def\hi@tmp@n}{}%
    \fi
    \edef\hi@kv@name{\hi@case@namefmt\hi@tmp@n \hi@case@testdot\hi@tmp@n}%
    \edef\hi@kv@aname{\hi@case@namefmt\hi@tmp@na}%
}
%
% Sets |\hi@kv@lname| when only a name is given. Primarily, the prefixes need to
% be revised and placed at the end.
% \#1 is ``The'' or empty, \#2 the rest of the name to include in the TOA name.
\def\hi@case@proc@toaname#1#2{%
    \edef\hi@kv@lname{%
        \noexpand\hi@fn@casefc{#2}%
        \@expandarg\ifblank{\hi@tmp@npre#1}{}{%
            ,\hi@case@proc@toaname@npre
            \ifblank{#1}{}{ \noexpand\hi@fn@casefc{#1}}%
        }%
    }%
}
%
% If |\hi@tmp@npre| is not empty, use it but chop off any space at the end.
%
\def\hi@case@proc@toaname@npre{%
    \ifx\hi@tmp@npre\@empty\else
        \space
        \@expand{\find@end{ }}\hi@tmp@npre i{\unexpanded}{\meaning\hi@tmp@npre}%
    \fi
}
\make@find@end{ }
%    \end{macrocode}
%
% Emits the text to be added in standard uses of a |name| in a case. \#1 is the
% name macro (|\hi@tmp@n| or |\hi@tmp@na|). It is assumed that any leading
% ``the'' has already been taken care of before here.
%
% This macro is completely expandable and will produce text amenable to an
% |\edef|.
%
%    \begin{macrocode}
\def\hi@case@namefmt#1{%
    \expandonce\hi@tmp@npre \noexpand\hi@fn@casefc{\expandonce#1}%
}
%    \end{macrocode}
%
% If the text of macro \#1 ends with a dot, then appends |\@hi@dottrue| to the
% end of the text of \#2. This can be placed inside an |\edef|.
%
%    \begin{macrocode}
\def\hi@case@testdot#1{%
    \@expandarg\if@endswithdot{#1}{\unexpanded{\protect\@hi@dottrue}}%
}
%    \end{macrocode}
%
% \DescribeMacro\hi@versus
% Appearance of the ``v.''\ in party names.
%
%    \begin{macrocode}
\DeclareRobustCommand\hi@versus{%
    \if@hi@in@toa
        \ifnum\hi@citelevel=\@ne
            \hfill\penalty\m@ne\space\hskip\z@\@plus-1fill\relax
        \else
            ~%
        \fi
    \else
        ~%
    \fi
    v.\@\space
}
%    \end{macrocode}
%</package>
%<*doc>


\paragraph{Selecting the Short Name}

In theory, there are up to six possible candidates for the short name of a case
citation: the two parties, the common name, and the \emph{ex rel.}\ components
of each of those. To decide which of them is the best, this package employs a
scoring system applied to each candidate name. The following factors are
considered:
\begin{itemize}
\item Names that are impermissible party names receive the lowest scores; they
will be used only if there is no other option.
\item Names with capitalized first words or the word ``of'' are less likely to
be good short names so they receive lower scores.
\item \emph{Ex rel.}\ names are disfavored over base party names to the extent
that none of the above factors apply.
\item Common names are preferred over topside parties, which are preferred over
bottomside parties, to the extent that none of the above factors apply.
\end{itemize}

%</doc>
%<*package>
%
% Compares the name against the reigning best name, stored in
% |\hi@tmp@bestname|, and the best score in |\hi@tmp@bestscore|. If the provided
% name (\#1) has a better score, then it replaces the macros for best
% name. \#2 is a baseline score based on the nature of the name being compared
% (e.g., topside parties are preferable to bottomside parties).
%
%    \begin{macrocode}
\def\hi@case@cmpname#1#2{%
    \appto\hi@tmp@allnames{#1;}%
    \hi@case@scorename{#1}{\hi@case@cmpname@{#1}{#2}}%
}
\def\hi@case@cmpname@#1#2#3{%
    \ifnumcomp{#2+#3}>{\hi@tmp@bestscore}{%
        \def\hi@tmp@bestname{#1}%
        \def\hi@tmp@bestscore{#2+#3}%
    }{}%
}
%    \end{macrocode}
%
% Scores a name \#1, running \#2 as a callback.
%    \begin{macrocode}
\def\hi@case@scorename#1#2{%
    % Empty name
    \ifblank{#1}{#2{-10000}}{%
        % Name in the nocite list
        \ifcsdef{nocite@\detokenize{#1}}{#2{1000}}{%
            % Capitalized first word
            \hi@iffirstwordcaps{#1}{#2{2000}}{%
            % Name with the word ``of''
                \find@in{ of }{#1}{\@firstofthree{#2{3000}}}{%
                    #2{4000}%
                }%
            }%
        }%
    }%
}
\make@find@in{ of }
%    \end{macrocode}
%
% Test if the first word of the name is all caps.
%    \begin{macrocode}
\def\hi@iffirstwordcaps#1{%
    \find@in{ }{#1}{\hi@iffirstwordcaps@}{\hi@iffirstwordcaps@{#1}{}}%
}
\make@find@in{ }
\def\hi@iffirstwordcaps@#1#2{%
    \uppercase{\def\reserved@a{#1}}%
    \ifdefstring\reserved@a{#1}%
}
%    \end{macrocode}
%</package>
%<*doc>

\paragraph{Formatting the Short Name}
Short names may be user-provided or automatically determined based on the
selection algorithm above. To format the short name, several tasks must be
performed:
\begin{itemize}
\item For user-given short names, the word ``the'' at the beginning needs to be
processed for |\hi@inline@the|.
\item For automatically determined short names, some conventional words are
removed from the name as described below.
\item The short name must have appropriate fonts set.
\item If the short name is not obviously connected to the full name, then the
full name inline name requires a parenthetical, and the full citation name (a)
requires a parenthetical the first time it is ever used and (2) should actually
use the short name on subsequent uses, as described in \sec{state-ffc}.
\end{itemize}

%</doc>
%<*package>
%
%    \begin{macrocode}
\def\hi@case@proc@shortnames{%
    % If the user gave an inline name, deal with ``the'' at the beginning.
    % Otherwise, it is assumed that leading ``the'' was dealt with, so apply the
    % short name simplification algorithm.
    \hi@ifset\hi@kv@inline{\hi@replacethe\hi@kv@inline{\def\hi@kv@inline}}{%
        \@expandarg\hi@case@shortfmt\hi@tmp@bestname{\def\hi@kv@inline}%
    }%
    % Abbreviate the short name
    \@expandarg\hi@abbrev@name\hi@kv@inline{\def\hi@kv@ainline}%
    % Determine if the inline name chosen requires a parenthetical
    \edef\reserved@a{%
        \noexpand\in@{\expandafter\strip@prefix\meaning\hi@kv@inline}%
            {\expandafter\strip@prefix\meaning\hi@tmp@allnames}%
    }\reserved@a
    \hi@ifset\hi@kv@inlineparen{\in@false}{}%
    % Add formatting to the names. Leading ``the'' should have already been
    % converted to |\hi@inline@the|.
    \edef\hi@kv@inline{%
        \noexpand\hi@fn@casesc{\expandonce\hi@kv@inline}%
        \hi@case@testdot\hi@kv@inline
    }
    \edef\hi@kv@ainline{%
        \noexpand\hi@fn@casesc{\expandonce\hi@kv@ainline}%
    }
    % Add the parenthetical to the full names if needed. This must be done down
    % here, because the fonts for the short names need to be set first.
    \ifin@\else
        \eappto\hi@kv@name{%
            \noexpand\hi@short@maybeshow{\@this@case}{%
                \space(``\expandonce\hi@kv@inline'')%
                \unexpanded{\protect\@hi@dotfalse}%
            }%
        }%
        \edef\hi@kv@aname{%
            \noexpand\hi@case@choosefullorainline{%
                \expandonce\hi@kv@aname
            }{\expandonce\hi@kv@ainline}%
        }%
    \fi
}
%    \end{macrocode}
%
% Chooses the form of the full citation title for cases, where there is a short
% form. If it is the first full citation of the case ever, then it will be "Full
% Case Name (``Short Name'')." If it is a repeat full citation, then it will be
% "Short Name."
%
%    \begin{macrocode}
\DeclareRobustCommand\hi@case@choosefullorainline[2]{%
    \hi@record@firstfullcite\@this@case{%
        #1\hi@short@maybeshow{\@this@case}{%
            \space(``#2'')\protect\@hi@dotfalse
        }%
    }{%
        #2%
    }%
}
%    \end{macrocode}
%
%</package>
%<*doc>

Regarding the removal of conventional words, these are typically business
or geographical designations that are uninformative for the short name. A list
of ignored words follows:
%</doc>
%<*package>
%
% Short case name adjustments: generally we want to drop common phrases from the
% ends of short case names. \#1 is the name to abbreviate, \#2 a callback.
%
% The way this macro works is somewhat obscure. |\hi@case@shortfmt@words|
% expands to a |\find@try\find@end| list pairing truncatable end-words with
% |\hi@case@shortfmt|. The callback \#2 is placed at the end. If a match is
% found, then |\find@end| executes |\hi@case@shortfmt|, which picks up as its
% arguments the string prefix (passed by |\find@end|) and the callback (left on
% the stack). As a result, |\hi@case@shortfmt| is recursively called with the
% correct arguments. If no match is found, then |\hi@case@shortfmt@run| is
% executed, which again picks up the correct arguments.
%
%    \begin{macrocode}
\def\hi@case@shortfmt#1#2{%
    \@expand{\find@try\find@end}\hi@case@shortfmt@words i{#1}{%
        \hi@case@shortfmt@run{#1}%
    }{#2}%
}
\def\hi@case@shortfmt@run#1#2{#2{#1}}
\def\hi@case@shortfmt@words{}
\def\hi@case@shortfmt@add#1{%
    \addto@macro\hi@case@shortfmt@words{%
        {#1}{\hi@case@shortfmt}%
    }%
    \make@find@end{#1}%
}
%    \end{macrocode}
%
% Note that |\hi@case@shortfmt| is run on the unabbreviated short name. However,
% just in case the user enters a word abbreviated to force it to be abbreviated
% in the inline context, abbreviations are included below to ensure their
% removal.
%
%</package>
%<*doc>
\begin{demo}
\def\hi@case@shortfmt@add#1{\hb@xt@ 7em{\chop@space{#1}\hss} }%
\leavevmode
%</doc>
%<*doc|package>
\hi@case@shortfmt@add{,}
\hi@case@shortfmt@add{ Inc.}
\hi@case@shortfmt@add{ Co.}
\hi@case@shortfmt@add{ Corp.}
\hi@case@shortfmt@add{ LP}
\hi@case@shortfmt@add{ LLP}
\hi@case@shortfmt@add{ LLLP}
\hi@case@shortfmt@add{ P'ship}
\hi@case@shortfmt@add{ Partnership}
\hi@case@shortfmt@add{ LLC}
\hi@case@shortfmt@add{ Ltd.}
\hi@case@shortfmt@add{ America}
\hi@case@shortfmt@add{ Am.}
\hi@case@shortfmt@add{ USA}
\hi@case@shortfmt@add{ International}
\hi@case@shortfmt@add{ Int'l}
\hi@case@shortfmt@add{ of}
%</doc|package>
%<*doc>
\end{demo}


\subsection{Nominative Reporters}

Generally, reporter names are not abbreviated or changed from what is given as
input to the parameter \param{rep}. This is because reporters are conventionally
known by their abbreviations, and there is no consistency in reporter
abbreviations. However, some common reporters have both an official series name
and nominative reporter names. Volume 21 of the \emph{United States Reports},
for example, is also volume 8 of Wheaton's reports, and typically both are
cited:
\begin{demo}
|21 U.S. (8 Wheat.) 1|
\end{demo}
In most cases, the pagination of the series reporter and the nominative reporter
is the same, and adding the nominative reporter is a simple matter of looking up
the corresponding name in a table. As a result, this package handles these
nominative reports automatically. For nominative reporters with different
pagination, use the \param{parallel} parameter.


%</doc>
%<*package>
%    \begin{macrocode}
%
% Adds a parenthetical nominative reporter, based on a table.
%
\def\hi@case@updaterep{%
    \hi@ifset\hi@kv@vol{%
        \hi@ifset\hi@kv@rep{%
            \@ifundefined{hi@trep@\hi@kv@vol @\hi@kv@rep}{}{%
                \protected@edef\hi@kv@rep{%
                    \hi@kv@rep\space(%
                        \csname hi@trep@\hi@kv@vol @\hi@kv@rep\endcsname
                    )%
                }%
            }%
        }{}%
    }{}%
}
\def\hi@abbrev#1#2#3{%
    \@namedef{hi@trep@#1@#2}{#3}%
}
\input hireps
%    \end{macrocode}
%
%</package>
%<*doc>



\subsection{Date Parenthetical}

\label{s:caseref-date}

Ordinarily the date parenthetical for cases contains the \param{court} and
\param{year} parameters. A complication arises when the case is being cited as
part of a \rtype{casedoc} reference. In that case two things need to happen:
\begin{itemize}
\item If this case has a reporter listed (\param{rep} is set), then a
parenthetical for the docket number must be added.
\item If the case document has its own date set, then that date overrides the
date given for the case. \unskip\footnote{This does not conform to some legal
citation practices, which attach the date of the case decision rather than the
date of the document. That practice seems questionable since the date of
interest to the reader is more likely the case document's date. Nevertheless, it
is simple enough to assign the case decision's date to the case document when
defining the case document reference.}
\end{itemize}

%</doc>
%<*package>
%
% \#1 is the ordinary content for the date parenthetical. \#2 is y or n
% depending on whether the court should be included in \rtype{casedoc}
% citations.
%
%    \begin{macrocode}
\def\hi@case@dateparen#1#2{%
    \noexpand\ifx\noexpand\hi@casedoc@docket\relax
        #1%
    \noexpand\else
        \hi@ifset\hi@kv@rep{%
            \hi@parens@add{\hi@paren@casedocdocket}{%
                \hi@ifset\hi@kv@docket{\hi@kv@docket}{%
                    \noexpand\hi@casedoc@docket
                }%
            }%
        }{}%
        \noexpand\ifx\noexpand\hi@casedoc@date\relax\noexpand\else
            \hi@parens@add{\hi@paren@date}{%
                \ifx#2y%
                    \hi@ifset\hi@kv@court{\hi@kv@court\space}{}%
                \fi
                \noexpand\hi@casedoc@date
            }%
        \noexpand\fi
    \noexpand\fi
}
%    \end{macrocode}
%</package>
%<*doc>



\subsection{Procedural History}

\label{s:caseref-history}

Procedural history for a case is given in the \param{prior} and
\param{subsequent} parameters, which are used to construct parentheticals. The
syntax for those parameters is:
\begin{grammar}
\meta{history} := \meta{phrase} \littext{: } \meta{ref} \\
\Example |reversed: prior-case|
\end{grammar}
The \meta{phrase} is an explanatory phrase that will be abbreviated and also
checked against a table of valid explanatory phrases. If a desired phrase is not
already listed, the macro
\DescribeMacro\ExplanatoryPhrase
|\ExplanatoryPhrase|\marg{phrase[,]} is used. Note that the comma at the end of
the argument is significant, as some explanatory phrases use a comma and others
do not.

The main task to be accomplished in producing case history is reconciliation of
names. If the case being defined and the case in its history parenthetical have
the same-named parties, then the name can be omitted from the case in the
history parenthetical. If they have different names, then the phrase ``sub
nom.''\ must be added to explanatory phrases that end with a comma.

%</doc>
%<*package>
%
% \paragraph{Input for Case History}
%
% Parses case history and creates appropriate parentheticals.
%
%    \begin{macrocode}
\def\hi@case@history{%
    \hi@ifset\hi@kv@prior{%
        \@expandarg\hi@expl@parse\hi@kv@prior{}\hi@paren@prior
    }\relax
    \hi@ifset\hi@kv@subsequent{%
        \@expandarg\hi@expl@parse
            \hi@kv@subsequent
            { \hi@fn@latin{sub nom.}}%
            \hi@paren@subseq
    }\relax
}
%    \end{macrocode}
%
% History for things that are not cases.
%    \begin{macrocode}
\def\hi@cite@history{%
    \hi@ifset\hi@kv@prior{%
        \@expand\hi@expl@parse\hi@kv@prior{}\hi@paren@prior
    }\relax
    \hi@ifset\hi@kv@subsequent{%
        \@expand\hi@expl@parse\hi@kv@subsequent{}\hi@paren@subseq
    }\relax
}
%    \end{macrocode}
%
% Parses the explanatory phrase and citation reference syntax. This macro splits
% the phrase and the citation by colon, comma, or space, and passes to
% |\hi@expl@abbrev| to abbreviate the phrase and create the parenthetical. \#1
% is the text to be parsed. \#2 is ``sub nom.''\ if that is to be used. \#3 is
% the parenthetical class.
%
%    \begin{macrocode}
\def\hi@expl@parse#1#2#3{%
    \find@in{: }{#1}{\hi@expl@abbrev{}}{%
        \find@last{, }{#1}{\hi@expl@abbrev{,}}{%
            \find@last{ }{#1}{\hi@expl@abbrev{}}{%
                \hi@expl@err
                \hi@expl@abbrev{}{}{#1}% Set procedural phrase to nothing
            }%
        }%
    }{#2}{#3}% Parameters passed to |\hi@expl@abbrev| in all cases
}
\make@find@in{: }
\make@find@in{, }
\make@find@in{ }
%    \end{macrocode}
%
% Abbreviates the procedural phrase and passes control go |\hi@expl@addparen| to
% add the parenthetical. \#1 trailing matter for the phrase, \#2 the phrase, \#3
% the citation, \#4 the sub-nom to be used, and \#5 the parenthetical class.
%
%    \begin{macrocode}
\def\hi@expl@abbrev#1#2#3#4#5{%
    \hi@abbrev@expl{#2#1}{\hi@expl@addparen}{#3}{#4}{#5}%
}
%    \end{macrocode}
%
% Adds the parenthetical for a procedural history item. \#1 is the explanatory
% phrase, \#2 the reference name, \#3 the ``sub nom.'', and \#4 the
% parenthetical class. The parenthetical essentially just contains the above
% parameters and calls |\hi@expl| for formatting, except that no parenthetical
% is added in the Table of Authorities and to subsequent full cites.
%
%    \begin{macrocode}
\def\hi@expl@addparen#1#2#3#4{%
    \push@toks\hi@param@parens{%
        \if@hi@in@toa\else
            \hi@record@firstfullcite\@this@case{%
                \hi@parens@add#4{\hi@expl{#1}{#2}{#3}}%
            }{}%
        \fi
    }%
}
%    \end{macrocode}
%
% Error.
%
%    \begin{macrocode}
\def\hi@expl@err{%
    \PackageError\hi@pkgname{Explanatory phrase in history missing}{%
        You must include explanatory history phrase and citation.
    }%
}
%    \end{macrocode}
%
% \paragraph{Displaying Case History}
%
% Shows an procedural phrase parenthetical. \#1 is the phrase, \#2 the reference
% name, \#3 any sub-nom text. Two choices must be made: (1) whether to display
% the sub-nom text, and (2) whether to suppress the second case's name. The
% macro |\hi@expl@determinename| determines whether the two cases have the same
% name, and the outcomes are as follows:
% \begin{demo}
% \begin{tabular}{lll}
% \textbf{Condition} & \textbf{Sub nom} & \textbf{Use name} \\
% \hline
% Rehearing/certiorari denied phrase & No & No \\
% No sub nom text & No & If different \\
% No comma at end of phrase & No & If different \\
% Both comma and sub nom, same name & No & No \\
% Both comma and sub nom, different name & Yes & Yes \\
% \end{tabular}
% \end{demo}
%
%    \begin{macrocode}
\def\hi@expl#1#2#3{%
    \find@end{,}{#1}{\hi@expl@{,}}{\hi@expl@{}{#1}}{#2}{#3}%
}
\make@find@end{,}
% \#1 is the phrase's comma, \#2 the phrase with no comma, \#3 the reference
% name, \#4 any sub nom text.
\def\hi@expl@#1#2#3#4{%
    \ifcsdef{hi@expl@noname@#2}{%
        \hi@fn@sig{#2}#1\space
        \def\@last@inline{#3}%
    }{%
        \ifstrempty{#4}{% No sub-nom text
            \hi@fn@sig{#2}#1\space
            \hi@expl@determinename{#3}{\def\@last@inline{#3}}{}%
        }{%
            \ifstrempty{#1}{% No comma
                \hi@fn@sig{#2}#1\space
                \hi@expl@determinename{#3}{\def\@last@inline{#3}}{}%
            }{%
                \hi@expl@determinename{#3}{%
                    \hi@fn@sig{#2}#1\space
                    \def\@last@inline{#3}%
                }{%
                    \hi@fn@sig{#2}#4\space
                }%
            }%
        }%
    }%
    \clause{#3}%
}
\@namedef{hi@expl@noname@cert. denied}{}
\@namedef{hi@expl@noname@and cert. denied}{}
\@namedef{hi@expl@noname@reh'g denied}{}
\@namedef{hi@expl@noname@and reh'g denied}{}
%    \end{macrocode}
%
% Tests whether the given reference has the same case name as the current
% |\@this@case| such that no sub-nom should be used.
% \#1 is the reference name to test, \#2 the callback if the references are
% considered to have the same name, and \#3 the callback if the references have
% different names.
%
%    \begin{macrocode}
\def\hi@expl@determinename#1#2#3{%
    \edef\reserved@a{%
        \noexpand\hi@expl@determinename@
            \expandonce{\csname case@#1@p\endcsname}%
            \expandonce{\csname case@#1@d\endcsname}%
            \expandonce{\csname case@#1@name\endcsname}%
            \expandonce{\csname case@\@this@case @p\endcsname}%
            \expandonce{\csname case@\@this@case @d\endcsname}%
            \expandonce{\csname case@\@this@case @name\endcsname}%
    }\reserved@a
    \@test\if@hi@name\fi{#3}{\@hi@nametrue#2}%
}
%    \end{macrocode}
%
% Sets |\if@hi@name| to false if there is a name match, and true if no match. In
% other words, |\if@hi@name| indicates whether the reference name ought to be
% included.
%
%    \begin{macrocode}
\def\hi@expl@determinename@#1#2#3#4#5#6{%
    \@hi@nametrue
    \ifx#1\relax\else % Don't do this if it's not a case
        \ifx#1#4\ifx#2#5\@hi@namefalse\fi\fi % (i)
        \ifx#1#5\ifx#2#4\@hi@namefalse\fi\fi
        \hi@ifset#1{}{% (iii)
            \ifx#3#4\@hi@namefalse\fi
            \ifx#3#5\@hi@namefalse\fi
            \ifx#3#6\@hi@namefalse\fi
        }%
        \hi@ifset#4{}{%
            \ifx#6#1\@hi@namefalse\fi
            \ifx#6#2\@hi@namefalse\fi
            \ifx#6#3\@hi@namefalse\fi
        }%
    \fi
}
%    \end{macrocode}
%</package>
%<*doc>

\input docparams


%</doc>
\hi@newcite\defcase{A case citation}{%
    \hi@citetype{#1}\hi@type@case
    \hi@case@def{#1}%
}
%<*doc>
\keyparameters{%
    {p, d}{Parties named in the case.}
    {name}{A common or single name for the case. Required unless parties are
    given.}
    {vol, rep, page, cite}{Citation to the reporter containing the case.}
    {docket}{The docket number of the case. Required if \param{cite} parameters
    are not given, or may optionally be given if \rtype{casedoc} references will
    use this case.}
    {court}{The court deciding the case, which may be omitted if the reporter
    uniquely identifies the court.}
    {year}{The date of decision. For citing cases not yet decided, a qualifier
    word should be included to explain the procedural posture.}
    {inline}{A user-defined short name for this case.}
}
\optparameters{
    {inlinedefendant}{Don't use the first party as the short name.}
    {dbid}{A database identifier for unreported cases. This is optional since
    the docket number and court are sufficient to identify an unreported case.}
    {slip}{Whether this is a slip opinion; this adds a note to the page
    numbers.}
    {prior, subsequent}{Procedural history of the case, see
    \sec{caseref-history}.}
    {enbanc, mem, percuriam}{Parentheticals denoting the nature of the
    opinion.}
}
\begin{demo}
\Example
\begin{verbatim}
\defcase{goodyear-dunlop}{
    parties={Goodyear Dunlop Tires Operations, SA v. Brown},
    cite=564 U.S. 915,
    year=2011,
}
\end{verbatim}
\Produces \emph{Goodyear Dunlop Tires Operations, SA v.~Brown}, 564 U.S. 915
(2011).
\end{demo}
In this example, note the use of the \param{parties} and \param{cite}
parameters, as shortcuts to entering \param{p}, \param{d}, \param{vol},
\param{rep}, and \param{page}. Also observe that the party names are entered
unabbreviated, but reporters and court names are entered abbreviated (because
those are never used unabbreviated, and the abbreviations are too inconsistent).

\begin{demo}
\Example
\begin{verbatim}
\defcase{nimitz}{
    name=In re Nimitz Technologies LLC,
    court=Fed. Cir.,
    percuriam,
    date=dec 8 2022,
    docket=2023-103,
}
\end{verbatim}
\Produces \emph{In re Nimitz Techs.\ LLC, No.\ 2023-103 (Fed.\ Cir.\ Dec.\ 8,
2022)}.
\end{demo}
This demonstrates a case with a single name rather than adversarial parties, and
with a docket number rather than a reporter citation.

%</doc>
\hi@newcite\defadmincase{An administrative decision}{%
    \hi@citetype{#1}\hi@type@admin
    \hi@case@def{#1}%
}
%<*doc>
This is the same as a case with \param{citetype} set to |admin|. Parameters are
identical to \rtype{case} with the following addition:

\keyparameters{{agency}{Same as \param{court} but abbreviates the name.}}

\begin{demo}
\Example
\begin{verbatim}
\ExplanatoryPhrase{affirmed without opinion,}
\defadmincase{biodelivery}{
    parties={BioDelivery Sciences International, Inc. v. RB
        Pharmaceuticals Ltd.},
    docket=No. IPR2014-00325,
    court=P.T.A.B.,
    date=jun 30 2015,
    slip,
    paren=final written decision,
    subsequent=affirmed without opinion: biodelivery-cafc
}
\end{verbatim}
\Produces \emph{BioDelivery Scis.\ Int'l, Inc.\ v.\ RB
        Pharms.\ Ltd.}, No.\ IPR2014-00325 (P.T.A.B. June 30, 2015) (final
        written decision), \emph{aff'd without opinion}, \ldots
\end{demo}
This shows an administrative decision citation with subsequent history. The
reference |biodelivery-cafc| would be defined elsewhere.

%</doc>
\hi@newcite\defcasedoc{Document used in a case}{%
    \hi@citetype{#1}\hi@type@other
    \hi@ifset\hi@kv@citation{}{%
        \PackageError\hi@pkgname{%
            Missing citation for casedoc #1
        }{%
            A reference to a case is required for \string\defcasedoc
        }%
    }%
    \hi@newcite@form{fc}{#1}{%
        \hi@citeguts{%
            \noexpand\@capnext
            \hi@kv@name
            \hi@maybepage{ at }%
            ,
            \def\noexpand\hi@casedoc@docket{\hi@kv@docket}%
            \def\noexpand\hi@casedoc@date{\hi@kv@year}%
            \noexpand\hi@noid
            \let\noexpand\@last@inline\relax
            \noexpand\clause{\hi@kv@citation}\noexpand\hi@clause@endflag
        }%
        \hi@inline@never{%
            \hi@record@cite{\hi@kv@citation}%
            \hi@ifset\hi@kv@number{%
                \hi@parens@add\hi@paren@docno{Doc.\ No.\ \hi@kv@number}%
            }{}%
            \the\hi@param@parens
        }%
    }%
    \hi@newcite@form{sc}{#1}{%
        \hi@citeguts{%
            \hi@kv@name
            \hi@maybepage{ at }%
            ,
            \def\noexpand\hi@casedoc@docket{\hi@kv@docket}%
            \noexpand\hi@noid
            \noexpand\clause{\hi@kv@citation}\noexpand\hi@clause@endflag
        }%
    }%
}
%<*doc>

\keyparameters{
    {name}{The name of the document.}
    {citation}{The reference name of the case in which this document was
    filed.}
    {docket}{The docket number of the case, if it was not already given in the
    case.}
    {year}{The date of the document itself.}
    {number}{The document number or docket list entry. This is added as a
    parenthetical, preceded by ``Doc.~No.''}
}

\begin{demo}
\Example
\begin{verbatim}
\defcasedoc{wyze}{
    citation=case: {
        parties={Wyze Labs, Inc. v. Beijing Rockrobo
            Technology Co.},
        docket=2:21-cv-941,
        court=W.D. Wash.,
        date=dismissed jan 3 2022,
    },
    name=Complaint for Declaratory Judgment,
    date=july 15 2021,
}
\end{verbatim}
\Produces Complaint for Declaratory Judgment, \emph{Wyze Labs, Inc.\ v.\ Beijing
Rockrobo Tech.\ Co.}, No.\ 2:21-cv-941 (W.D. Wash.\ July 15, 2021).
\end{demo}
Here, the underlying case is defined using an anonymous reference definition
(see \sec{anonymous}).

%</doc>
%<*package>
%    \begin{macrocode}
\let\hi@casedoc@docket\relax
\let\hi@casedoc@date\relax
%    \end{macrocode}
%</package>
\hi@newcite\defcourtdoc{A document in the docket of current litigation}{
    \@expand\hi@abbrev@cdoc\hi@kv@name i{\def\hi@kv@name}%
    \hi@ifset\hi@kv@inline{%
        \@expand\hi@abbrev@cdoc\hi@kv@inline i{\def\hi@kv@inline}%
    }{%
        \hi@ifset\hi@kv@number{%
            \protected@edef\hi@kv@inline{Doc.\ No.\ \hi@kv@number}%
        }{%
            \let\hi@kv@inline\hi@kv@name
        }%
    }%
    \hi@newcite@form{fc}{#1}{%
        \hi@citeguts{%
            \hi@kv@name
            \hi@testfordot\hi@kv@name
            \hi@maybepage{\hi@page@space\hi@kv@name}%
            \hi@ifset\hi@kv@year{, \hi@kv@year\noexpand\@hi@dotfalse}{}%
        }%
        \hi@ifset\hi@kv@number{%
            \hi@parens@add\hi@paren@docno{Doc.\ No.\ \hi@kv@number}%
        }{}%
        \the\hi@param@parens
    }%
    \hi@newcite@form{sc}{#1}{%
        \hi@citeguts{%
            \hi@kv@inline
            \hi@testfordot\hi@kv@inline
            \hi@maybepage{\hi@page@space\hi@kv@inline}%
        }%
    }%
}
%<*doc>
It is intended for use in a legal brief or filing, to cite to other documents in
the docket of the same case in which the document is being filed. For example,
a brief in opposition to a motion may use this reference type to cite arguments
in the motion papers. To cite briefs or documents filed in another case, use
\rtype{casedoc}.

This package does not use parentheses around citations to court documents. This
is because it makes the citation form incompatible with other citations (e.g.,
if one cites a court document and a statute in one string cite). Furthermore,
most court documents filed today do not appear to use parentheses, and courts do
not appear to require them. If they are desired, commands such as |\sentence| or
|\clause| can be surrounded with parentheses.

In most appellate work and particularly before the Supreme Court, this reference
type is rarely used. The more common practice now is to name the document to be
cited in text and then include ``(at \meta{page})'' within the sentence:
\begin{demo}
Petitioner's brief argues (at 8) that the sky is red, but that is wrong.
\end{demo}
There is not really a need for automation of these kinds of citations.

\keyparameters{
    {name, inline}{Full and short names for the document. They will be
    abbreviated according to the \abbscheme{cdoc} scheme.}
    {number}{The document number of the document being cited, if one is present.
    The phrase ``Doc.~No.''\ will be prepended to it, and the number will be
    made a parenthetical to the citation. If \param{inline} is not given, then
    the document number will be used as the short name. (Otherwise the short
    form will be the same as the full name.)}
    {year}{The date the document was filed.}
}
%</doc>
%<*test>

\section{caseref.dtx}

\subsection{Short Name Utilities}

\AssertTrue{\hi@iffirstwordcaps{HELLO}}
\AssertFalse{\hi@iffirstwordcaps{Hello}}

% Short name processing
\AssertCallback{\hi@case@shortfmt{Test}}{Test}
\AssertCallback{\hi@case@shortfmt{Test, Inc.}}{Test}
\AssertCallback{\hi@case@shortfmt{Test USA, Inc.}}{Test}

% Scoring tests
\AssertCallback[\AssertNum]{\hi@case@scorename{Wheaton}}{>0}
\AssertCallback[\AssertNum]{\hi@case@scorename{}}{<0}
\AssertCallback[\AssertNum]{\hi@case@scorename{Wheaton}}{>0}
\AssertCallback[\AssertNum]{\hi@case@scorename{Bank of America}}{>0}
\AssertCallback[\AssertNum]{\hi@case@scorename{FTC}}{>0}


\subsection{Name Formatting}

\subsubsection{Named Parties}

\begingroup
    \hi@undefine\hi@kv@name
    \hi@undefine\hi@kv@inline
    \def\hi@kv@p{Wheaton}
    \def\hi@kv@d{Peters}
    \hi@case@proc
    \AssertMacro\hi@kv@name{\hi@fn@casefc{Wheaton\hi@versus Peters}}
    \AssertMacro\hi@kv@lname{\hi@fn@casefc{Wheaton\hi@versus Peters}}
    \AssertMacro\hi@kv@aname{\hi@fn@casefc{Wheaton\hi@versus Peters}}
    \AssertMacro\hi@kv@ainline{\hi@fn@casesc{Wheaton}}
    \AssertMacro\hi@kv@inline{\hi@fn@casesc{Wheaton}}
\endgroup

\subsubsection{Ex Rel.\ Party}

\begingroup
    \hi@undefine\hi@kv@name
    \hi@undefine\hi@kv@inline
    \def\hi@kv@p{Commonwealth ex rel. Wheaton}
    \def\hi@kv@d{Board of Trade}
    \hi@case@proc
    \AssertMacro\hi@kv@name{%
        \hi@fn@casefc{%
            Commonwealth\hi@case@exrel Wheaton\hi@versus Board of Trade%
        }%
    }%
    \AssertMacro\hi@kv@lname{%
        \hi@fn@casefc{%
            Commonwealth\hi@case@exrel Wheaton\hi@versus Board of Trade%
        }%
    }%
    \AssertMacro\hi@kv@aname{%
        \hi@fn@casefc{%
            Commonwealth\hi@case@exrel Wheaton\hi@versus Bd. of Trade%
        }%
    }%
    \AssertMacro\hi@kv@ainline{\hi@fn@casesc{Wheaton}}
    \AssertMacro\hi@kv@inline{\hi@fn@casesc{Wheaton}}
\endgroup

\subsubsection{Parties Only}

\begingroup
    \hi@undefine\hi@kv@name
    \hi@undefine\hi@kv@inline
    \def\hi@kv@p{Federal Trade Commission}
    \def\hi@kv@d{Actavis, Incorporated}
    \hi@case@proc
    \AssertMacro\hi@kv@name
        {\hi@fn@casefc{Federal Trade Commission\hi@versus Actavis,
        Inc.}\protect\@hi@dottrue}
    \AssertMacro\hi@kv@lname
        {\hi@fn@casefc{Federal Trade Commission\hi@versus Actavis,
        Inc.}\protect\@hi@dottrue}
    \AssertMacro\hi@kv@aname
        {\hi@fn@casefc{Fed. Trade Comm'n\hi@versus Actavis, Inc.}}
    \AssertMacro\hi@kv@ainline{\hi@fn@casesc{Fed. Trade Comm'n}}
    \AssertMacro\hi@kv@inline{\hi@fn@casesc{Federal Trade Commission}}
\endgroup

\subsubsection{Parties Only, Capitalized Name}

\begingroup
    \hi@undefine\hi@kv@name
    \hi@undefine\hi@kv@inline
    \def\hi@kv@p{FTC}
    \def\hi@kv@d{Actavis, Incorporated}
    \hi@case@proc
    \AssertMacro\hi@kv@name
        {\hi@fn@casefc{FTC\hi@versus Actavis, Inc.}\protect\@hi@dottrue}
    \AssertMacro\hi@kv@lname
        {\hi@fn@casefc{FTC\hi@versus Actavis, Inc.}\protect\@hi@dottrue}
    \AssertMacro\hi@kv@aname
        {\hi@fn@casefc{FTC\hi@versus Actavis, Inc.}}
    \AssertMacro\hi@kv@ainline{\hi@fn@casesc{Actavis}}
    \AssertMacro\hi@kv@inline{\hi@fn@casesc{Actavis}}
\endgroup

\subsubsection{Parties and User-Defined Inline}

\begingroup
    \hi@undefine\hi@kv@name
    \def\@this@case{actavis}
    \def\hi@kv@inline{Actavis II}
    \def\hi@kv@p{FTC}
    \def\hi@kv@d{Actavis, Incorporated}
    \hi@case@proc
    \AssertMacro\hi@kv@name{%
        \hi@fn@casefc{FTC\hi@versus Actavis, Inc.}\protect\@hi@dottrue
        \hi@short@maybeshow{actavis}{ % space
            (``\hi@fn@casesc{Actavis II}'')%
            \protect\@hi@dotfalse
        }%
    }%
    \AssertMacro\hi@kv@lname
        {\hi@fn@casefc{FTC\hi@versus Actavis, Inc.}\protect\@hi@dottrue}
    \AssertMacro\hi@kv@aname{%
        \hi@case@choosefullorainline{%
            \hi@fn@casefc{FTC\hi@versus Actavis, Inc.}%
        }{%
            \hi@fn@casesc{Actavis II}%
        }%
    }
    \AssertMacro\hi@kv@ainline{\hi@fn@casesc{Actavis II}}
    \AssertMacro\hi@kv@inline{\hi@fn@casesc{Actavis II}}
\endgroup

\subsubsection{Parties and Name}

\begingroup
    \def\hi@kv@name{The Reverse Settlement Litigation}
    \hi@undefine\hi@kv@inline
    \def\hi@kv@p{FTC}
    \def\hi@kv@d{Actavis, Incorporated}
    \hi@case@proc
    \AssertMacro\hi@kv@name{%
        \hi@fn@casefc{FTC\hi@versus Actavis, Inc.} % space
        (\hi@fn@casefc{The Reverse Settlement Litigation})%
    }%
    \AssertMacro\hi@kv@lname{%
        \hi@fn@casefc{FTC\hi@versus Actavis, Inc.} % space
        (\hi@fn@casefc{The Reverse Settlement Litigation})%
    }
    \AssertMacro\hi@kv@aname{%
        \hi@fn@casefc{FTC\hi@versus Actavis, Inc.} % space
        (\hi@fn@casefc{The Reverse Settlement Litig.})%
    }
    \AssertMacro\hi@kv@inline{%
        \hi@fn@casesc{\hi@inline@the Reverse Settlement Litigation}%
    }
    \AssertMacro\hi@kv@ainline{%
        \hi@fn@casesc{\hi@inline@the Reverse Settlement Litig.}%
    }
\endgroup

\subsubsection{Name Only, with Prefix}

\begingroup
    \def\hi@kv@name{In re Wands}
    \hi@undefine\hi@kv@inline
    \hi@undefine\hi@kv@p
    \hi@undefine\hi@kv@d
    \hi@case@proc
    \AssertMacro\hi@kv@name{%
        \hi@fn@caseproc{In re} \hi@fn@casefc{Wands}%
    }%
    \AssertMacro\hi@kv@lname{%
        \hi@fn@casefc{Wands}, \hi@fn@caseproc{In re}%
    }
    \AssertMacro\hi@kv@aname{%
        \hi@fn@caseproc{In re} \hi@fn@casefc{Wands}%
    }
    \AssertMacro\hi@kv@inline{%
        \hi@fn@casesc{Wands}%
    }
    \AssertMacro\hi@kv@ainline{%
        \hi@fn@casesc{Wands}%
    }
\endgroup

\subsubsection{Name Only, with Prefix and The}

\begingroup
    \def\hi@kv@name{In re The United States Ship}
    \hi@undefine\hi@kv@inline
    \hi@undefine\hi@kv@p
    \hi@undefine\hi@kv@d
    \hi@case@proc
    \AssertMacro\hi@kv@name{%
        \hi@fn@caseproc{In re} \hi@fn@casefc{The United States Ship}%
    }%
    \AssertMacro\hi@kv@lname{%
        \hi@fn@casefc{The United States Ship}, \hi@fn@caseproc{In re}%
    }
    \AssertMacro\hi@kv@aname{%
        \hi@fn@caseproc{In re} \hi@fn@casefc{The U.S. Ship}%
    }
    \AssertMacro\hi@kv@inline{%
        \hi@fn@casesc{The United States Ship}%
    }
    \AssertMacro\hi@kv@ainline{%
        \hi@fn@casesc{The U.S. Ship}%
    }
\endgroup

\subsubsection{Name Only, with The}

\begingroup
    \def\hi@kv@name{The Slaughter-House Cases}
    \hi@undefine\hi@kv@inline
    \hi@undefine\hi@kv@p
    \hi@undefine\hi@kv@d
    \hi@case@proc
    \AssertMacro\hi@kv@name{%
        \hi@fn@casefc{\hi@inline@the Slaughter-House Cases}%
    }%
    \AssertMacro\hi@kv@lname{%
        \hi@fn@casefc{Slaughter-House Cases}, \hi@fn@casefc{The}%
    }
    \AssertMacro\hi@kv@aname{%
        \hi@fn@casefc{The Slaughter-House Cases}%
    }
    \AssertMacro\hi@kv@inline{%
        \hi@fn@casesc{\hi@inline@the Slaughter-House Cases}%
    }
    \AssertMacro\hi@kv@ainline{%
        \hi@fn@casesc{\hi@inline@the Slaughter-House Cases}%
    }
\endgroup

\subsection{Nominative Reporters}

\begingroup
    \def\hi@kv@rep{U.S.}
    \def\hi@kv@vol{21}
    \hi@case@updaterep
    \AssertMacro\hi@kv@rep{U.S. (8 Wheat.)}
\endgroup


\subsection{Date Parenthetical}

\begingroup

    \def\hi@kv@rep{U.S.}
    \protected@edef\TestParen{%
        \hi@case@dateparen{\hi@parens@add\hi@paren@date{1900}}{y}%
    }
    \hi@parens@reset
    \TestParen
    \AssertMacro\hi@parens{\@paren{\hi@paren@date}{1900}}

    \def\hi@casedoc@docket{No. 12-345}
    \def\hi@casedoc@date{Jan. 5, 1899}
    \hi@parens@reset
    \TestParen
    \AssertMacro\hi@parens{%
        \@paren{\hi@paren@date}{\hi@casedoc@date}%
        \@paren{\hi@paren@casedocdocket}{\hi@casedoc@docket}%
    }%

    \hi@undefine\hi@kv@rep
    \protected@edef\TestParen{%
        \hi@case@dateparen{\hi@parens@add\hi@paren@date{1900}}{y}%
    }
    \hi@parens@reset
    \TestParen
    \AssertMacro\hi@parens{%
        \@paren{\hi@paren@date}{\hi@casedoc@date}%
    }%

\endgroup



\subsection{Reference Citation Tests}

\defcase{fsu}{
Board of Education ex rel. Board of Trustees v.
American Bioscience, Inc., 333 F.3d 1330 (Fed. Cir. 2003)
}
\AssertBox{\note{fsu}}{%
    \InFootnote{Bd.\ of Educ.\ \emph{ex rel.}\ Bd.\ of Trs.\ v.\ Am.\
    Bioscience, Inc., 333 F.3d 1330 (Fed.\ Cir.\ 2003).}%
}

\defcase{wheaton}{
Wheaton v. Peters, 33 U.S. 591 (1834)
}

\AssertBox{\inline{wheaton}}{\emph{Wheaton v.\ Peters}}

\AssertBox{\note{wheaton}}{%
    \InFootnote{Wheaton v.\ Peters, 33 U.S. (8 Pet.)\ 591 (1834).}%
}

\defcasedoc{wheaton-brief}{
name=Brief for Appellant,
date=Jan. 1 1833,
docket=No. 12-34,
citation=wheaton,
}
\AssertBox{\note{wheaton-brief at 5}}{%
    \InFootnote{Brief for Appellant at 5, \emph{Wheaton}, 33 U.S. (8
    Pet.)\ 591 (Jan.\ 1, 1833) (No.\ 12-34).}%
}

\subsection{Procedural Phrases}

\AssertCallback{\hi@abbrev@expl{certiorari denied}}{cert. denied,}
\AssertCallback{\hi@abbrev@expl{certiorari denied,}}{cert. denied,}
\AssertCallback{\hi@abbrev@expl{cert. denied}}{cert. denied,}
\AssertCallback{\hi@abbrev@expl{cert. denied,}}{cert. denied,}

\defcase{unocal-dct}{
parties=Oil Co. v. Atlantic,
cite=34 F. Supp. 2d 1208,
court=C.D. Cal.,
year=1998,
subsequent=affirmed: unocal-cafc,
}

\defcase{unocal-cafc}{
parties=Oil Co. v. Atlantic,
cite=208 F.3d 989,
court=Fed. Cir.,
year=2000,                                                    
}

\defcase{unocal-rf-dct}{
parties=Oil Co. v. Atlantic Richfield,
cite=34 F. Supp. 2d 1208,
court=C.D. Cal.,
year=1998,
subsequent=affirmed: unocal-cafc,
}

\defcase{unocal-rf-rd-dct}{
parties=Oil Co. v. Atlantic Richfield,
cite=34 F. Supp. 2d 1208,
court=C.D. Cal.,
year=1998,
subsequent=rehearing denied: unocal-cafc,
}

\AssertBox{\note{unocal-dct}}{%
    \InFootnote{Oil Co.\ v.\ Atlantic, 34 F. Supp.\
    2d 1208 (C.D. Cal.\ 1998), \emph{aff'd}{}, 208 F.3d 989 (Fed.\ Cir.\
    2000).}}

\fullcite{unocal-cafc}

\AssertBox{\tiny\note{unocal-rf-dct}}{%
    \tiny
    \InFootnote{Oil Co.\ v.\ Atl.\ Richfield, 34 F. Supp.\
    2d 1208 (C.D. Cal.\ 1998), \emph{aff'd} \emph{sub nom.}\ Oil Co.\ v.\
    Atlantic, 208 F.3d 989 (Fed.\ Cir.\ 2000).}}

\fullcite{unocal-cafc}

\AssertBox{\tiny\note{unocal-rf-rd-dct}}{%
    \tiny
    \InFootnote{Oil Co.\ v.\ Atl.\ Richfield, 34 F. Supp.\
    2d 1208 (C.D. Cal.\ 1998), \emph{reh'g denied}{}, 208 F.3d 989 (Fed.\ Cir.\
    2000).}}

\defcase{plessy}{
Plessy v. Ferguson, 163 U.S. 537 (1896);
subsequent=overruled by: brown,
}
\defcase{brown}{
p=Brown,
d=Board of Education of Topeka,
cite=347 U.S. 483,
year=1954,
}

\AssertBox{\tiny\note{plessy}}{%
    \tiny
    \InFootnote{%
        Plessy v.\ Ferguson, 163 U.S. 537 (1896), \emph{overruled by}
        Brown v.\ Bd.\ of Educ.\ of Topeka, 347 U.S. 483 (1954).%
    }%
}



%</test>
