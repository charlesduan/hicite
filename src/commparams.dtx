%%
%% \iffalse filename: commparams.dtx \fi
%%
%<*doc>
\input driver
\thisis{commparams}{Common Parameters}

These parameters are frequently used across many reference types. A general
description of each is given, but unlike the generally-applicable parameters in
\sec{genparams}, some reference types may use these parameters in special or
idiosyncratic ways, which should be described in the documentation for each
reference type.

\input docparams

%</doc>
\hi@newparam{authln}{Last name of the first author, for short citations}
%<*doc>
The \param{author} parameter will set this parameter automatically. This
parameter can be set after giving the authors to customize the last name.
%</doc>
%
\hi@defparam{author}{Author of the reference}{%
    \hi@nameproc@person{#1}{\hi@namelist@addtwo\hi@kv@author\hi@kv@authln}%
}
%<*doc>
The value should be a name as described in \sec{names}. If there are multiple
authors, provide this parameter multiple times, but see also parameter
\param{noetal} if entering three or more names. For institutional authors, see
parameter \param{instauth}.
%</doc>
%
\hi@defparam{cite}{Volume-reporter-page citation}{%
    \hi@cp@cite{#1}\KV@hi@vol\KV@hi@rep\KV@hi@page\KV@hi@year
}
%<*doc>
The value for this parameter should follow the syntax:
\begin{grammar}
\meta{vol} \meta{rep} \meta{page} [ \littext{(}\meta{year}\littext{)} ] \\
\Example |123 F.3d 456 (2009)|
\end{grammar}
and this parameter will set the parameters \param{vol}, \param{rep},
\param{page}, and \param{year} accordingly. See the discussion of
volume-reporter-page parsing in \sec{cparse} for more details.
%</doc>
%
\hi@param@alias{date}{year}
\hi@newparam{inline}{Short name for the reference}
\hi@defparam{instauth}{Institutional author}{%
    \hi@nameproc@inst{#1}{\hi@namelist@addinst}{\hi@kv@author}%
    \hi@nameproc@inst{#1}{\hi@namelist@addinst@ln}{\hi@kv@authln}%
}
%<*doc>
Adds an institutional author to the \param{author} parameter list.
%</doc>
%
\hi@param@alias{jcite}{cite}
\hi@param@alias{journal}{rep}
\hi@param@alias{journaltitle}{rep}
\hi@param@alias{booktitle}{rep}
\hi@newparam{name}{Name of the reference}
%<*doc>
Nearly every reference type accepts this parameter, which generally should be
the title of the reference. See the relevant reference type description for more
information.
%</doc>
%
\hi@newparam{number}{Number (for constitutional amendments, session laws)}
%<*doc>
The |number| parameter generally is used for a serial number associated with the
work, which may or may not contain text prefixes. Examples are patent numbers
(\rtype{patent}), bill numbers (\rtype{bill}), and constitutional amendments
(\rtype{constamend}). For most reference types (patents being an exception), the
number is not analyzed or formatted in any way.

For \rtype{book} and \rtype{workingpaper} references, consider also the
\param{serial} parameter to set the \param{publisher} parameter as well.
%</doc>
%
\hi@newparam{page}{Page number}
%<*doc>
See \sec{pages} for the usual formatting of this parameter's value.
%</doc>
\hi@param@alias{pages}{page}
\hi@newparam{rep}{Reporter or journal name}%
%<*doc>
Whether the value of this parameter needs to be abbreviated will depend on the
reference type.
%</doc>
%
\hi@param@alias{series}{number}
\hi@param@alias{short}{inline}
\hi@param@alias{shorttitle}{inline}
\hi@param@alias{src}{rep}
\hi@param@alias{subdiv}{page}
\hi@param@alias{title}{name}
\hi@newparam{type}{Type of work}
%<*doc>
For \rtype{jrnart} references, this would be ``Note'' or ``Comment,'' for
example. Other types of references use this parameter to replace default values
indicating the work type: \rtype{comments} for example will use this value
instead of the word ``Comments.'' The use of this parameter is highly specific
to the reference type.
%</doc>
%
\hi@defparam{vol}{Volume number}{%
    \find@in{:}{#1}{\@twodef\hi@kv@vol\hi@kv@issue}{\def\hi@kv@vol{#1}}
}
%<*package>
%    \begin{macrocode}
\make@find@in{:}
%    \end{macrocode}
%</package>
\hi@param@alias{volume}{vol}
\hi@defparam{year}{Year/date of the case, article, regulation}{%
    \hi@date@parse{#1}{\def\hi@kv@year}%
}
