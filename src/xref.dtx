%%
%% \iffalse filename: xref.dtx \fi
%%
%<*doc>
\input driver
\thisis{xref}{Cross-References}

\hia* offers comprehensive support for cross-references in citations. The model
is conceptually similar to \LaTeX, in which the writer uses commands to label
parts of the document and then references those labels. However, legal citation
requires and \hia* supports much more complex cross-referencing. It is possible
to label ranges of a document in addition to specific points, and the package
automatically determines the best wording to describe the labeled regions. More
importantly, use of cross-references is integrated into the citation system, so
cross-references can include signals, parentheticals, and other features of
citations.


\subsection{Types of Cross-References}

A cross-reference citation points to another part of the document, but the
writer must specify whether that other part is a page, footnote, section, or
other element of the document. Thus, the package must support different types of
cross-references. The following types are currently supported.

\begin{itemize}
\item Points in or ranges of body text. In legal memoranda, these are cited by
the page numbers where the range lies. For law review articles, they are cited
with reference to the footnote numbers that the range encompasses.

\item Footnotes, cited by the footnote number or range of numbers.

\item Sections, cited by the fully qualified section number (e.g., ``Section
II.A.3'').

\item Floats (tables or figures), cited by the float type and number (e.g.,
``Figure 3'').
\end{itemize}

\subsection{Labeling Document Parts}

The following commands are used to mark parts of a document with labels.

\DescribeMacro\hilabel
\cmd\hilabel\oarg{text}\marg{label} labels the current point in text. It detects
whether it is inside a footnote, table, figure, or body text, and adjusts the
label accordingly.

\DescribeMacro\hiendlabel
\cmd\hiendlabel\marg{label} closes a labeled range opened with |\hilabel|. Note
that the type of text marked at the beginning and end must be the same (e.g.,
|\hilabel| called in a footnote cannot be followed by |\hiendlabel| in body
text). Currently, only contiguous ranges can be labeled, although a desirable
additional feature for the package would be support for non-contiguous
cross-reference ranges.

\DescribeMacro\hitanlabel
\DescribeMacro\hiendtanlabel
\DescribeMacro\hiflabel
\DescribeMacro\hinotelabel
Particularly for law review articles, it is sometimes desirable to
cross-reference ``notes $x$--$y$ and accompanying text.'' This is done with
\cmd\hitanlabel\ and \cmd\hiendtanlabel. There are also provided commands
|\hiflabel| and |\hinotelabel| for labeling floats and footnotes, respectively.

%</doc>
%<*package>
%    \begin{macrocode}
\newcommand*\hilabel[2][]{%
    \@bsphack
    \@testcase
        \if@hi@note\fi{\hi@xref@writenote}%
        \ifx\@captype\@undefined\fi{\hi@xref@writetext}%
        \default{\hi@xref@writefloat}%
    {#2}{#1}%
    \@esphack
}
\newcommand*\hiendlabel[1]{%
    \hilabel{end:#1}%
}
\newcommand*\hitanlabel[2][]{%
    \@bsphack
    \hi@xref@writetan{#2}{#1}%
    \@esphack
}
\newcommand*\hiendtanlabel[1]{%
    \hitanlabel{end:#1}%
}
% Legacy code
\newcommand*\hiflabel[2][]{%
    \@bsphack
    \hi@xref@writefloat{#2}{#1}%
    \@esphack
}
\newcommand*\hinotelabel[2][]{%
    \@bsphack
    \hi@xref@writenote{#2}{#1}%
    \@esphack
}
%    \end{macrocode}
%
%</package>
%<*doc>

\DescribeMacro\hisectlabel
\cmd\hisectlabel\marg{label} labels the current section.
\DescribeMacro\hiendsectlabel
\cmd\hiendsectlabel\marg{label} labels an end of a section range.

For law review articles, top-level cross-references will use the word ``Part''
while other cross-references will use ``Section.'' (The distinction is
determined by whether the section number contains a dot.) For legal memoranda,
``Section'' will always be used. To change this, redefine the internal macro
|\hi@xref@sect@choose|.

%</doc>
%<*package>
% 
%    \begin{macrocode}
\newcommand*\hisectlabel[2][]{%
    \@bsphack
    \@expand{\hi@xref@sect@choose{#1}{#2}}{\@currentlabel}{i}%
    \@esphack
}
\def\hiendsectlabel#1{\hisectlabel{end:#1}}
%    \end{macrocode}
%
% Chooses the prefix word for section cross-references. This adds separate hooks
% for review-style and memorandum-style documents. \#1 is the optional
% pre-supra/infra text, \#2 the cross-reference name, and \#3 the expanded
% section number.
%
%    \begin{macrocode}
\appto\hi@hooks@review{%
    \def\hi@xref@sect@choose#1#2#3{%
        \find@in{.}{#3}{%
            \@firstofthree{\hi@xref@sect@add{Section}{Sections}}%
        }{%
            \hi@xref@sect@add{Part}{Parts}%
        }{#1}{#2}{#3}%
    }%
}
\appto\hi@hooks@memo{%
    \def\hi@xref@sect@choose#1#2#3{%
        \hi@xref@sect@add{Section}{Sections}{#1}{#2}{#3}%
    }%
}
\make@find@in{.}
%    \end{macrocode}
%
% Adds the section cross-reference to the aux file. \#1 is the singular section
% name, \#2 the plural, \#3 the pre-supra/infra text, \#4 the reference name,
% and \#5 the expanded section number.
%
%    \begin{macrocode}
\def\hi@xref@sect@add#1#2#3#4#5{%
    \hi@xref@writetoaux{#4}{#3}{#1 }{#2 }{#5}{}{}%
}
%    \end{macrocode}
%
% \subsection{Internal Cross-Reference Tracking}
%
% Internally, we use the following to store cross-reference labeling
% information. For a given reference \meta{name}, we define |\hir@|\meta{name}
% to be not |\relax|, which determines whether to use \emph{supra} or
% \emph{infra}. We also write to the aux file a |\newlabel| command, defining
% reference \meta{name} as a four-group value:
% \begin{itemize}
% \item Text before \emph{supra}/\emph{infra}
% \item Text before the reference number
% \item The reference number
% \item Text after the reference number
% \end{itemize}
% 
% For ranges, we also emit to the aux file a second |\newlabel| command,
% defining reference |end:|\meta{name} as a two-group value:
% \begin{itemize}
% \item Text before the reference number if a plural form is needed
% \item The reference number at the end of the range
% \end{itemize}
%
% This macro constructs and writes the relevant line to the aux file. \#1 is the
% name, \#2 the pre-supra/infra text, \#3 the pre-number text, \#4 the plural
% pre-number text, \#5 the number, \#6 the post-number text, and \#7 any
% pre-write command to execute on non-end cross-references (done within a
% group).
%
%    \begin{macrocode}
\def\hi@xref@writetoaux#1#2#3#4#5#6#7{%
    \begingroup
        \find@start{end:}{#1}{%
            \protected@write\@auxout{}{\string\newlabel{#1}{{#4}{#5}}}%
            \hi@xref@checktype{#3}{#6}%
        }{%
            \protected@csxdef{hir@#1}{{#3}{#6}}%
            #7%
            \protected@write\@auxout{}{\string\newlabel{#1}{{#2}{#3}{#5}{#6}}}%
        }%
    \endgroup
}
\make@find@start{end:}
%    \end{macrocode}
%
% Checks that the type of |end:|\meta{label} matches the type of \meta{label}.
% \#1 and \#2 are the text surrounding the note number to use for checking; \#3
% is \meta{label}.
%
%    \begin{macrocode}
\def\hi@xref@checktype#1#2#3{%
    \protected@edef\reserved@a{{#1}{#2}}%
    \expandafter\ifx\csname hir@#3\endcsname\reserved@a \else
        \hi@xref@checktype@err{#3}%
    \fi
}
\def\hi@xref@checktype@err#1{%
    \PackageError\hi@pkgname{%
        Cross-reference #1 was ended with a different type than it was
        opened with%
    }{%
        Make sure your \string\hilabel\space and \string\hiendlabel\space
        commands are placed in matching contexts (figures, footnotes, etc.)%
    }%
}
%    \end{macrocode}
%
% The following |\hi@xref@write|\meta{type} macros all write a particular data
% element to the aux file. \#1 is the reference name, \#2 the pre-supra/infra
% text.
%
%    \begin{macrocode}
\def\hi@xref@opt@noterefs{%
    \def\hi@xref@writetext##1##2{%
        \hi@xref@writetoaux{##1}{##2}{%
            text accompanying note
        }{%
            text accompanying notes
        }{\thefootnote}{}{\advance\c@footnote\@ne}%
    }
    \def\hi@xref@writetan##1##2{%
        \hi@xref@writetoaux{##1}{##2}{note }{notes }{\thefootnote}{ % space
            and accompanying text%
        }{\advance\c@footnote\@ne}%
    }
}
\def\hi@xref@opt@pagerefs{%
    \def\hi@xref@writetext##1##2{%
        \hi@xref@writetoaux{##1}{##2}{p. }{pp. }{\thepage}{}{}%
    }
    \def\hi@xref@writetan##1##2{%
        \hi@xref@writetoaux{##1}{##2}{p. }{pp. }{\thepage}{ and notes}{}%
    }
}
\appto\hi@hooks@review{\hi@xref@opt@noterefs}
\appto\hi@hooks@memo{\hi@xref@opt@pagerefs}
\def\hi@xref@writenote#1#2{%
    \hi@xref@writetoaux{#1}{#2}{note }{notes }{\thefootnote}{}{}%
}
\def\hi@xref@writefloat#1#2{%
    \hi@xref@writetoaux{#1}{#2}%
        {\csname \@captype name\endcsname\space}% Both Table and Figure can
        {\csname \@captype name\endcsname s }%    be pluralized with just `s'
        {\@currentlabel}{}{}%
}
%    \end{macrocode}
%
%</package>
%<*doc>


\subsection{Using Cross-References}

The ordinary citation commands are used to produce a cross-reference citation,
using the pseudo-reference \rtype{this} with the ``page number'' being the
cross-reference label. For example, assuming that |section-i| is defined as a
label:
\begin{quote}
\ttfamily
|\sentence|\{see this at section-i (providing background)\}.
\end{quote}
As this example suggests, cross-reference citation items can include
parentheticals and signals, just like ordinary citation items.

Cross-references to floats and sections can also be used inline, producing text
like ``Figure $X$.'' Again, the syntax mirrors inline citations for other
references:
\begin{quote}
\ttfamily
Data is shown in |\inline|\{this at section-i\}.
\end{quote}

%</doc>
%<*package>
%
% Defines the pseudo-reference macro for cross-citations. Note that there are no
% \emph{id.}\ citations for cross-references, so that form is set to be the same
% as the full citation.
%
%    \begin{macrocode}
\def\fc@this{%
    \ifx\@this@page\relax \hi@xref@nopageerr \else
        \if@hi@inline
            \expandafter\hi@xref@inline\@this@page
        \else
            \expandafter\hi@xref@cite\@this@page
        \fi
    \fi
}
\def\hi@xref@nopageerr{%
    \PackageError\hi@pkgname{No page number given for cross-reference}{%
        Give the name of the cross-reference label as the page number%
    }%
}
\def\hi@xref@inline#1#2#3{%
    \ifblank{#1}{#3}{#1}%
}
\def\hi@xref@cite#1#2#3{%
    \ifx\@this@opt\relax
        \hi@name@only{\ifblank{#1}{}{#1 }}%
    \else
        \@this@opt\space
    \fi
    #2 #3%
}
\let\idc@this\fc@this
%    \end{macrocode}
%
% |\@this@page| will be defined as a three-element macro: first, the text going
% before supra/infra, second, the word supra or infra, and third, the numbers
% going after.
%
%    \begin{macrocode}
\def\pc@this#1{%
    \@ifundefined{r@#1}{%
        \G@refundefinedtrue
        \PackageWarning\hi@pkgname{Reference `#1' undefined}%
        \def\@this@page{{}{}{\textbf{??#1}}}%
    }{%
        \@ifundefined{r@end:#1}{%
            \@expand{%
                \@unbrace{\hi@xref@noend{#1}}%
            }{\csname r@#1\endcsname}{ii}%
        }{%
            \@expand{%
                \@unbrace{%
                    \@expand{%
                        \@unbrace{\hi@xref@withend{#1}}%
                    }{\csname r@#1\endcsname}{ii}%
                }%
            }{\csname r@end:#1\endcsname}{ii}%
        }%
    }%
}
%    \end{macrocode}
%
% \#1 is the reference name, \#2--\#5 are the parameters of the label.
%    \begin{macrocode}
\def\hi@xref@noend#1#2#3#4#5{%
    \hi@xref@withend{#1}{#2}{#3}{#4}{#5}{#3}{#4}%
}
%    \end{macrocode}
%
% \#1 is the reference name, \#2--\#5 are the parameters of the label, \#6-\#7
% are the parameters of the end label.
%    \begin{macrocode}
\def\hi@xref@withend#1#2#3#4#5#6#7{%
    \def\@this@page{{#2}}%
    \@ifundefined{hir@#1}{%
        \appto\@this@page{{\hi@fn@sig{infra}}}%
    }{%
        \appto\@this@page{{\hi@fn@sig{supra}}}%
    }%
    \def\reserved@a{#4}\def\reserved@b{#7}%
    \ifx\reserved@a\reserved@b
        \appto\@this@page{{#3#4#5}}%
    \else
        \appto\@this@page{{#6#4--#7#5}}%
    \fi
}
%    \end{macrocode}
%
%</package>
%<*test>

\section{xref.dtx}

\let\WrittenText\relax
\let\OrigProtectedWrite\protected@write
\def\protected@write#1#2#3{%
    \begingroup
        \let\thepage\relax
        #2%
        \let\protect\@unexpandable@protect
        \xdef\WrittenText{#3}%
    \endgroup
}

\begingroup
    \@hi@notetrue
    \c@footnote=3
    \AssertBox{%
        \footnote{a \hilabel{lbl} b}%
    }{\@footnotemark[a b]}%
    \AssertMacro[\AssertExpandedEqual]\WrittenText{%
        \string\newlabel{lbl}{{}{note }{4}{}}%
    }
    \AssertMacro{\hir@lbl}{{note }{}}
\endgroup


\begingroup
    \@hi@notetrue
    \c@footnote=3
    \AssertBox{%
        \footnote{a \hilabel[discussion]{lbl} b}%
    }{\@footnotemark[a b]}%
    \AssertMacro[\AssertExpandedEqual]\WrittenText{%
        \string\newlabel{lbl}{{discussion}{note }{4}{}}%
    }
    \AssertMacro{\hir@lbl}{{note }{}}
\endgroup


\begingroup
    \@hi@notetrue
    \c@footnote=3
    \hilabel[discussion]{lbl}%
    \AssertBox{%
        \footnote{a \hiendlabel{lbl} b}%
    }{\@footnotemark[a b]}%
    \AssertMacro[\AssertExpandedEqual]\WrittenText{%
        \string\newlabel{end:lbl}{{notes }{4}}%
    }
    \AssertMacro{\hir@lbl}{{note }{}}
\endgroup

\subsection{In Text}

\begingroup
    \c@footnote=3
    \AssertBox{%
        a \hilabel{lbl} b%
    }{a b}%
    \AssertMacro[\AssertExpandedEqual]\WrittenText{%
        \string\newlabel{lbl}{{}{text accompanying note }{4}{}}%
    }
    \AssertMacro{\hir@lbl}{{text accompanying note }{}}
\endgroup

\begingroup
    \c@footnote=2 \hilabel{lbl}
    \c@footnote=3
    \AssertBox{%
        a \hiendlabel{lbl} b%
    }{a b}%
    \AssertMacro[\AssertExpandedEqual]\WrittenText{%
        \string\newlabel{end:lbl}{{text accompanying notes }{3}}%
    }
    \AssertMacro{\hir@lbl}{{text accompanying note }{}}
\endgroup


\begingroup
    \c@footnote=3
    \AssertBox{%
        a \hitanlabel{lbl} b%
    }{a b}%
    \AssertMacro[\AssertExpandedEqual]\WrittenText{%
        \string\newlabel{lbl}{{}{note }{4}{ and accompanying text}}%
    }
    \AssertMacro{\hir@lbl}{{note }{ and accompanying text}}
\endgroup

\begingroup
    \c@footnote=2 \hitanlabel{lbl}
    \c@footnote=3
    \AssertBox{%
        a \hiendtanlabel{lbl} b%
    }{a b}%
    \AssertMacro[\AssertExpandedEqual]\WrittenText{%
        \string\newlabel{end:lbl}{{notes }{3}}%
    }
    \AssertMacro{\hir@lbl}{{note }{ and accompanying text}}
\endgroup



\subsection{Floats}

\begingroup
    \def\@currentlabel{4}
    \def\@captype{figure}%
    \AssertBox{%
        a \hilabel{lbl} b%
    }{a b}%
    \AssertMacro[\AssertExpandedEqual]\WrittenText{%
        \string\newlabel{lbl}{{}{Figure }{4}{}}%
    }
    \AssertMacro{\hir@lbl}{{Figure }{}}
\endgroup

\begingroup
    \def\@captype{figure}%
    \def\@currentlabel{3}
    \hilabel{lbl}
    \def\@currentlabel{4}
    \AssertBox{%
        a \hiendlabel{lbl} b%
    }{a b}%
    \AssertMacro[\AssertExpandedEqual]\WrittenText{%
        \string\newlabel{end:lbl}{{Figures }{4}}%
    }
    \AssertMacro{\hir@lbl}{{Figure }{}}
\endgroup


\subsection{Sections}


\begingroup
    \def\@currentlabel{III.A}
    \AssertBox{%
        a \hisectlabel[background]{lbl} b%
    }{a b}%
    \AssertMacro[\AssertExpandedEqual]\WrittenText{%
        \string\newlabel{lbl}{{background}{Section }{III.A}{}}%
    }
    \AssertMacro{\hir@lbl}{{Section }{}}


    \def\@currentlabel{III}
    \AssertBox{%
        a \hisectlabel[background]{lbl} b%
    }{a b}%
    \AssertMacro[\AssertExpandedEqual]\WrittenText{%
        \string\newlabel{lbl}{{background}{Part }{III}{}}%
    }
    \AssertMacro{\hir@lbl}{{Part }{}}
\endgroup

\begingroup
    \def\@currentlabel{III.A}
    \hisectlabel{lbl}
    \def\@currentlabel{III.B}
    \AssertBox{%
        a \hiendsectlabel{lbl} b%
    }{a b}%
    \AssertMacro[\AssertExpandedEqual]\WrittenText{%
        \string\newlabel{end:lbl}{{Sections }{III.B}}%
    }
    \AssertMacro{\hir@lbl}{{Section }{}}
\endgroup

\let\protected@write\OrigProtectedWrite

\subsection{Parsing}

\global\let\hir@lbl\relax

\begingroup

    \def\r@lbl{{discussion}{note }{3}{ and accompanying text}}
    \pc@this{lbl}
    \AssertMacro\@this@page{%
        {discussion}{\hi@fn@sig{infra}}{note 3 and accompanying text}%
    }%

    \@namedef{r@end:lbl}{{notes }{3}}
    \pc@this{lbl}
    \AssertMacro\@this@page{%
        {discussion}{\hi@fn@sig{infra}}{note 3 and accompanying text}%
    }%

    \@namedef{r@end:lbl}{{notes }{5}}
    \pc@this{lbl}
    \AssertMacro\@this@page{%
        {discussion}{\hi@fn@sig{infra}}{notes 3--5 and accompanying text}%
    }%

    \def\hir@lbl{{note }{ and accompanying text}}%
    \pc@this{lbl}
    \AssertMacro\@this@page{%
        {discussion}{\hi@fn@sig{supra}}{notes 3--5 and accompanying text}%
    }%


\endgroup

\subsection{Formatting}

\begingroup

    \let\@this@opt\relax

    \def\@this@page{{}{\hi@fn@sig{infra}}{Section III}}
    \AssertBox{\fc@this}{\emph{infra} Section III}
    \def\@this@page{{discussion}{\hi@fn@sig{supra}}{notes 2--3}}
    \AssertBox{\fc@this}{discussion \emph{supra} notes 2--3}

\endgroup

\begingroup

    \def\@this@opt{remarks}
    \def\@this@page{{}{\hi@fn@sig{infra}}{Section III}}
    \AssertBox{\fc@this}{remarks \emph{infra} Section III}
    \def\@this@page{{discussion}{\hi@fn@sig{supra}}{notes 2--3}}
    \AssertBox{\fc@this}{remarks \emph{supra} notes 2--3}

\endgroup

\begingroup

    \let\@this@opt\relax
    \@hi@inlinetrue
    \def\@this@page{{}{\hi@fn@sig{infra}}{Section III}}
    \AssertBox{\fc@this}{Section III}
    \def\@this@page{{discussion}{\hi@fn@sig{supra}}{notes 2--3}}
    \AssertBox{\fc@this}{discussion}

\endgroup

%</test>
