%%
%% \iffalse filename: refparams.dtx \fi
%%
%<*doc>
\input driver
\thisis{refparams}{Other Parameters}

These parameters are used by other specific reference types.

\input docparams

%</doc>
\hi@param@alias{broadcaster}{instauth}
\hi@defparam{citation}{Citation to another reference}{%
    \hi@anonref{#1}{\def\hi@kv@citation}%
}
%<*doc>
The value may be a reference name or an anonymous reference definition (see
\sec{anonymous}).
Different reference types will use this parameter for different
purposes. For example, \rtype{jrnart} will cite the reference as the article
title (used for notes reviewing cases), \rtype{casedoc} will use it as the case
in which a document was filed, and \rtype{citecontainer} uses it as the
contained work (the container work is given in the parameter \param{in}).
%</doc>
%
\hi@newparam{commenter}{Names of parties submitting agency comments}
\hi@defparam{countries}{Countries to a treaty}{%
    \hi@abbrev@name{#1}{\def\hi@kv@countries}%
}
%<*doc>
The country names will be abbreviated.
%</doc>
%
\hi@newparam@noval{inlineparen}{Always include a parenthetical for inline form}
\hi@defparam{instto}{Institutional recipient of letter, memorandum, etc.}{%
    \hi@nameproc@inst{#1}{\hi@namelist@addinst}{\hi@kv@to}%
}
%<*doc>
Adds an institutional recipient to the \param{to} parameter list.
%</doc>
%
\hi@newparam{intdoc}{Document number for international documents}
%<*doc>
International resolutions and other documents may have two numbers: a serial
number and a document control number issued by the international body. This
parameter is for the second number. It should not include a prefix such as
``U.N. Doc.,'' as that will be added separately.
%</doc>
%
\hi@param@alias{issuedate}{issueyear}
\hi@defparam{issueyear}{Date a patent was issued}{%
    \hi@date@parse[issued]{#1}{\def\hi@kv@year}%
}
%<*doc>
This is equivalent to |date=issued |\meta{date}.
%</doc>
%
\hi@newparam@noval{noinlineparen}{Never include a parenthetical for inline form}
\hi@newparam@noval{noshorttitle}{Do not include title for short cites}
\hi@newparam{onfile}{Where the document is on file}
\hi@newparam{place}{Place (for speeches, constitutions, statutes)}
\hi@param@alias{producer}{instauth}
\hi@param@alias{pubdate}{pubyear}
\hi@defparam{pubyear}{Year of publication (for French statutes)}{%
    \hi@date@parse{#1}{\def\hi@kv@pubyear}%
}
%<*doc>
This is an additional date parameter used for French statutes.
%</doc>
%
\hi@param@alias{sponsor}{publisher}
\hi@param@alias{state}{place}
\hi@defparam{struct}{Structure for data of multivolume works}{%
    \global\@namedef{hi@struct@set@\@this@case}{#1}%
    \expandafter\def\expandafter\hi@kv@struct\expandafter{%
        \csname hi@struct@set@\@this@case\endcsname
    }%
}
%<*doc>
See the discussion of parameter structures in \sec{struct}.
%</doc>
%
\hi@defparam{to}{Recipient of a letter}{%
    \hi@nameproc@person{#1}{\hi@namelist@addtwo\hi@kv@to\relax}%
}
%<*doc>
This is primarily used for \rtype{letter} references. The value should be a name
as described in \sec{names}. If there are multiple recipients, provide this
parameter multiple times, but see also parameter \param{noetal} if entering
three or more names. For institutional authors, see parameter \param{instto}.
%</doc>
%
%<*package>
%
% Author lists.
%    \begin{macrocode}
\def\hi@kv@author@sortable{%
    \noexpand\@gobble{\hi@kv@authln}\@capnext\hi@kv@author
}
%    \end{macrocode}
%</package>
