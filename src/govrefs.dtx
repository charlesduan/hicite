%%
%% \iffalse filename: specrefs.dtx \fi
%
%<*doc>
\input driver
\thisis{govrefs}{Government Works}

These references are for legislative and executive materials that do not fall
into the above categories.

\input docparams

%</doc>
\hi@newcite\defbill{An bill or resolution of Congress}{%
    \hi@citetype{#1}\hi@type@other
    \hi@inlinestatname{#1}%
    \hi@ifset\hi@kv@congress{}{%
        \PackageError\hi@pkgname{%
            Definition of #1 is missing the Congress number%
        }{Provide the number in the definition}%
    }%
    \hi@ifset\hi@kv@status{%
        \@expand{\find@in{ by }}\hi@kv@status i{\hi@bill@abbrevstatus}{}%
    }{}%
    \hi@newcite@form{fc}{#1}{%
        \hi@inline@never{%
            \hi@citeguts{%
                \hi@ifset\hi@kv@name{\@capnext\hi@kv@name, }{}%
                \hi@kv@number,
                \expandafter\hi@numtotxt@numeric\expandafter{\hi@kv@congress}~%
                Cong.%
                \hi@maybepage{ }%
            }%
            \hi@parens@add{\hi@paren@date}{%
                \hi@ifset\hi@kv@status{as \hi@kv@status, }{}%
                \hi@kv@year
            }%
            \the\hi@param@parens
        }%
        \hi@inline@only{%
            \hi@citeguts{%
                \hi@ifset\hi@kv@name{\hi@inline@the\hi@kv@name}{\hi@kv@number}%
            }%
        }%
    }%
    \hi@newcite@form{sc}{#1}{%
        \hi@inline@never{%
            \hi@citeguts{%
                \hi@kv@number
                \hi@maybepage{, \hi@page@atorsect}%
            }%
        }%
        \hi@inline@only{%
            \hi@citeguts{%
                \hi@ifset\hi@kv@inline{\hi@kv@inline}{\hi@kv@number}%
            }%
        }%
    }%
}
%<*doc>

\keyparameters{
    {number}{The serial number of the bill. This is used as-is, and must include
    any prefix (e.g., ``S. 1234'').}
    {congress}{The term number of the Congress in which the bill was introduced.
    This should just be a number, and an ordinal suffix will be added
    automatically.}
    {year}{The date of the bill. By default, the date should be the date of
    introduction. Otherwise, use a date qualifier or the \param{status}
    parameter as described below.}
}
\optparameters{
    {name, inline}{\statuteparamdesc}
    {status}{The status of the version of the bill being cited, for example
    ``introduced,'' ``reported by \meta{committee},'' or ``passed by Senate.''
    This will be added before the date, so no qualifier should be added before
    the date if this parameter is used. Text after ``by'' will be abbreviated
    according to the \texttt{leg} abbreviation scheme (see \sec{abbrevs}).}
}
%</doc>
%<*package>
%    \begin{macrocode}
\def\hi@bill@abbrevstatus#1#2{%
    \def\hi@kv@status{#1 by }%
    \hi@abbrev@leg{#2}{\appto\hi@kv@status}%
}
%    \end{macrocode}
%</package>
%
\hi@newcite\defcongdoc{A Congressional numbered document}{%
    \hi@citetype{#1}\hi@type@other
    \hi@newcite@form{fc}{#1}{%
        \hi@ifset\hi@kv@author{%
            \noexpand\@gobble{\hi@kv@authln}%
        }{}% For sorting purposes
        \hi@citeguts{%
            \hi@ifset\hi@kv@author{%
                \noexpand\hi@fn@bookauthor{\hi@kv@author},
            }{}%
            \hi@ifset\hi@kv@name{%
                \noexpand\hi@fn@booktitle{\hi@kv@name},
            }{}%
            \noexpand\hi@fn@congdoc{\hi@kv@number}%
            \hi@ifset\hi@kv@page{%
                , \@format@page@macro\hi@kv@page
            }{}%
            \hi@maybepage{, \hi@page@atorsect}%
        }%
        \hi@parens@add{\hi@paren@date}{\hi@kv@year}%
        \the\hi@param@parens
    }%
    \protected@edef\reserved@a{%
        \noexpand\hi@fn@congdoc{\hi@kv@number}%
    }%
    \@expand{\hi@supra@form{#1}}\reserved@a i%
    \expandafter\hi@congdoc@expandnumber\expandafter{\hi@kv@number}%
    \hi@newcite@form{lc}{#1}{%
        \hi@ifset\hi@kv@author{%
            \noexpand\@gobble{\hi@kv@authln}%
        }{}% For sorting purposes
        \hi@citeguts{%
            \hi@ifset\hi@kv@author{%
                \noexpand\hi@fn@bookauthor{\hi@kv@author},
            }{}%
            \hi@ifset\hi@kv@name{%
                \noexpand\hi@fn@booktitle{\hi@kv@name},
            }{}%
            \noexpand\hi@fn@congdoc{\reserved@a}%
            \hi@ifset\hi@kv@page{%
                , \@format@page@macro\hi@kv@page
            }{}%
            \hi@maybepage{, \hi@page@atorsect}%
        }%
        \hi@parens@add{\hi@paren@date}{\hi@kv@year}%
        \the\hi@param@parens
    }%
}
%<*doc>
This is for reports, and documents. Unnumbered documents should use
\rtype{congprint}.

\keyparameters{
    {number}{The document number. This should include the abbreviated
    description for the number (e.g., ``H.R. Rep. No.''). The description must
    match a predefined list of permitted descriptions, and those descriptions
    will be de-abbreviated to make inline citation forms.}
    {year}{The date of the report.}
}
\optparameters{
    {author, name}{The author(s) and title of the document.}
    {page}{A pin cite that will be included in the full citation. This might be
    used, for example, for multipart reports (i.e., \texttt{page} could be
    \texttt{part~2}).}
}
%</doc>
%
%<*package>
%    \begin{macrocode}
\def\hi@congdoc@expandnumber#1{%
    \@expand{\find@try\find@start}\hi@congdoc@list i{#1}{%
        \PackageError\hi@pkgname{%
            #1 is not a valid congressional document number
        }{Use a valid name such as H.R. Rep. No. or S. Doc. No.}%
    }%
}
\def\hi@congdoc@do#1#2{%
    \def\reserved@a{#1#2}%
}
\def\hi@congdoc@list{}
\def\hi@congdoc@add#1#2{%
    \make@find@start{H.R. #1 No. }
    \make@find@start{S. #1 No. }
    \addto@macro\hi@congdoc@list{%
        {H.R. #1 No. }{\hi@congdoc@do{House #2 No. }}%
        {S. #1 No. }{\hi@congdoc@do{Senate #2 No. }}%
    }%
}
\hi@congdoc@add{Rep.}{Report}
\hi@congdoc@add{Conf. Rep.}{Conference Report}
\hi@congdoc@add{Doc.}{Document}
\hi@congdoc@add{Exec. Doc.}{Executive Document}
\hi@congdoc@add{Misc. Doc.}{Miscellaneous Document}
%    \end{macrocode}
%</package>
\hi@newcite\defcongprint{A Congressional unnumbered document}{%
    \hi@citetype{#1}\hi@type@other
    \hi@newcite@form{fc}{#1}{%
        \hi@ifset\hi@kv@author{%
            \noexpand\@gobble{\hi@kv@authln}%
        }{}% For sorting purposes
        \hi@citeguts{%
            \hi@ifset\hi@kv@author{%
                \noexpand\hi@fn@bookauthor{\hi@kv@author},
            }{%
                \hi@ifset\hi@kv@committee{%
                    \hi@noexpand\hi@fn@bookauthor{%
                        Staff of \hi@kv@committee,
                    }%
                }{}%
            }%
            \noexpand\hi@fn@bookauthor{%
                \expandafter\hi@numtotxt@numeric\expandafter{\hi@kv@congress}%
                ~Cong.,
            },
            \noexpand\hi@fn@booktitle{\hi@kv@name}%
            \hi@maybepage{\hi@page@space\hi@kv@name}%
        }%
        \hi@parens@add\hi@paren@date{Comm. Print \hi@kv@year}%
        \the\hi@param@parens
    }%
    \protected@edef\reserved@a{%
        \noexpand\hi@fn@booktitle{\hi@kv@name}%
    }%
    \@expand{\hi@supra@form{#1}}\reserved@a i%
}
%<*doc>
This is essentially a more structured citation to a book.

\keyparameters{
    {author}{Author(s) of the report or document.}
    {committee}{If no author is given but this parameter is, then the author
    becomes ``Staff of \meta{committee}.'' The committee name is abbreviated.}
    {name}{The title of the document.}
    {congress}{The term number of the Congress in which the bill was introduced.
    This should just be a number, and an ordinal suffix will be added
    automatically.}
    {year}{The year of publication.}
}
%</doc>
%
\hi@newcite\defcongrec{A Congressional Record debate}{%
    \expandafter\hi@congrec@testpg\hi@kv@page\@stop
    \hi@book@pubparen
    \hi@citetype{#1}\hi@type@other
    \hi@newcite@form{fc}{#1}{%
        \hi@citeguts{%
            \hi@usevol{\hi@ifset\hi@kv@vol{\hi@kv@vol\space}{}}%
            \noexpand\hi@fn@congdoc{\hi@kv@rep}\space
            \hi@ifpage{\hi@maybepage{}}{\@format@page@macro\hi@kv@page}%
        }%
        \hi@parens@add\hi@paren@date{%
            \hi@book@paren
            \hi@pstruct@use{year}%
        }%
        \the\hi@param@parens
    }%
    \hi@newcite@form{sc}{#1}{%
        \hi@citeguts{%
            \hi@kv@vol\space
            \hi@kv@rep\space at\space
            \hi@ifpage{\hi@maybepage{}}{\hi@kv@page}%
        }%
    }%
}
%<*doc>
This supports both the daily edition and the permanent edition, determined based
on the page number. Note also the \rtype{cr} reference below.

\keyparameters{
    {vol}{The volume number.}
    {rep}{The name of the record, typically ``Cong. Rec.''}
    {page}{The page number where the relevant debate begins. For the daily
    edition, include the letter prefix in the page number (preceded with ``!''
    to ensure it is not treated as a division name, see \sec{pages-forms}). The
    phrase ``daily ed.''\ will automatically be added to the date parenthetical
    in this case.}
    {year}{The date of the debate or of publication of the volume. It typically
    is a full date for the daily edition, and a year for the permanent edition.}
}
\bookparenparameters\ This is primarily useful for volume 1 of the \emph{Annals
of Congress}.

%</doc>
%
%<*package>
%
% If the page number in a Congressional Record citation 
%    \begin{macrocode}
\def\hi@congrec@testpg#1#2\@stop{%
    \@ifdigit{#1}{}{%
        \def\hi@kv@edition{daily ed.}%
    }%
}
%</package>
%<*doc>

\subsection{\texttt{cr}: The Congressional Record}

\label{rtype:cr}

As an alternative to \rtype{congrec} above, the predefined reference |cr| can be
used to cite arbitrary pages in the \emph{Congressional Record} permanent
edition. The date should be given as the optional argument:
\begin{quote}
|\sentence{see 123 cr[1977] 17147}.|
Produces: \emph{See} 123 \textsc{Cong.\ Rec.}\ 17147 (1977).
\end{quote}

%</doc>
%<*package>
%    \begin{macrocode}
\def\lc@cr{Congressional Record}
\def\tcpg@cr{} % Note: not sure if the volumes will show up
\def\fc@cr{%
    \hi@citeguts{%
        \@this@vol\space
        \hi@fn@congdoc{Cong. Rec.}\space
        \@this@page
    }%
    \ifx\@this@opt\relax\else
        \hi@parens@add\hi@paren@date{\@expandarg\hi@date@parse\@this@opt}%
    \fi
}
\hi@citetype{cr}\hi@type@other
%    \end{macrocode}
%</package>
\hi@newcite\defhearing{A hearing in Congress}{%
    \hi@citetype{#1}\hi@type@other
    \hi@ifset\hi@kv@type{}{\def\hi@kv@type{Hearing}}%
    \hi@newcite@form{fc}{#1}{%
        \hi@citeguts{%
            \noexpand\hi@usevol{}%
            \noexpand\hi@fn@hearing{%
                \hi@ifset\hi@kv@name{\hi@kv@name: }{}%
                \hi@kv@type\space
                \hi@ifset\hi@kv@number{on \hi@kv@number\space}{}%
                Before the \hi@kv@committee
            }\hi@font@comma
            \expandafter\hi@numtotxt@numeric\expandafter{\hi@kv@congress}~Cong.%
            \noexpand\@hi@dottrue
            \hi@maybepage{ }%
        }%
        \hi@parens@add{\hi@paren@date}{\hi@kv@year}%
        \the\hi@param@parens
    }%
    \protected@edef\reserved@a{%
        \noexpand\hi@fn@hearing{%
            \hi@ifset\hi@kv@name{%
                \hi@kv@name
            }{%
                Hearing\space
                \hi@ifset\hi@kv@bill{%
                    on \hi@kv@bill
                }{%
                    Before the \hi@kv@committee
                }%
            }%
        }%
    }%
    \@expand{\hi@supra@form{#1}}\reserved@a i%
}
%<*doc>
These should be hearings where the Goverment Printing Office has printed a
transcript of the proceedings. Use \rtype{testimony} to cite separately
published testimony in a hearing.

\keyparameters{
    {committee}{The committee before which the hearing was held.}
    {congress}{The term number of the Congress in which the bill was introduced.
    This should just be a number, and an ordinal suffix will be added
    automatically.}
}
\optparameters{
    {type}{The nature of the proceeding, default ``Hearing.''}
    {name}{A title to the hearing, if given.}
    {number}{A bill number (same as the \rtype{bill} reference type), to be
    given if the hearing deals with a specific bill.}
}
%</doc>
%
%<*package>
%    \begin{macrocode}
\make@find@in{, }
%    \end{macrocode}
%</package>
\hi@newcite\defprespaper{A Presidential Paper}{%
    \hi@citetype{#1}\hi@type@other
    \hi@newcite@form{fc}{#1}{%
        \hi@citeguts{%
            \hi@kv@name, % Title
            \hi@kv@vol\space % Volume
            \hi@kv@rep\space % Title
            \hi@kv@page % Page number
            \hi@maybepage{, }%
        }%
        \hi@parens@add{\hi@paren@date}{\hi@kv@year}%
        \the\hi@param@parens
    }%
}
%<*doc>

\keyparameters{
    {name}{The title of the paper.}
    {vol, rep, page, cite}{Citation information for the paper. Typically
    \param{rep} is ``Pub. Papers'' or ``Weekly Comp.~Pres.~Doc.''}
    {year}{The date of the document.}
}
%</doc>
%
\hi@newcite\defdcpd{Daily Compilation of Presidential Documents}{
    \hi@citetype{#1}\hi@type@other
    \hi@newcite@form{fc}{#1}{%
        \hi@citeguts{%
            \hi@kv@name,
            \hi@fn@jrntitle{Daily Comp. Pres. Doc.} No. \hi@kv@number
            \hi@maybepage{, at }%
        }%
        \hi@parens@add{\hi@paren@date}{\hi@kv@year}%
        \the\hi@param@parens
    }%
    \hi@newcite@form{sc}{#1}{%
        \hi@citeguts{%
            \hi@kv@name,
            \hi@fn@jrntitle{Daily Comp. Pres. Doc.} No. \hi@kv@number
            \hi@maybepage{, at }%
        }%
    }%
}
%<*doc>
Because documents in the DCPD are numbered rather than consecutively paginated,
a different reference type is required.

\keyparameters{
    {name}{The name of the document.}
    {number}{The number of the document.}
    {year}{The date of the document.}
}
%</doc>
%
\hi@newcite\defpatent{A U.S. patent}{%
    \hi@citetype{#1}\hi@type@other
    \let\reserved@a\@empty
    \@expand{\find@start{D}}\hi@kv@number i{%
        \def\reserved@a{D}\def\hi@kv@number
    }{}%
    \ifnum\hi@kv@number<\@m
        \let\hi@kv@inline\hi@kv@number
    \else
        \@tempcnta\hi@kv@number\relax
        \divide\@tempcnta\@m \multiply\@tempcnta-\@m
        \advance\@tempcnta\hi@kv@number\relax
        \def\hi@kv@inline{'\the\@tempcnta}%
    \fi
    \edef\hi@kv@number{%
        \expandafter\hi@numtotxt@comma\expandafter{\hi@kv@number}%
    }%
    %
    % So long as the date contains some prefix, I don't really care if it's
    % ``issued'' or ``filed''.
    %
    \@expand{%
        \find@start@cs{/hi@fn@dateprefix}%
    }{\hi@kv@year}{i}{\@gobble}{%
        \PackageWarning\hi@pkgname{%
            For patent citation #1, the date must include a\MessageBreak
            qualifier such as `filed' or `issued'.\MessageBreak
            This occurred%
        }%
    }%
    \hi@newcite@form{fc}{#1}{%
        \hi@citeguts{%
            \hi@ifset\hi@kv@name{\@capnext\hi@kv@name, }\@empty
            U.S. Patent No. \reserved@a\hi@kv@number
            \hi@maybepage{ }%
        }%
        \hi@inline@never{%
            \hi@parens@add{\hi@paren@date}{\hi@kv@year}%
            \the\hi@param@parens
        }%
    }%
    \hi@newcite@form{sc}{#1}{%
        \hi@citeguts{%
            \noexpand\hi@inline@the
            \hi@kv@inline\space Patent%
            \hi@maybepage{ }%
        }%
    }%
}
%<*doc>

\keyparameters{
    {number}{The patent number. It should be entered as an unpunctuated raw
    number, preceded by ``D'' for design patents. Commas will be added to the
    number, and the last three digits will be used for the short form.}
    {year}{The date of the patent. The date \emph{must} contain a qualifier,
    likely ``issued'' or ``filed.''}
    {issueyear}{Equivalent to setting \param{year} to
    \textsc{issued~}\meta{issueyear}.}
}
\optparameters{
    {name}{The title of the patent, if relevant.}
}
%</doc>
%
%<*package>
%    \begin{macrocode}
\make@find@start{D}
\make@find@start@cs{/hi@fn@dateprefix}{\hi@fn@dateprefix}
%    \end{macrocode}
%</package>
\hi@newcite\defgovdoc{A government document}{%
    \hi@ifset\hi@kv@court{}{%
        \hi@ifset\hi@kv@author{\let\hi@kv@court\hi@kv@author}{}%
    }%
    \hi@cite@history
    \hi@citetype{#1}\hi@type@other
    \hi@newcite@form{fc}{#1}{%
        \hi@citeguts{%
            \hi@ifset\hi@kv@number{%
                \@capnext\hi@kv@number,
            }{}%
            \hi@ifset\hi@kv@name{%
                \@capnext\hi@kv@name
            }{}%
            \hi@ifset\hi@kv@rep{%
                ,
                \hi@ifset\hi@kv@vol{\hi@kv@vol\space}{}%
                \hi@kv@rep\space \hi@kv@page
                \hi@maybepage{, }%
            }{%
                \hi@ifset\hi@kv@docket{%
                    , \hi@kv@docket
                    \hi@maybepage{, }%
                }{%
                    \hi@maybepage{ }%
                }%
            }%
        }%
        \hi@case@dateparen{%
            \hi@parens@add\hi@paren@date{%
                \hi@param@optspc\hi@kv@court\hi@kv@year
            }%
        }y%
        \the\hi@param@parens
    }
    % TODO think about the general case for the short form
    \hi@ifset\hi@kv@rep{%
        \hi@newcite@form{sc}{#1}{%
            \hi@citeguts{%
                \hi@ifset\hi@kv@vol{\hi@kv@vol\space}{}%
                \hi@kv@rep
                \hi@pageordefault{ at }{ \hi@kv@page}%
            }%
        }%
    }{%
        \@expand{\hi@supra@form{#1}}\hi@kv@name i%
    }%
}
%<*doc>
This is a generic reference type amenable primarily to two types of government
documents. First, it can be used for \emph{Federal Register} notices and other
government documents published in agency reporters or compilations. In this
case, the abbreviated reporter and such information should be included as
parameters. In most cases, this produces output not unlike \rtype{admincase},
the main difference that the document title has no special font applied to it
(\rtype{admincase} would apply whatever font is applied to case names).

Second, this reference type can be used without the reporter parameter, to cite
short government papers or promulgations. The traditional route is to cite such
documents as \rtype{book} references, but for one-page fact sheets or brief
policy statements this gives too much apparent authoritative weight. However,
the agency or institution name is placed in the date parenthetical, consistent
with other government or court documents, thereby emphasizing the governmental
origin of the document.

\keyparameters{
    {name}{The title of the document.}
    {author, instauth, agency, court}{The name of the agency or institution
    issuing the document. It should be abbreviated according to the
    \texttt{name} scheme (see \sec{abbrevs}). If \param{instauth} or
    \param{agency} are used, this will be done automatically.}
    {vol, rep, page, cite}{The citation locator information for the document, if
    the document was published in a reporter. The reporter should be abbreviated
    when entered.}
    {docket}{If citation locator information is not entered, a docket number may
    be given to aid in identifying the document.}
    {year}{The date of the reference.}
}
\optparameters{
    {prior, subsequent}{Procedural history associated with the document, as
    described in \sec{caseref-history}.}
    {number}{A serial number for the document, placed before the document
    title.}
}
%</doc>
%
\hi@alias@cite\deffedreg\defgovdoc
\hi@newcite\deftestimony{Testimony in a congressional hearing}{%
    \hi@citetype{#1}\hi@type@other
    \hi@ifset\hi@kv@type{}{\def\hi@kv@type{Testimony}}%
    \hi@ifset\hi@kv@citation{}{%
        \hi@ifset\hi@kv@in{\let\hi@kv@citation\hi@kv@in}{%
            \PackageError\hi@pkgname{%
                \string\deftestimony\space requires parameter `citation'%
            }{Add the parameter}%
        }%
    }%
    \hi@newcite@form{fc}{#1}{%
        \hi@citeguts{%
            \hi@kv@type\space of
            \hi@ifset\hi@kv@author{\hi@kv@author@sortable}{}%
            \hi@maybepage{ \hi@page@atorsect}%
            ,\space
            \noexpand\clause{\hi@kv@citation}\noexpand\hi@clause@endflag
        }
        \the\hi@param@parens
    }%
    \@expand{\hi@supra@form{#1}}\hi@kv@author i%
}
%<*doc>
Many congressional hearings are not published today, given the availability of
video recordings. As a result, to cite a witness's testimony at a hearing, it is
often necessary to use the witness's own copy, usually made available online.
There is no well-defined way to cite this testimony. One option, commonly used,
is to cite the privately published testimony as a \rtype{hearing} reference.
This is plainly wrong and confusing, especially if multiple witnesses'
testimonies are to be cited. Another way would be as a \rtype{book} reference,
but that obscures the testimony's legislative involvement. It is also not clear
what the title should be or where the congressional subcommittee should be
identified when taking this approach.

This package defines a new reference type for these privately published
testimony documents. It uses a \rtype{citecontainer}-like approach, with the
hearing in which the testimony was delivered being defined separately as a
reference to be used as part of the \rtype{testimony} definition. This has the
benefit that, when two different witnesses' testimonies are cited, the hearing
is cited with the short form for the second witness cited.


\keyparameters{
    {citation, in}{The \rtype{hearing} reference name or anonymous reference
    definition.}
    {author}{The witness testifying.}
}
\optparameters{
    {type}{The nature of the testimony, by default ``Testimony.''}
}
%</doc>
%
\hi@newcite\defcomments{Comments on an agency proceeding}{%
    \hi@citetype{#1}\hi@type@other
    \hi@ifset\hi@kv@type{}{\def\hi@kv@type{Comments}}%
    \hi@newcite@form{fc}{#1}{%
        \hi@citeguts{%
            \hi@ifset\hi@kv@author{\hi@kv@author@sortable, }{}%
            \hi@kv@type\space of
            \hi@kv@commenter
            \hi@maybepage{ }%
            ,\space
            \noexpand\hi@fn@arttitle{%
                \hi@kv@name
            }%
            \hi@ifset\hi@kv@rep{%
                , \hi@kv@vol\space \hi@kv@rep\space \hi@kv@page
            }{}%`
        }%
        \hi@parens@add{\hi@paren@date}{\hi@kv@court\space\hi@kv@year}%
        \the\hi@param@parens
    }%
    \protected@edef\reserved@a{%
        \noexpand\hi@fn@arttitle{Comments on \hi@kv@name}%
    }%
    \@expand{\hi@supra@form{#1}}\reserved@a i%
}
%<*doc>
This is a mildly deprecated reference type for comments filed in an agency
proceeding. (It remains here because it is used in the package author's CV\@.)
The better option is actually to use \rtype{casedoc}, defining the agency
proceeding in which the comments were filed as a separate reference
(good choices would be \rtype{admincase} and \rtype{govdoc}). In that case, the
title of the document would be given as ``Comments of \meta{party}.''

\keyparameters{
    {commenter}{Name of the entity filing the comments.}
    {name}{The title of the proceeding in which the comments were filed.}
    {vol, rep, page, cite}{Identifier information for the proceeding (typically
    a \emph{Federal Register} citation if the comments were filed pursuant to
    such a notice).}
    {year}{The date of the comments.}
}
\optparameters{
    {author}{The author of the comment, usually the signatory attorney. This
    probably should not be included in most cases, in the same way that the
    signing attorney is not identified for legal briefs in \rtype{casedoc}.}
}
%</doc>
%
\hi@newcite\defsecfiling{An SEC filing}{%
    \hi@citetype{#1}\hi@type@other
    \hi@newcite@form{fc}{#1}{%
        \hi@citeguts{%
            \hi@ifset\hi@kv@author{\hi@kv@author@sortable, }{}%
            \hi@kv@name\space
            (Form \hi@kv@number)%
            \hi@maybepage{, at }%
        }%
        \hi@parens@add{\hi@paren@date}{\hi@kv@year}%
        \the\hi@param@parens
    }%
}
%<*doc>

\keyparameters{
    {instauth}{The company filing the report.}
    {name}{The name of the report, typically along the lines of ``Annual
    Report.''}
    {number}{The number of the form filed (e.g., ``10-K'').}
    {year}{The date of filing.}
}
%</doc>
%
