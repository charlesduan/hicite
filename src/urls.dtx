%%
%% \iffalse filename: urls.dtx \fi
%%
%<*doc>
\input driver
\thisis{urls}{URLs}

Every reference type can include a URL. \hia* formats URLs, makes them clickable
links, and inserts line breaks into them. In that last respect, \hia*'s
procedure is superior to other \LaTeX\ URL packages, which fail when a
hyperlinked URL in a footnote breaks across a page, a common occurrence for law
review articles.

\subsection{Input Syntax}

The \param{url} and \param{opturl} parameters add a URL to a reference. The URL
should be entered as unformatted text, except that percent signs should be
escaped with backslashes to avoid being interpreted as comment symbols.

URLs will allow line breaks after main punctuation, similar to the package
|url.sty|. \unskip\footnote{An exception is for the percent sign, where the line
break will be inserted \emph{before} the symbol, so that URL-encoded entities
are kept intact.}
If there is a long string of alphanumeric characters, then this
package may not find a suitable breakpoint. Including \verb*|\ | (that is, a
backslashed space) in the URL will force a breakpoint to be inserted into the
URL, with no effect on appearance.

\DescribeMacro\url
The command |\url|\marg{url} can be used to format a URL in text.

%</doc>
%<*package>
%
% User command to display a URL.
%
%    \begin{macrocode}
\def\url#1{%
    \hi@parseurl{#1}\reserved@a\reserved@a
}%
%    \end{macrocode}
%
% Parses the url (\#1) and puts it into control sequence (\#2). The resulting
% URL should have appropriate breakpoints put in, and be totally unexpandable or
% protected.
%    \begin{macrocode}
\def\hi@parseurl#1#2{%
    \global\advance\hi@url@count\@ne
    \let\hi@url@cur\hi@url@count
    \hi@url@link{#1}%
    \hi@url@text{#1}%
    \hi@url@assign{#2}%
}
\def\hi@url@assign@default#1{%
    \letcs#1{hi@u\the\hi@url@cur @text}%
    \@expand\if@endswithdot{\csname hi@u\the\hi@url@cur @text\endcsname}{ii}%
        {\addto@macro#1{\protect\@hi@dottrue}}%
}
\let\hi@url@assign\hi@url@assign@default
%    \end{macrocode}
%
%
% \subsection{Data Structures}
%
% Every URL encountered will be assigned a number, which will be associated with
% data relating to the URL:
% \begin{itemize}
% \item |\hi@u#@link|: Text insertable as a PDF hyperlink declaration.
%
% \item |\hi@u#@text|: Text insertable into the body of a document, providing
% line breaking and spacing.
%
% \item |\hi@u#@declared|: Indicates that the URL has been declared for a PDF
% document.
% \end{itemize}
%
% |\hi@url@declare|\marg{num}
% inserts code declaring the URL that has been assigned number \meta{num}. A
% macro |\hi@u|\meta{num}|@link| should be defined as the properly escaped URL.
% This should be called once within the document.
%
%    \begin{macrocode}
\DeclareRobustCommand\hi@url@declare[1]{%
    \@ifundefined{hi@u#1@declared}{%
        \edef\reserved@a{%
            \noexpand\special{%
                pdf:object @HIURL#1 <<
                    /S /URI
                    /URI (\csname hi@u#1@link\endcsname)
                >>
            }%
        }\reserved@a
        \global\cslet{hi@u#1@declared}\@empty
    }{}%
}
%    \end{macrocode}
%
% |\hi@url@insert|\marg{num}\marg{text}
% inserts \meta{text} with a PDF hyperlink to the url identified by \meta{num}.
%
%    \begin{macrocode}
\DeclareRobustCommand\hi@url@insert[2]{%
    \@ifundefined{hi@u#1@declared}{\hi@url@declare{#1}}{}%
    \noexpand\special{pdf:bann <<
        /Type /Annot
        /Subtype /Link
        /Border [0 0 0]
        /A @HIURL#1
    >>}#2\noexpand\special{pdf:eann}%
}%
%    \end{macrocode}
%
% |\hi@url@insertbox|\marg{text}
% inserts \meta{text} within an |\hbox|, with a PDF hyperlink to the url
% identified by the current URL count number.
%
%    \begin{macrocode}
\DeclareRobustCommand\hi@url@insertbox[1]{%
    \hbox{\@expandarg\hi@url@insert{\the\hi@url@cur}{#1}}%
}
\newcount\hi@url@count
\newcount\hi@url@cur
\make@find@start{http://}
\make@find@start{https://}
\make@find@start{ftp://}
\make@find@start{mailto:}
%    \end{macrocode}
%
%
% \subsection{Converting to PDF-Compatible Link}
%
% Parses a URL for purposes of forming it into a link URL. This primarily strips
% the URL of backslash escape characters. The result is placed in |\hi@u#@link|,
% where the number is |\hi@url@cur|.
%
%    \begin{macrocode}
\def\hi@url@link#1{%
    \let\reserved@a\@empty
    % Force the link to start with https:// unless it has a protocol already
    \find@try\find@start{%
        {http://}\@gobble{https://}\@gobble{ftp://}\@gobble{mailto:}\@gobble
    }{#1}{\def\reserved@a{https://}}%
    \hi@url@link@{#1}%
    \global\cslet{hi@u\the\hi@url@cur @link}\reserved@a
}
\def\hi@url@link@#1{%
    \find@word{#1}%
        \hi@url@link@chars
        \hi@url@link@groups
        \hi@url@link@words
        \relax
}
\def\hi@url@link@words#1#2{%
    \addto@macro\reserved@a{#1}%
    \hi@url@link@{#2}%
}
\def\hi@url@link@groups#1#2{%
    \hi@url@link@{\{#1\}#2}%
}
\def\hi@url@link@chars#1#2{%
    \@test \ifcat\relax\noexpand#1\fi{%
        % If the char is a macro
        \@ifundefined{hi@uchar@\string#1}{%
            \expandafter\addto@macro\expandafter\reserved@a\expandafter{%
                \string#1%
            }%
        }{%
            \@expand{\addto@macro\reserved@a}{%
                \expandafter\@gobble\string#1%
            }{ii}%
        }%
    }{%
        % All other non-word chars
        \ifblank{#1}{}{% Ignore bare spaces
            \addto@macro\reserved@a{#1}%
        }%
    }%
    \hi@url@link@{#2}%
}
%    \end{macrocode}
%
%
%
% \subsection{Converting to Text}
%
%
% Formats a URL as text intended for insertion into a paragraph. The resulting
% content is stored in the macro |\hi@u#@text|, where the number is
% |\hi@url@cur|.
%    \begin{macrocode}
\def\hi@url@text#1{%
    \edef\reserved@a{\noexpand\hi@url@cur=\the\hi@url@cur \relax}%
    \hi@url@text@next{#1}\reserved@a
    \global\cslet{hi@u\the\hi@url@cur @text}\reserved@a
}

\def\hi@url@text@next#1#2{%
    \find@word{#1}{%
        \hi@url@text@chars{#2}%
    }{%
        \hi@url@text@groups{#2}%
    }{%
        \hi@url@text@words{#2}%
    }{}%
}
%    \end{macrocode}
%
% 
% \#1 is the macro for the formatted URL; \#2 is the text to process; \#3 is the
% rest of the URL.
%    \begin{macrocode}
\def\hi@url@text@words#1#2#3{%
    \addto@macro#1{\hi@url@insertbox{#2}}%
    \hi@url@text@next{#3}{#1}%
}
\def\hi@url@text@groups#1#2#3{%
    \hi@url@text@next{\{#2\}#3}{#1}%
}
%    \end{macrocode}
%
% Parse a non-word character in a URL. \#1 is the macro to which the formatted
% URL will be added; \#2 is the character, and \#3 is the rest of the URL to be
% processed.
%
%    \begin{macrocode}
\def\hi@url@text@chars#1#2#3{%
    \@test \ifcat\relax\noexpand#2\fi{%
%    \end{macrocode}
%
% If the character (\#2) is a command (e.g., |\%|), see if a definition is
% given for this command. If so, then execute the definition for the
% command (e.g., |\hi@uchar@\%|). Otherwise, treat the command like text in
% the URL and parse it along with the remainder of the URL.
%    \begin{macrocode}
        \@ifundefined{hi@uchar@\string#2}{%
            \expandafter\hi@url@text\expandafter{\string#2#3}{#1}%
        }{%
            \csname hi@uchar@\string#2\endcsname{#1}{#3}%
            \hi@url@text@next{#3}{#1}%
        }%
    }{%
%    \end{macrocode}
% If \#2 is not a command, then see if a definition is given for this
% character. If so, execute that definition. Otherwise, just add the
% character to the URL. In all cases, continue with parsing the
% remainder of the string.
%    \begin{macrocode}
        \@ifundefined{hi@uchar@\@backslashchar\string#2}{%
            \addto@macro#1{\hi@url@insertbox #2}%
        }{%
            \csname hi@uchar@\@backslashchar\string#2\endcsname{#1}{#3}%
        }%
        \hi@url@text@next{#3}{#1}%
    }%
}
%    \end{macrocode}
%
% We now define macros of the form |\hi@uchar@|\meta{char} for each character
% that will receive special processing. Each macro will take two arguments: \#1
% is the URL macro for adding content, and \#2 is the trailing text of the URL.
%
% The original lists of characters were borrowed from url.sty, by Donald
% Arsenau. At this point the code no longer resembles the original beyond
% standard TeX programming conventions, but I'd like to credit the original
% package.
%
% These characters accept a breakpoint after them.
%    \begin{macrocode}
\def\do#1{%
    \@namedef{hi@uchar@\string #1}##1##2{%
        \addto@macro##1{\hi@url@insertbox}%
        \@expand{\addto@macro##1}{\expandafter\@gobble\string #1}{ii}%
        \find@word{##2}\@gobbletwo\@gobbletwo{%
            \addto@macro##1{\hi@uchar@break}\@gobbletwo
        }{}%
    }%
}
\do\.\do\@\do\!\do\;\do\]\do\)\do\,\do\+\do\=\do\#\do\:
%    \end{macrocode}
%
% These characters accept a breakpoint \emph{and} add a little extra space.
%    \begin{macrocode}
\def\do#1{%
    \@namedef{hi@uchar@\string #1}##1##2{%
        \addto@macro##1{\hi@uchar@spacenobreak\hi@url@insertbox}%
        \@expand{\addto@macro##1}{\expandafter\@gobble\string #1}{ii}%
        \find@word{##2}{%
            \addto@macro##1{\hi@uchar@spacenobreak}\@gobbletwo
        }{%
            \addto@macro##1{\hi@uchar@spacenobreak}\@gobbletwo
        }{%
            \addto@macro##1{\hi@uchar@spacebreak}\@gobbletwo
        }{}%
    }%
}
\do\/\do\?\do\&\do\-
%    \end{macrocode}
%
% These characters accept no breakpoints, but are often backslashed; including
% them forces the removal of the backslash.
%    \begin{macrocode}
\def\do#1{%
    \@namedef{hi@uchar@\string #1}##1##2{%
        \addto@macro##1{\hi@url@insertbox}%
        \@expand{\addto@macro##1}{\expandafter\@gobble\string #1}{ii}%
    }%
}
\do\^\do\$
%    \end{macrocode}
%
% These characters require special treatment when they are seen, and a
% breakpoint is inserted if the next character is a word character.
%    \begin{macrocode}
\def\do#1#2{%
    \@namedef{hi@uchar@\string #1}##1##2{%
        \addto@macro##1{\hi@url@insertbox{#2}}%
        \find@word{##2}\@gobbletwo\@gobbletwo{%
            \addto@macro##1{\hi@uchar@break}\@gobbletwo
        }{}%
    }%
}
\do\<{$\protect\langle$}%
% There's some bug in doc.sty that requires this
\expandafter\do\csname|\endcsname{$\protect\mid$}%
\do\{{$\protect\lbrace$}%
\do\\{$\protect\backslash$}%
\do\ {}%
\do\_{\_}%
%    \end{macrocode}
%
% These characters require special treatment. The first ones are not legal
% breakpoints; the percent sign is but the breakpoint comes before the
% character. (Ideally the percent would be recognized as an entity escape,
% collecting the next two characters, but that would require modifying the
% URL parser.)
%
%    \begin{macrocode}
\def\do#1#2{%
    \@namedef{hi@uchar@\string #1}##1##2{%
        \addto@macro##1{\hi@url@insertbox{#2}}%
    }%
}
\do\>{$\protect\rangle$}
\do\}{$\protect\rbrace$}
\do\~{\char`\~}%
\@namedef{hi@uchar@\string\%}#1#2{%
    \ifstrempty{#2}{}{\addto@macro#1{\hi@uchar@spacebreak}}%
    \addto@macro#1{\hi@url@insertbox{\%}}%
}%
%    \end{macrocode}
%
% Macros for types of breakpoints and spaces.
%    \begin{macrocode}
\newskip\urlbreakskip \urlbreakskip=0pt plus .4pt
\DeclareRobustCommand\hi@uchar@break{%
    \penalty\exhyphenpenalty
}
\DeclareRobustCommand\hi@uchar@spacenobreak{%
    \unskip\unpenalty\nobreak
    \hskip\urlbreakskip
}
\DeclareRobustCommand\hi@uchar@spacebreak{%
    \penalty\exhyphenpenalty\hskip\urlbreakskip
}
%    \end{macrocode}
%
% Now actually apply commands to the character classes.
%    \begin{macrocode}
\@namedef{hi@uchar@\string\ }#1#2{%
    \ifstrempty{#2}{}{\addto@macro#1{\hi@uchar@break}}%
}
\@namedef{hi@uchar@\string\^^M}#1#2{%
    \ifstrempty{#2}{}{\addto@macro#1{\hi@uchar@break}}%
}
%    \end{macrocode}
%</package>
%<*doc>

\subsection{Alternate Presentations of URLs}

\label{s:urls-appearance}

By default, URLs are shown in text. This format, logical for print publications
with no other easy access to hyperlinks, can look lengthy and cumbersome in
legal memoranda and other contexts. Accordingly, several other options are
provided.

In documents with tables of authorities, URLs can be presented in an alternate
format: The URLs are included only in the table listing, and citations in the
document include a marker ``\emph{available online}.'' A footnote is added after
the first such omitted URL, informing the reader that omitted URLs are in the
Table of Authorities. To select this format, use the package option
\opt{toaurl}.


This format has the advantage of producing cleaner-looking briefs, particularly
given the large font-to-text-field ratios that most courts require of briefs
(such that URLs could take up a large fraction of a page). The main disadvantage
is that it actually increases word count, since URLs typically count only as one
word.

The text to be shown in place of URLs can be redefined with the macro
\DescribeMacro\ToaUrlMark
|\ToaUrlMark|.
The text to be shown in the footnote after the first such replaced URL can be
redefined with the macro
\DescribeMacro\ToaUrlText
|\ToaUrlText|.

%</doc>
%<*package>
%
% \DescribeMacro\hi@showurl
% All displays of URLs are filtered through |\hi@showurl|\marg{text}. By
% default, it just displays its argument.
%
%    \begin{macrocode}
\def\hi@urlstyle@full{%
    \def\hi@showurl##1{##1}%
}
%    \end{macrocode}
%
% Provides for URL being shown only in the table of authorities. \#1 is the
% pre-URL signal; \#2 is the URL itself.
%
%    \begin{macrocode}
\def\hi@urlstyle@toa{%
    \def\hi@showurl##1{%
        \if@hi@in@toa
            ##1%
        \else
            \ToaUrlMark
            \hi@showurl@toatext
        \fi
    }
    \def\hi@showurl@toatext{%
        \hi@defer@content{\ToaUrlText}%
        \global\let\hi@showurl@toatext\@empty
    }
    \def\ToaUrlText{%
        Locations of authorities available online are shown in the Table of
        Authorities.
    }
    \def\ToaUrlMark{\emph{available online}}
}
%    \end{macrocode}
%</package>
%<*doc>

Alternatively, in documents intended to be used only electronically, URLs can be
replaced with a word that is hyperlinked.
To select this format, use the package option \opt{linkurl}.

\DescribeMacro\LinkText
The text to be used in place of the URL is defined in the macro |\LinkText|.
No line break protection is performed on the word, so it should be an
unbreakable word or the macro should be defined to include an unbreakable
|\hbox|.

%</doc>
%<*package>
%    \begin{macrocode}
\def\hi@url@assign@link#1{%
    \edef#1{%
        \noexpand\leavevmode
        \noexpand\hi@url@insert{\the\hi@url@cur}{\noexpand\LinkText}%
        \noexpand\hi@linkurl@auxwrite{\the\hi@url@cur}%
    }%
}
\def\hi@urlstyle@link{%
    \let\hi@url@assign\hi@url@assign@link
}
\def\hi@linkurl@auxwrite#1{%
    \addtocontents{tou}{\protect\l@hi@tou{#1}{\thepage}}%
}
\def\LinkText{\emph{link}}
%    \end{macrocode}
%</package>
%<*doc>
%
% If the URL link style is selected, then a table of URLs may be generated with
% the command |\TableOfUrls|. This ensures that print copies of a document
% contain any URLs that were otherwise omitted by replacement with the
% |\LinkText| word.
%
%</doc>
%<*package>
%    \begin{macrocode}
\def\TableOfUrls{%
    \makeatletter
    \@input{\jobname.tou}%
    \makeatother
    \if@filesw
        \newwrite\tf@tou
        \immediate\openout\tf@tou \jobname.tou\relax
    \fi
}
\def\l@hi@tou#1#2{%
    \@dottedtocline{1}{0em}{1.5em}{\csname hi@u#1@text\endcsname}{#2}%
}
%    \end{macrocode}
%</package>
%<*test>

\section{urls.dtx}

\begingroup

The following should not produce overfull boxes. Also ensure that the links
themselves work as expected.

\immediate\write16{THERE SHOULD BE NO OVERFULL BOXES AFTER THIS LINE}

\def\testbox#1{%
    \vrule\vbox{%
        \hsize 1in \relax
        \url{#1}%
    }\vrule\quad
}

\testbox{http://www.google.com/search?q=hello+world}
\testbox{http://www.google.com/search?q=hello\%20world\%20how\%20are\%20you}
\testbox{http://www.google.com/search?q=antidis establishment arianism}

\immediate\write16{THERE SHOULD BE NO OVERFULL BOXES BEFORE THIS LINE}

\endgroup

%</test>
