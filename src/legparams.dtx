%%
%% \iffalse filename: legparams.dtx \fi
%%
%<*doc>
\input driver
\thisis{legparams}{Parameters for Legislative Materials}

These parameters are specific to legislative materials, including statutes and
bills.

\input docparams

%</doc>
\hi@param@alias{bill}{number}
%<*doc>
As described for \rtype{bill} references, the \param{bill} (or \param{number})
parameter should be the full identifier of the bill (e.g., |S. 123|).
%</doc>
%
\hi@defparam{chapter}{Session law chapter}{%
    \KV@hi@sesslawid{ch.}%
    \def\hi@kv@number{#1}%
}
%<*doc>
This sets parameter \param{number} to the given value, and changes parameter
\param{sesslawid} to ``ch.'' It is used for older statute references that were
organized by chapters rather than law numbers.
%</doc>
%
\hi@defparam{codified}{Citation for codification of law or regulation}{%
    \hi@defkey@codified{codified at}{#1}%
}
%<*doc>
The value should be a formatted citation (as described for parameter
\param{cite}). Typically this parameter should be given for reference types
like \rtype{statsess} or \rtype{govdoc}, but it can be given for anything.
%</doc>
%<*package>
% To make a ``codified at'' citation, we run the \param{cite} parameter command
% inside a group, and then collect the resulting values.
%    \begin{macrocode}
\def\hi@defkey@codified#1#2{%
    \begingroup
        \hi@undefine\hi@kv@year
        \KV@hi@cite{#2}%
        \edef\reserved@a{%
            #1
            \expandonce\hi@kv@vol \space
            \expandonce\hi@kv@rep \space
            \expandafter\@format@pageno\expandafter{\hi@kv@page}%
            \hi@ifset\hi@kv@year{ (\expandonce\hi@kv@year)}{}%
        }%
        \expandafter\hi@defkey@codified@\expandafter{\reserved@a}%
}
\def\hi@defkey@codified@#1{%
    \endgroup
    \def\hi@kv@codified{#1}%
}
%    \end{macrocode}
%</package>
%
\hi@defparam{codifiedamended}{Citation for codification of law or regulation}{%
    \hi@defkey@codified{codified as amended at}{#1}%
}
%<*doc>
Like \param{codified} except the text before the citation will be ``codified as
amended at.''
%</doc>
%
\hi@defparam{committee}{House or Senate committee}{%
    \hi@abbrev@leg{#1}{\def\hi@kv@committee}%
}
%<*doc>
The name will be abbreviated.
%</doc>
%
\hi@newparam{congress}{Congressional session number}
%<*doc>
This has to be a parameter separate from \param{number} because \rtype{bill}
references already use \param{number} for the bill identifier.
%</doc>
%
\hi@newparam{origsect}{Original section number of a statute in a code}
%<*doc>
This is used, for example, when a U.S. Code section corresponds to a section of
a statute with different numbering (for example, section 337 of the Tariff Act
of 1930 being 19 U.S.C. \textsection~1337). It is specific to \rtype{statcode}
references, and should be obviated with the newer reference type
\rtype{stattitle}.
%</doc>
%
\hi@param@alias{publiclaw}{number}
\hi@param@alias{publno}{number}
\hi@param@alias{sesslawid}{type}
%<*doc>
By default, this is ``Pub.~L.~No.,'' and this parameter enables changing it.
%</doc>
%
\hi@newparam{regnal}{Regnal citation for English pre-1963 statutes}
\hi@newparam{status}{Status of bill}
