%%
%% \iffalse filename: iface.dtx \fi
%%
%<*doc>
\input driver
\thisis{iface}{Citation Forms and Commands}

Depending on grammatical context, a legal citation in a work will take on one of
several forms. \hia* supports these different grammatical forms of citation by
offering several citation commands in this section.
\unskip\footnote{\hia* also produces a distinct citation form for tables of
authorities, but writers do not use this citation form directly.} This section
describes each citation form along with the command used to access the form.

%</doc>
%<*package>
%
% These variables communicate what type of citation is being produced within a
% citation command.
%
% Conditional for whether or not we are doing an inline citation. This is used
% to determine, for instance, whether to abbreviate party names.
%    \begin{macrocode}
\newif\if@hi@inline \@hi@inlinefalse
%    \end{macrocode}
%
% Conditional that is true if we're writing a sentence, rather than a clause.
%    \begin{macrocode}
\newif\if@hi@sent
%    \end{macrocode}
%
% Conditional that is false if we're doing an inline followup cite (so don't
% show the case name).
%    \begin{macrocode}
\newif\if@hi@name \@hi@nametrue
%    \end{macrocode}
%
% Conditional for if this citation is an adjective form (no leading ``the'')
%    \begin{macrocode}
\newif\if@hi@adjective
%    \end{macrocode}
%
%</package>
%<*doc>

\subsection{Sentences}

A \emph{sentence}-form citation is set off from the text as a separate
sentence:
\begin{demo}
|Segregation is unconstitutional. \sentence{see brown at 495}.| \\
\Produces Segregation is unconstitutional. \emph{See} \emph{Brown v.~Bd.\ of
Educ.}, 347 U.S. 483 (1954).
\end{demo}

\DescribeMacro\sentence
As seen above, |\sentence| produces this style of citation.
The trailing period
is mandatory; it avoids mistakes and assists word processors' detection of
missed capitalization. In legal memoranda, the command may appear anywhere in a
document. In law review articles, |\sentence| must appear in a
footnote, and if it appears in body text, it will be automatically converted to
|\note|. However, a sentence-style citation at the beginning of a paragraph in
body text (typically used to cite a block quotation) cannot be automatically
converted, so a warning is generated.

%</doc>
%<*package>
%
% Law reviews and legal memoranda treat sentence citations in body text slightly
% differently, so this code defines |\sentence| depending on the document type.
%
%    \begin{macrocode}
\appto\hi@hooks@review{
    \def\sentence#1{%
        \@test\if@hi@note\fi{%
            \hi@sentence{#1}%
        }{%
            \ifhmode\else
                \hi@sentence@review@warn
                \leavevmode
            \fi
            \unskip
            \footnote{\sentence{#1}.}%
            \@gobble@dot
        }%
    }
    \def\hi@sentence@review@warn{%
        \PackageWarning\hi@pkgname{%
            A sentence citation was used in text, at the beginning\MessageBreak
            of a paragraph; it needs to be moved to the end of the\MessageBreak
            previous paragraph.\MessageBreak
            This occurred after footnote \the\c@footnote%
        }%
    }
}
\appto\hi@hooks@memo{
    \def\sentence{\hi@sentence}%
}
%    \end{macrocode}
%
% For sentence-style citations, flags are set indicating the style and that the
% first letter should be capitalized. The only trailing punctuation check is
% whether the citation had a dot at the end. If not, then an additional period
% to end the sentence must be added. In any event, any deferred footnote is
% placed after the trailing period.
%
%    \begin{macrocode}
\def\hi@sentence#1{%
    \hi@pse@parse{#1}{\hi@draw@citation{\@hi@captrue\@hi@senttrue}{}}%
    \futurelet\@let@token\hi@end@sentence
}
\def\hi@end@sentence{%
    \ifdim\lastkern>\z@ \unkern \fi
    \@test \ifx\@let@token.\fi{%
        \if@hi@dot
            \@re@dot@space
        \else
            .%
        \fi
        \hi@deferred@note@endstring
        \@gobble
    }{%
        \PackageWarning\hi@pkgname{%
            Sentence citation did not end with period.\MessageBreak
            This occurred%
        }%
        % Add the period
        .\hi@deferred@note@endstring\space
        \expandafter\ignorespaces
    }%
}
%    \end{macrocode}
%
%</package>
%<*doc>





\subsection{Clauses, Instructions, and Nested Citations}


A \emph{clause}-form citation falls within a sentence, preceded by a comma and
followed by a comma unless there is other punctuation.
Typically clause-style citations are used when different citations are needed
for different parts of a single long sentence.
\begin{demo}
|Education is important, \clause{see brown at 493}, and segregation has a ``detrimental effect,'' \clause{brown at 494}.| \\
\Produces Education is important, \emph{see} \emph{Brown}, 347 U.S. at 493, and
segregation has a ``detrimental effect,'' \emph{id.}\ at 494.
\end{demo}

\DescribeMacro\clause
The |\clause| command makes a clause-form citation. It should be surrounded by
appropriate punctuation, and in particular there must be punctuation immediately
following the command. In law review articles,
|\clause| must appear in a footnote, and if it appears in body text, it will be
automatically converted to |\note|. However, a clause citation followed by
punctuation that is not a comma cannot be automatically converted, so a warning
is generated.

\paragraph{Instructional Citations}
Citations may also appear in instructional sentences that end with a |\clause|
citation. Typically these will be imperative sentences with an unitalicized
spelled-out signal:
\begin{demo}
|On scientific evidence, see \clause{brown at 494 note 11}.| \\
\Produces On scientific evidence, see \emph{Brown}, 347 U.S. at 494 n.11.
\end{demo}
Although this context differs grammatically from other |\clause| citations, the
formatting is largely identical so only one command is provided.
\unskip\footnote{Some style guides use a slightly different format in
which citation items are separated with commas and the word ``and'' rather than
semicolons. This is a somewhat rare usage that could be accomplished by
issuing separate \cmd\clause\ commands for each citation item, so there seems to
be no particular need to implement a separate command for this use case.}

\paragraph{Nested Citations}
Citations may also appear nested in parentheticals within other citations. This
can be accomplished by using |\clause| inside the parenthetical:
\begin{demo}
|\sentence{see brown at 494-495 (overruling \clause{plessy})}.| \\
\Produces \emph{See} \emph{Brown}, 347 U.S. at 494--95 (overruling \emph{Plessy
v.~Ferguson}, 163 U.S. 537 (1896)).
\end{demo}
More on nested citations, including a more convenient shortcut
form, may be found in \sec{parens}.

%</doc>
%<*package>
%
%    \begin{macrocode}
\appto\hi@hooks@review{
    \def\clause#1{%
        \@test\if@hi@note\fi{%
            \hi@clause{#1}%
        }{%
            \unskip
            \footnote{\sentence{#1}.}%
            \futurelet\@let@token\hi@clause@review@end
        }%
    }
    \def\hi@clause@review@end{%
        \ifx\@let@token,
            \expandafter\@gobble
        \else
            \hi@clause@review@warn
        \fi
    }%
    \def\hi@clause@review@warn{%
        \PackageWarning\hi@pkgname{%
            A clause citation was used in text, but its ending\MessageBreak
            punctuation could not be reconciled automatically.\MessageBreak
            This occurred at footnote \the\c@footnote%
        }%
    }
}
\appto\hi@hooks@memo{
    \def\clause{\hi@clause}
}
%
% For clauses, no capitalization at the beginning of the sentence is requested.
% Trailing punctuation handling is especially complicated in view of deferred
% footnotes, because footnotes are to be placed after some punctuation but
% before others. As a result, the end of the clause citation tests many
% different types of punctuation for proper handling.
%
%    \begin{macrocode}
\def\hi@clause#1{%
    \hi@pse@parse{#1}{\hi@draw@citation{\@hi@capfalse\@hi@sentfalse}{}}%
    \futurelet\@let@token\hi@end@clause
}
%    \end{macrocode}
%
% This macro is not intended for execution, but is placed at the end of
% parentheticals. That way, when |\clause| is used as the last element of a
% parenthetical, we can give it special treatment to deal with trailing
% punctuation.
%
%    \begin{macrocode}
\def\hi@clause@endflag{\iffalse\hi@clause@endflag\fi}
%    \end{macrocode}
%
% This is how a clause deals with ending punctuation.
%
%    \begin{macrocode}
\def\hi@end@clause{%
%    \end{macrocode}
%
% If the clause ended with a dot and it is the terminator of a
% parenthetical, then leave |\@hi@dottrue| on. Otherwise reset to
% |\@hi@dotfalse| (deferred if the clause is immediately followed by a dot).
% This is necessary for dot management in comma-type parentheticals; see
% |\hi@paren@show|.
%
%    \begin{macrocode}
    \if@hi@dot
        \ifx\@let@token\hi@clause@endflag \else
            \ifx\@let@token.\else
                \@hi@dotfalse
            \fi
        \fi
    \fi
%    \end{macrocode}
%
% Now we test each type of trailing punctuation, considering whether to put any
% deferred footnote before or after it.
%
%    \begin{macrocode}
    \@testcase
    \ifx\@let@token,\fi{%
        \ifdim\lastkern>\z@ \unkern \fi
        ,\hi@deferred@note@endstring\@gobble
    }%
    \expandafter\ifx\space\@let@token\fi{%
        \PackageWarning\hi@pkgname{%
            \string\clause\space not terminated with punctuation%
        }%
        \hi@deferred@note@endstring
    }%
    \ifx\@let@token.\fi{%
        \ifdim\lastkern>\z@ \unkern \fi
        \if@hi@dot
            \@hi@dotfalse % For parenthetical endings
            \@re@dot@space
        \else
            .%
        \fi
        \hi@deferred@note@endstring\@gobble
    }%
    \ifx\@let@token\hi@clause@endflag\fi{\hi@deferred@note@endstring\@gobble}%
    \ifx\@let@token;\fi{\@let@token\hi@deferred@note@endstring\@gobble}%
    \ifx\@let@token:\fi{\@let@token\hi@deferred@note@endstring\@gobble}%
    \ifx\@let@token!\fi{\@let@token\hi@deferred@note@endstring\@gobble}%
    \ifx\@let@token?\fi{\@let@token\hi@deferred@note@endstring\@gobble}%
    \ifx\@let@token-\fi{\hi@deferred@note@endstring}%
    \ifx\@let@token)\fi{\hi@deferred@note@endstring}%
    \default{%
        \PackageWarning\hi@pkgname{%
            \string\clause\space not terminated with punctuation%
        }%
        \hi@deferred@note@endstring
    }%
}
%    \end{macrocode}
%
%</package>
%<*doc>




\subsection{Inline Citations}


An \emph{inline}-form citation acts as a part of speech within a sentence,
typically a noun or an adjective. Naming a case in text uses an
inline citation:
\begin{demo}
|\Inline{brown} followed \inline{sweatt}.| \\
\Produces \emph{Brown} followed \emph{Sweatt v.~Painter}.
\end{demo}

The following commands produce inline citations. Any of them may be used in text
or footnotes, regardless of the document type.
\begin{itemize}
\item \DescribeMacro\inline |\inline| makes a typical inline citation.

\item \DescribeMacro\Inline |\Inline| capitalizes the first letter of the
citation.

\item \DescribeMacro\adjective |\adjective| removes leading ``the'' from the
citation text, making it amenable to use as an adjective in a sentence.

\item \DescribeMacro\Adjective |\Adjective| is like |\adjective| but capitalizes
the first letter.
\end{itemize}

\hia*'s support for inline citations is the most comprehensive for cases and
statutes, where there are reasonably well-defined conventions for how to form
such citations. For other reference types, such as books and articles, the
package uses author names to the extent given. Support for more complete
coverage of inline citations is a potential future improvement to the package.

Using inline citation commands is preferable to manually entering inline
citations in text for several reasons. First, inline
citations often have long and short forms, so programmatic formatting of inline
citations ensures that the right one is chosen even if paragraphs or sections of
text are moved around.
Furthermore, inline citations can affect the surrounding non-inline citations.
When citing a case for the first time, it is typical to omit the case name from
a sentence citation when the case name was used inline:
\begin{demo}
|\Inline{brown} involved class actions. \sentence{see brown at 495}.| \\
\Produces \emph{Brown} involved class actions. \emph{See} 347 U.S. at 495.
\end{demo}
This package detects such situations and removes duplicative names
automatically.


\paragraph{Inline Citations and \emph{Id.}}\label{s:iface-invisible}
One of the most nuanced considerations for inline citations is the effect on
\emph{id.}\ short forms. Consider the following:
\begin{demo}
|\sentence{brown at 496}. The Court overruled \inline{plessy}. \sentence{brown at 494-495}.| \\
\Produces \emph{Brown}, 347 U.S. at 496. The Court overruled \emph{Plessy}.
\textbf{???}
\end{demo}
Absent the inline citation to \emph{Plessy}, the final citation in the above
sentence should be ``\emph{Id.}\ at 494--95.'' But with the inline citation, it
is ambiguous which case the \emph{Id.}\ refers to. The better option, which
\hia* follows, is to use the named short form.

If this is undesirable, one can cause the \emph{Id.}\ form to be used by hiding
the inline citation with braces:
\begin{demo}
|\sentence{brown at 496}. The Court overruled {\inline{plessy}}. \sentence{brown at 494-495}.| \\
\Produces \emph{Brown}, 347 U.S. at 496. The Court overruled \emph{Plessy}.
\emph{Id.}\ at 494--95.
\end{demo}

By contrast, consider:
\begin{demo}
|\sentence{brown at 496}. The Court applied \inline{14th-amend}. \sentence{brown at 494}.|
\Produces \emph{Brown}, 347 U.S. at 496. The Court applied the Fourteenth
Amendment. \emph{Id.}~at 494.
\end{demo}
Here, it is unambiguous that the \emph{Id.}\ does not refer to the constitution.
\hia* implements this distinction by designating the inline forms of statutes,
constitutions, and a few other references as ``invisible'' to the
\emph{id.}~citation algorithm. This behavior can be overridden by using the
|\noid| command.



%</doc>
%<*package>
%
% User commands for accessing inline citations.
%    \begin{macrocode}
\def\inline#1{\hi@inline{#1}{\@hi@capfalse\@hi@sentfalse}}
\def\Inline#1{\hi@inline{#1}{\@hi@captrue\@hi@senttrue}}
\def\adjective#1{\hi@inline{#1}{\@hi@capfalse\@hi@sentfalse\@hi@adjectivetrue}}
\def\Adjective#1{\hi@inline{#1}{\@hi@captrue\@hi@senttrue\@hi@adjectivetrue}}
%    \end{macrocode}
%
% This command may be used in reference definition macros to render inlines
% invisible as described above.
%
%    \begin{macrocode}
\newif\if@hi@invis@inline
\DeclareRobustCommand\hi@invis@inline{\if@hi@inline \@hi@invis@inlinetrue \fi}
%    \end{macrocode}
%
% Inline citations require a significant amount of state management because of
% how they affect surrounding citations. The overall strategy is to save the
% state to the extent possible, produce the citation, and then selectively
% restore state with appropriate changes (unless the citation was invisible, in
% which case all state is restored).
%
%    \begin{macrocode}
\def\hi@inline#1#2{%
    \let\hi@inline@lastcase\@last@case
    \hi@state@save\hi@inline@restorestate
    \@hi@invis@inlinefalse
    \@hi@inlinetrue
    \hi@pse@parse{#1}{\hi@draw@citation{#2}{}}%
    \@hi@inlinefalse
    \@hi@adjectivefalse
    \if@hi@invis@inline
        \hi@inline@restorestate
    \else
        \def\hi@inline@setlast{\def\@last@inline}%
        \expandafter\addto@macro\expandafter\hi@inline@setlast\expandafter{%
            \expandafter{\@last@case}%
        }%
        \hi@inline@post
        \hi@inline@setlast
    \fi
    \futurelet\@let@token\hi@end@inline
}
%    \end{macrocode}
%
% In a law review article, non-invisible inline citations inside a footnote
% restore state only when the inline citation is identical to the last citation.
% Otherwise the state is reset. Outside a footnote, state is always restored
% because anything written in text does not affect the footnotes.
%
% For legal memoranda, state is restored only when the inline citation is
% identical to the last citation.
%
%    \begin{macrocode}
\appto\hi@hooks@review{
    \def\hi@inline@post{%
        \if@hi@note
            \ifx\hi@inline@lastcase\@last@case
                \hi@inline@restorestate
            \else
                \hi@state@reset
            \fi
        \else
            \hi@inline@restorestate
        \fi
    }
}
\appto\hi@hooks@memo{
    \def\hi@inline@post{%
        \ifx\hi@inline@lastcase\@last@case
            \hi@inline@restorestate
        \else
            \hi@state@reset
        \fi
    }
}
%    \end{macrocode}
%
% Dealing with trailing punctuation.
%
%    \begin{macrocode}
\def\hi@end@inline{%
    \@testcase
    \ifx\@let@token.\fi{%
        \unkern
        %
        % Set \@hi@dotfalse only after determining whether to gobble the dot.
        %
        \expandafter\@hi@dotfalse
        \if@hi@dot
            \@re@dot@space\expandafter\@gobble
        \fi
    }%
    \ifx\@let@token,\fi{\@hi@dotfalse\unkern}%
    \ifx\@let@token\hi@clause@endflag\fi{}% Leave \@hi@dottrue alone
    \default{\@hi@dotfalse}%
}
%    \end{macrocode}
%
% Decides whether case names should be omitted after an inline citation to a
% case. The rule followed here is that the name is omitted after a matching
% inline cite except when the case is the first citation in a footnote.
%
%    \begin{macrocode}
\def\hi@noname@sometimes{
    \def\hi@noname{%
        \if@hi@note
            \ifnum\hi@note@cites>\z@ \@hi@namefalse \fi
        \else
            \@hi@namefalse
        \fi
    }
}
%    \end{macrocode}
%
% Alternatively, the name can always be suppressed when the appropriate option
% is set.
%
%    \begin{macrocode}
\def\hi@noname@always{
    \def\hi@noname{\@hi@namefalse}
}
%    \end{macrocode}
%</package>
%<*doc>



\subsection{Note}

\DescribeMacro\note |\note| makes a sentence-style citation inside a footnote;
it is essentially a synonym for:
\begin{demo}
|\unskip\footnote{\sentence{|\meta{string}|}.}|
\end{demo}
It provides two conveniences for the author. First, the |\unskip| means that a
space before |\note| is automatically removed, making \TeX\ code more readable.
Second, |\note| can take a second argument of text (which can include sentence
and clause citations) appended after the citation:
\begin{demo}
|\note{case}{For further information, see \clause{case2}.}|
\end{demo}

%</doc>
%<*package>
%
% The note command first checks if it is inside a footnote already; if so it
% converts to a sentence citation. It also performs a variety of trailing
% punctuation checks.
%
%    \begin{macrocode}
\def\note#1{%
    \if@hi@note
        \PackageWarning\hi@pkgname{%
            Cannot create footnote citation inside a footnote%
        }%
        \sentence{#1}.\space\expandafter\ignorespaces
    \else
        \unskip
        \def\hi@note@content{\sentence{#1}.}\expandafter\hi@note@post
    \fi
}%
\def\hi@note@post{\futurelet\hi@note@token\hi@note@post@}
\def\hi@note@post@switch{%
    {\hiendnotelabel}{}%
    {\hinotelabel}{}%
    {\egroup}{}%
    { }{}%
    {-}{}%
    {)}{}%
    {,}{\hi@note@post@warn \spacefactor\sfcode`\,\relax \@gobble}%
    {.}{\hi@note@post@warn \spacefactor\sfcode`\.\relax \@gobble}%
}
\def\hi@note@post@warn{%
    \PackageWarning\hi@pkgname{%
        Unexpected character after \string\note: \meaning\hi@note@token
    }%
}
\def\hi@note@post@{%
    \@test\ifx\hi@note@token\bgroup\fi{%
        \hi@note@post@more
    }{%
        \expandafter\footnote\expandafter{\hi@note@content}%
        \expandafter\hi@note@post@@\hi@note@post@switch\@stop
    }%
}
\def\hi@note@post@more#1{%
    \addto@macro\hi@note@content{ #1}%
    \hi@note@post
}
\def\hi@note@post@@#1#2#3\@stop{%
    \@test\ifx\hi@note@token#1\fi{%
        #2%
    }{%
        \ifstrempty{#3}{%
            \hi@note@post@warn
        }{%
            \hi@note@post@@#3\@stop
        }%
    }%
}
%    \end{macrocode}
%</package>
%<*doc>



\subsection{Optional Clause}



\DescribeMacro\optclause
The |\optclause| command is placed immediately after an |\inline| citation and
takes no arguments. Its purpose is to serve a style of citation that some
writers prefer, in which a clause-style citation (in text or in a footnote,
depending on the type of document) immediately follows the first inline citation
to a reference:
\begin{demo}
|In \inline{brown}\optclause, the Court held...| \\
\Produces
\emph{[When case already cited]} In \emph{Brown}, the Court held\ldots \\
\emph{[When case not already cited]} In \emph{Brown v.~Board of Education}, 347
U.S. 483 (1954), the Court held\ldots
\end{demo}
In other words, this is identical to |\inline{ref}, \clause{ref at 570},| if the
reference has not been cited so far, but produces nothing if it has been cited
already.
\unskip\footnote{Personally, I do not like this style of citation, as it breaks
the rule that a citation comes after the proposition being cited. But it is a
common enough style that I provide it as an option.}

%</doc>
%<*package>

%    \begin{macrocode}
\appto\hi@hooks@review{
    \def\optclause{%
        \@test\if@hi@note\fi{%
            \hi@optclause@memo
        }{%
            \hi@record@choose@cite{\@last@inline}{%
                \futurelet\@let@token\hi@optclause@with
            }{%
                \futurelet\@let@token\hi@optclause@without
            }%
        }%
    }
    \def\hi@optclause@without{%
        \ifcat a\@let@token\space \fi
    }
    \def\hi@optclause@with{%
        \@test\ifx\ \@let@token\fi{%
            \footnote{\sentence{*}.}\ %
        }{%
            \@test\ifcat\space\@let@token\fi{%
                \footnote{\sentence{*}.}%
            }{%
                \@test\ifcat a\@let@token\fi{%
                    \footnote{\sentence{*}.}\space
                }{%
                    \hi@optclause@hop
                }%
            }%
        }%
    }
    \def\hi@optclause@hop#1{%
        #1\futurelet\@let@token\hi@optclause@with
    }
}
\appto\hi@hooks@memo{
    %
    % Adds a clause for *, with no pin cite, unless the case is being
    % short-cited.
    \def\optclause{\hi@optclause@memo}
}
\def\hi@optclause@memo{%
    \hi@record@choose@cite{\@last@inline}{%
        \unkern, \clause{*}\hi@clause@endflag
        \futurelet\@let@token\hi@optclause@endwith
    }{\futurelet\@let@token\hi@optclause@endwithout}%
}
\def\hi@optclause@endwith{%
    \ifx\ \@let@token,\else
        \ifcat a\@let@token,\space \fi
    \fi
}
\def\hi@optclause@endwithout{%
    \ifx\@let@token,\unkern\else
        \ifx\@let@token.\unkern\else
            \ifcat a\@let@token\space \fi
        \fi
    \fi
}
%    \end{macrocode}
%
%
%
%</package>
%<*doc>


\subsection{Miscellaneous Commands}

A few additional commands are defined here.

\DescribeMacro\textellipsis
Ellipses in legal documents are idiosyncratic, using full spaces between dots.
The package redefines ellipses in \LaTeX\ to conform.

%</doc>
%<*package>
%    \begin{macrocode}
\AtBeginDocument{%
    \def\textellipsis{%
        \ifhmode\unskip\@\hi@ellipsis@kern\fi
        .\hi@ellipsis@kern
        .\hi@ellipsis@kern
        .\hbox{\vrule \@width 0pt}% Looks better with microtype
        \futurelet\@let@token\hi@ellipsis@end
    }%
}
%    \end{macrocode}
%
% The |times| package used to give me problems---I don't know if it is still
% problematic.
%    \begin{macrocode}
\@ifpackageloaded{times}{%
    \def\hi@ellipsis@kern{\kern\fontdimen3\font}%
    \def\hi@ellipsis@space{\hskip\fontdimen3\font\relax}%
}{%
    \let\hi@ellipsis@kern~%
    \let\hi@ellipsis@space\space
}
%    \end{macrocode}
%
% At the end of an ellipsis, test if the next character is a letter, for
% purposes of adjusting the |\spacefactor|.
%
%    \begin{macrocode}
\def\hi@ellipsis@end{%
    \@test\ifx\@let@token\fi{%
        \futurelet\@let@token\hi@ellipsis@end
    }{%
        \@test\ifcat A\noexpand\@let@token\fi{%
            \@\hi@ellipsis@space
        }{%
            \hi@ellipsis@kern
            \spacefactor999\relax
        }%
    }%
}
%    \end{macrocode}
%</package>
%<*doc>

\DescribeMacro{\sic}
The |\sic!| command inserts a properly formatted [\emph{sic}] into text. The
exclamation point deals with \TeX\ spacing issues.

%</doc>
%<*package>
%    \begin{macrocode}
\def\sic!{[\emph{sic}]}
%    \end{macrocode}
%</package>
%<*doc>

Several macros affect memory of citation state to change what forms of citations
will be produced.
\begin{itemize}
\item
\DescribeMacro{\noid}
The |\noid| command informs the package not to use \emph{id.}\ for the next
citation, by resetting memory of the last reference cited.

\item
\DescribeMacro\fullcite
Similarly, the |\fullcite|\marg{ref} command informs the package to produce a
full citation of \meta{ref} by resetting memory of that reference's use.

\item
\DescribeMacro\nofullcite
By contrast, the |\nofullcite|\marg{ref} effectively cites \meta{ref} without
displaying anything, so that the next citation to that reference is a short
form.

\item
\DescribeMacro\fulltitle
The |\fulltitle|\marg{ref} command resets memory of the last statutory title
cited.

\item
\DescribeMacro\addtotoa
The |\addtotoa|\marg{ref} command adds \meta{ref} to the table of authorities
without producing any citation text.
\end{itemize}
%
%</doc>
%<*package>
%    \begin{macrocode}
\def\hi@noid{%
    \let\@last@case\relax\let\@last@page\relax\let\@last@vol\relax
    \let\@last@title\relax
}
\let\noid\hi@noid
\def\fullcite#1{%
    \expandafter\let\csname dfc@#1\endcsname\relax
    \expandafter\let\csname dfi@#1\endcsname\relax
    \noid
}
\def\nofullcite#1{\hi@record@cite{#1}}
\def\fulltitle{\let\@this@title\relax}
\def\addtotoa#1#2{\@bsphack\hi@addtotoa{#1}{#2}\@esphack}
%    \end{macrocode}
%</package>
%<*doc>

\subsection{Shorter Commands}

The following are synonyms to the citation commands thus described.

\begin{demo}
\begin{tabular}{ll}
\textbf{Command} & \textbf{Synonym} \\
\hline
\cmd\sentence & \DescribeMacro\Ct \cmd\Ct \\
\cmd\clause & \DescribeMacro\ct \cmd\ct \\
\cmd\Inline & \DescribeMacro\Ci\cmd\Ci \\
\cmd\inline & \DescribeMacro\ci\cmd\ci \\
\cmd\note & \DescribeMacro\cn\cmd\cn \\
\cmd\addtotoa & \DescribeMacro\cx\cmd\cx \\
\end{tabular}
\end{demo}
%</doc>
%<*package>
\newcommand\Ct{\sentence}
\newcommand\ct{\clause}
\newcommand\Ci{\Inline}
\newcommand\ci{\inline}
\newcommand\cn{\note}
\newcommand\cx{\addtotoa}
%</package>
%
%
